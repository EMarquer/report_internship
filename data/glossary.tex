%\newacronym{svm}{SVM}{support vector machine}
%\newacronym[description={description}]{label}{short}}{long}}
%sort={empty set}, long, longplural, plural
%to use it you call \gls{svm}, \glspl{}, \Gls, \Glspl, \acrlong{label}, 
%\acrfull{label}

%\newglossaryentry{potato}{name={potato},plural={potatoes}, 
%	description={starchy tuber}}

%\newglossaryentry{dl}{name={DL},plural={potatoes}, 
%	description={starchy tuber}}
\newglossary[eng]{entity}{end}{edt}{Entités, Instituts et ???} % TODO need better name

\makenoidxglossaries

%%%%%%%%%%%%%%%%%%%%%%%%%%%%%%%%%%%%%%%%%%%%%%%%%%%%%%%%%%%%%%%%%%%%%%%%%%%%%%%%
%   Institues, Entities                                                        %
%%%%%%%%%%%%%%%%%%%%%%%%%%%%%%%%%%%%%%%%%%%%%%%%%%%%%%%%%%%%%%%%%%%%%%%%%%%%%%%%
\newglossaryentry{loria}{
	type=entity,
	name={LORIA},
	description={Laboratoire Lorrain d'Informatique et ses Applications},
	first={Laboratoire Lorrain d'Informatique et ses Applications (LORIA)}}
\newglossaryentry{cnrs}{
	type=entity,
	name={CNRS},
	description={Centre National de la Recherche Scientifique},
	first={Centre National de la Recherche Scientifique (CNRS)}}
\newglossaryentry{inria}{
	type=entity,
	name={INRIA},
	description={Institut National de Recherche en Informatique et en Automatique},
	first={Institut National de Recherche en Informatique et en Automatique (INRIA)}}
\newglossaryentry{ul}{
	type=entity,
	name={UL},
	description={Université de Lorraine},
	first={Université de Lorraine (UL)}}
\newglossaryentry{synalp}{
	type=entity,
	name={SYNALP},
	description={},
	first={SyNaLP (Symbolic and statistical NLP)}}
\newglossaryentry{papud}{
	type=entity,
	name={PAPUD},
	description={},
	first={PAPUD (???)}}

%%%%%%%%%%%%%%%%%%%%%%%%%%%%%%%%%%%%%%%%%%%%%%%%%%%%%%%%%%%%%%%%%%%%%%%%%%%%%%%%
%   Machine Learning                                                           %
%%%%%%%%%%%%%%%%%%%%%%%%%%%%%%%%%%%%%%%%%%%%%%%%%%%%%%%%%%%%%%%%%%%%%%%%%%%%%%%%
\newglosacr{ml}{ML}
{%
	L'Apprentissage Automatique (\foreign{Machine Learning} en anglais) est un 
	ensemble de "méthodes [statistiques] permettant à une machine (au sens 
	large) d'évoluer par un processus systématique, et ainsi de remplir des 
	tâches difficiles ou problématiques par des moyens algorithmiques plus 
	classiques". \cite{wikiml} 
	%https://fr.wikipedia.org/wiki/Apprentissage_automatique
	%TODO describe Machine Learning
}{Apprentissage Profond (\foreign{Machine Learning} en anglais, ML)}

\newglosacrsee{dl}{DL}
{%
	L'Apprentissage Profond (\foreign{Deep Learning} en anglais) est 
	une méthode de Machine Learning utilisant la technique des réseaux de 
	neurones. %https://fr.wikipedia.org/wiki/Apprentissage_profond
	%TODO describe DL
}{Apprentissage Profond (\foreign{Deep Learning} en anglais, DL)}
{ml}

%%%%%%%%%%%%%%%%%%%%%%%%%%%%%%%%%%%%%%%%%%%%%%%%%%%%%%%%%%%%%%%%%%%%%%%%%%%%%%%%
%   Neural Networks                                                            %
%%%%%%%%%%%%%%%%%%%%%%%%%%%%%%%%%%%%%%%%%%%%%%%%%%%%%%%%%%%%%%%%%%%%%%%%%%%%%%%%
\newglosacr{nn}{NN}
{%
	Un Réseau de Neurones Artificiels (\foreign{Neural Network} en 
	anglais) est %TODO describe NN %\cite{wikinn}
}{Réseau de Neurones Artificiels (\foreign{Neural Network} en anglais, NN)}

\newglosacrsee{rnn}{RNN}
{%
	Un Réseau de Neurones Artificiels Récurrents, plus 
	simplement Réseau de Neurones Récurrents (\foreign{Recurrent Neural 
	Network} en anglais) est un réseau de neurones artificiels suivant une 
	architecture dite récurrente.
	\newline
	Ce genre de réseau est utilisé pour travailler avec des séquences d'entrées 
	et/ou de sorties; il y a transmission d'information entre chaque élément de 
	la séquence. \cite{wikirnn}
	%TODO describe RNN
}{Réseau de Neurones Récurrents (\foreign{Recurrent Neural Network} en anglais, RNN)}
{nn}

\newglosacrsee{msnn}{MSNN}
{%
	%TODO describe MSNN
	%https://en.wikipedia.org/wiki/Language_model
}{Réseau de Neurones Récurrents Multi-Échelles (\foreign{Multi-Scale Recurrent Neural Network} en anglais, MSNN)}
{rnn}

\newglosacrsee{gmsnn}{GMSNN}
{%
	(le nombre de couches augmente selon le nombre d'entrées)
	%TODO describe GMSNN
	%https://en.wikipedia.org/wiki/Language_model
}{Réseau de Neurones Récurrents Multi-Échelles Croissant (\foreign{Growing Multi-Scale Recurrent Neural Network} en anglais, GMSNN)}
{msnn}


%%%%%%%%%%%%%%%%%%%%%%%%%%%%%%%%%%%%%%%%%%%%%%%%%%%%%%%%%%%%%%%%%%%%%%%%%%%%%%%%
%   Language Models                                                            %
%%%%%%%%%%%%%%%%%%%%%%%%%%%%%%%%%%%%%%%%%%%%%%%%%%%%%%%%%%%%%%%%%%%%%%%%%%%%%%%%
\newglosacr{lm}{LM}
{%
	Un Modèle de la Langue (\foreign{Language Model} en anglais) est une "distribution de probabilité sur une séquence de mots [ou de caractères]", 
	utilisé pour estimer la probabilité d'apparition du prochain mot.
	Autrement dit, c'est une représentation servant à prédire le prochain mot à partir des mots précédents. \cite{wikilm}
	%TODO describe LM
	%https://en.wikipedia.org/wiki/Language_model
}{Modèle de la Langue (\foreign{Language Model} en anglais, LM)}

\newglosacrsee{char-lm}{Char-LM}
{%
	{Un Modèle de la Langue au niveau du Caractère (\foreign{Character-level Language Model} en anglais) est un Modèle de la Langue qui prédit non pas le prochain mot à partir des mots précédents, mais le prochain caractère à partir des caractères précédents}
	%TODO describe LM
}{Modèle de la Langue au niveau du Caractère (\foreign{Character-level Language Model} en anglais, Char-LM)}
{lm}

\newglosacrsee{char-gmsnn-lm}{Char-GMSNN-LM}
{%
	Un Modèle du Langage au niveau du Caractère basé sur un Réseau de Neurones 
	Multi-Échelles Croissant. C'est un modèle du langage basé sur un réseau de 
	neurones artificiels. Ce réseau de neurones est un GMSNN utilisé comme Char-LM.}
{Modèle du Langage au niveau du Caractère basé sur un Réseau de Neurones Multi-Échelles Croissant (\foreign{Character-level Growing Multi-Scale Language Model} en anglais, Char-LM)}
{char-lm,gmsnn}

%%%%%%%%%%%%%%%%%%%%%%%%%%%%%%%%%%%%%%%%%%%%%%%%%%%%%%%%%%%%%%%%%%%%%%%%%%%%%%%%
%   Other technical expressions                                                %
%%%%%%%%%%%%%%%%%%%%%%%%%%%%%%%%%%%%%%%%%%%%%%%%%%%%%%%%%%%%%%%%%%%%%%%%%%%%%%%%
\newglossaryentry{methodes statistiques}{
	name={méthodes statistiques},
	description={}}
\newglossaryentry{matrice}{
name={matrice},
plural={matrices},
description={}
}
\newglosacrsee{gpu}{GPU}
{%
	Un processeur graphique (\foreign{Graphical Processing Unit} en anglais) est un composant d'ordinateur spécialisé, qui montre d'excellentes performances dans les calculs impliquant des matrices (ex. : images)\glsadd{matrice}
}{processeur graphique (\foreign{Graphical Processing Unit} en anglais, GPU)}
{matrice}
