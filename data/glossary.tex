%\newacronym{svm}{SVM}{support vector machine}
%\newacronym[description={description}]{label}{short}}{long}}
%sort={empty set}, long, longplural, plural
%to use it you call \gls{svm}, \glspl{}, \Gls, \Glspl, \acrlong{label}, 
%\acrfull{label}

%\newglossaryentry{potato}{name={potato},plural={potatoes}, 
%	description={starchy tuber}}

%\newglossaryentry{dl}{name={DL},plural={potatoes}, 
%	description={starchy tuber}}
\newglossary[eng]{entity}{end}{edt}{Entités, projets et sigles} % TODO need better name

\makenoidxglossaries

%\makeglossaries

%%%%%%%%%%%%%%%%%%%%%%%%%%%%%%%%%%%%%%%%%%%%%%%%%%%%%%%%%%%%%%%%%%%%%%%%%%%%%%%%
%   Institues, Entities                                                        %
%%%%%%%%%%%%%%%%%%%%%%%%%%%%%%%%%%%%%%%%%%%%%%%%%%%%%%%%%%%%%%%%%%%%%%%%%%%%%%%%
\newglossaryentry{loria}{
	type=entity,
	name={LORIA},
	text={LORIA},
	description={Laboratoire Lorrain d'Informatique et ses Applications},
	first={Laboratoire Lorrain d'Informatique et ses Applications (LORIA)}}

\newglossaryentry{cnrs}{
	type=entity,
	name={CNRS},
	text={CNRS},
	description={Centre National de la Recherche Scientifique},
	first={Centre National de la Recherche Scientifique (CNRS)}}

\newglossaryentry{inria}{
	type=entity,
	name={INRIA},
	text={INRIA},
	description={Institut National de Recherche en Informatique et en Automatique},
	first={Institut National de Recherche en Informatique et en Automatique (INRIA)}}

\newglossaryentry{ul}{
	type=entity,
	name={Université de Lorraine},
	text={Université de Lorraine},
	description={Université de Lorraine},
	first={Université de Lorraine (UL)}}

\newglossaryentry{gitlab}{
	type=entity,
	name={Gitlab},
	text={Gitlab},
	description={Gitlab},
	first={Gitlab}}

\newglossaryentry{synalp}{
	type=entity,
	name={SYNALP},
	text={SYNALP},
	description={SYNALP (\foreign{SYmbolic and statistical NAtural Language Processing}) est une équipe de recherche du département 4 du LORIA},
	first={SYNALP (\foreign{SYmbolic and statistical NAtural Language Processing})},
	}

\newglossaryentry{bull}{
	type=entity,
	name={BULL},
	text={BULL},
	description={}}

\newglossaryentry{atos}{
	type=entity,
	name={ATOS},
	text={ATOS},
	description={}}

\newglossaryentry{eureka}{
	type=entity,
	name={EUREKA},
	text={EUREKA},
	description={\og EUREKA est une initiative européenne, intergouvernementale, destinée à renforcer la compétitivité de l’industrie européenne.\fg{} D'après Wikipedia \footfullcite{wiki_eureka}.}, %https://fr.wikipedia.org/wiki/EUREKA
	first={EUREKA}}

\newglossaryentry{itea3}{
	type=entity,
	name={ITEA3},
	text={ITEA3},
	description={troisième instance d'ITEA (\foreign{Information Technology for European Advancement}), une initiative de recherche, développement et innovation du réseau EUREKA\glsadd{eureka}.},
	first={ITEA (\foreign{Information Technology for European Advancement})},
	see=eureka}

\newglossaryentry{papud}{
	type=entity,
	name={PAPUD},
	text={PAPUD},
	description={\foreign{Profiling and Analysis Platform Using Deep Learning}},
	first={PAPUD (\foreign{Profiling and Analysis Platform Using Deep Learning})}}

\newglossaryentry{areq}{
	type=entity,
	name={AREQ},
	text={AREQ},
	description={Assemblée des Responsables des Équipes},
	first={Assemblée des Responsables des Équipes (AREQ)}}

\newglossaryentry{project_papud}{
	type=entity,
	name={projet PAPUD},
	text={projet PAPUD},
	description={projet ITEA3-PAPUD, cas d'utilisation BULL},
	first={projet ITEA3-PAPUD, cas d'utilisation BULL\glsadd{itea3}\glsadd{papud}},
	}

\newglossaryentry{project_gmsnn}{
	type=entity,
	name={{projet GMSNN}},
	text={{projet GMSNN}},
	description={{projet basé sur une proposition innovante d'architecture de réseau de neurones, faite par M. Christophe Cerisara}},
	first={{projet Réseau de Neurones Récurrents Multi-Échelles Croissant (\foreign{Growing Multi-Scale Recurrent Neural Network} en anglais, GMSNN)}},
	}
\newglossaryentry{IDMC-}{type=\acronymtype,sort={IDMC},name={IDMC},text={IDMC},description={Indtitut des sciences du Digital, Management et Cognition}}
\glsadd{IDMC-}
\newglossaryentry{SYNALP-}{type=\acronymtype,sort={SYNALP},name={SYNALP},text={SYNALP},description={\foreign{SYmbolic and statistical NAtural Language Processing}}}
\glsadd{SYNALP-}
\newglossaryentry{LORIA-}{type=\acronymtype,sort={LORIA},name={LORIA},text={LORIA},description={Laboratoire Lorrain d'Informatique et ses Applications}}
\glsadd{LORIA-}
\newglossaryentry{PAPUD-}{type=\acronymtype,sort={PAPUD},name={PAPUD},text={PAPUD},description={\foreign{Profiling and Analysis Platform Using Deep Learning}}}
\glsadd{PAPUD-}
\newglossaryentry{ITEA-}{type=\acronymtype,sort={ITEA},name={ITEA},text={ITEA},description={\foreign{Information Technology for European Advancement}}}
\glsadd{ITEA-}
\newglossaryentry{CNRS-}{type=\acronymtype,sort={CNRS},name={CNRS},text={CNRS},description={Centre National de la Recherche Scientifique}}
\glsadd{CNRS-}
\newglossaryentry{UL-}{type=\acronymtype,sort={UL},name={UL},text={UL},description={Université de Lorraine}}
\glsadd{UL-}
\newglossaryentry{INRIA-}{type=\acronymtype,sort={INRIA},name={INRIA},text={INRIA},description={Institut National de Recherche en Informatique et en Automatique}}
\glsadd{INRIA-}
\newglossaryentry{EUREKA-}{type=\acronymtype,sort={EUREKA},name={EUREKA},text={EUREKA},description={aucune description de ce sigle n'a été trouvée}}
\glsadd{EUREKA-}

\newglossaryentry{AREQ-}{type=\acronymtype,sort={AREQ},name={AREQ},text={AREQ},description={Assemblée des Responsables des Équipes}}
\glsadd{AREQ-}

\newglossaryentry{UMR-}{type=\acronymtype,sort={UMR},name={UMR},text={UMR},description={Unité Mixte de Recherche}}
\glsadd{UMR-}

%\newglossaryentry{GMSNN-}{type=\acronymtype,sort={GMSNN},name={GMSNN},text={GMSNN},description={\foreign{Growing Multi-Scale Recurrent Neural Network}}}
%\glsadd{GMSNN-}
%\newglossaryentry{RNN-}{type=\acronymtype,sort={RNN},name={RNN},text={RNN},description={\foreign{Recurrent Neural Network}}}
%\glsadd{RNN-}
%\newglossaryentry{LSTM-}{type=\acronymtype,sort={LSTM},name={LSTM},text={LSTM},description={réseau récurrent à mémoire à court et long terme (\foreign{Long Short Term Memory}}}
%\glsadd{LSTM-}

%TAL
\newglossaryentry{XML-}{type=\acronymtype,sort={XML},name={XML},text={XML},description={\foreign{eXtensible Markup Language}}}
\glsadd{XML-}
\newglossaryentry{CSV-}{type=\acronymtype,sort={CSV},name={CSV},text={CSV},description={\foreign{comma-separated values} (valeurs séparées par des virgules)}}
\glsadd{CSV-}

\newglossaryentry{Lin.-}{type=\acronymtype,sort={Lin.},name={Lin.},text={Lin.},description={module linéaire}}
\glsadd{Lin.-}
\newglossaryentry{Emb.-}{type=\acronymtype,sort={Emb.},name={Emb.},text={Emb.},description={module d'\foreign{embedding}}}
\glsadd{Emb.-}
\newglossaryentry{Sem.-}{type=\acronymtype,sort={Sem.},name={Sem.},text={Sem.},description={semaine}}
\glsadd{Sem.-}

\newglossaryentry{log_n-}{type=\acronymtype,sort={log_n},name={$log_n$},text={$log_n$},description={logarithme en base $n$}}
\glsadd{log_n-}

\newglossaryentry{h-}{type=\acronymtype,sort={h},name={h},text={h},description={heure}}
\glsadd{h-}
\newglossaryentry{min-}{type=\acronymtype,sort={min},name={min},text={min},description={minute}}
\glsadd{min-}
\newglossaryentry{KiB-}{type=\acronymtype,sort={KiB},name={KiB},text={KiB},description={\foreign{kibibyte} (kilooctet informatioque, 1024 octets)}}
\glsadd{KiB-}
\newglossaryentry{MiB-}{type=\acronymtype,sort={MiB},name={MiB},text={MiB},description={\foreign{mebibyte} (mégaoctet informatioque, 1024 KiB)}}
\glsadd{MiB-}
\newglossaryentry{GiB-}{type=\acronymtype,sort={GiB},name={GiB},text={GiB},description={\foreign{gibibyte} (gigaoctet informatioque, 1024 MiB)}}
\glsadd{GiB-}
\newglossaryentry{BPC-}{type=\acronymtype,sort={BPC},name={BPC},text={BPC},description={\foreign{Bytes Per Character} (octets par caractère, mesure de qualité de compression)}}
\glsadd{BPC-}




%([\w\.]*) \{(.*)\}
%\\newglossaryentry{\1-}{type=\\acronymtype,sort={\1},name={\1},text={\1},description={\2}}\n\\glsadd{\1-}

%%%%%%%%%%%%%%%%%%%%%%%%%%%%%%%%%%%%%%%%%%%%%%%%%%%%%%%%%%%%%%%%%%%%%%%%%%%%%%%%
%   Machine Learning                                                           %
%%%%%%%%%%%%%%%%%%%%%%%%%%%%%%%%%%%%%%%%%%%%%%%%%%%%%%%%%%%%%%%%%%%%%%%%%%%%%%%%
\newglossaryentry{ml}{
	name={{Apprentissage automatique}},
	text={{apprentissage automatique}},
	description={{\defref{def:ml}}},
	first={{apprentissage automatique (\foreign{Machine Learning} en anglais)}}}

\newglossaryentry{dl}{
	name={{Apprentissage profond}},
	text={{apprentissage profond}},
	description={{\defref{def:dl}}},
	first={{apprentissage profond (\foreign{Deep Learning} en anglais)}}}

\newglossaryentry{preprocessing}{
	name={{Prétraitement des données}},
	text={{prétraitement}},
	description={{\defref{def:preprocessing}}},
	first={{prétraitement des données (\foreign{preprocessing} en anglais)}}}


\newglossaryentry{model}{
	name={{Modèle}},
	text={{modèle}},
	plural={{modèles}},
	description={{\defref{def:model}}}}

\newglossaryentry{module}{
	name={{Module}},
	text={{module}},
	plural={{modules}},
	description={{\defref{def:module}}}}

\newglossaryentry{architecture}{
	name={{Architecture}},
	text={{architecture}},
	description={{\defref{def:architecture}}}}

%%%%%%%%%%%%%%%%%%%%%%%%%%%%%%%%%%%%%%%%%%%%%%%%%%%%%%%%%%%%%%%%%%%%%%%%%%%%%%%%
%   Neural Networks                                                            %
%%%%%%%%%%%%%%%%%%%%%%%%%%%%%%%%%%%%%%%%%%%%%%%%%%%%%%%%%%%%%%%%%%%%%%%%%%%%%%%%
\newglossaryentry{nonlinearite}{
	name={{Non-linéarité}},
	text={{non-linéarité}},
	description={{\defref{def:nonlinearite}}}}

\newglossaryentry{nn}{
	sort={{reseau de neurones}},
	name={{Réseau de neurones}},
	text={{réseau de neurones artificiels}},
	plural={{réseaux de neurones}},
	description={{\defref{def:nn}}},
	firstplural={{réseaux de neurones artificiels, ou plus simplement réseaux de neurones (\foreign{Neural Networks} en anglais)}},
	first={{réseau de neurones artificiels ou plus simplement réseau de neurones (\foreign{Neural Network} en anglais)}}}

\newglossaryentry{module_gmsnn}{
	type=entity,
	sort={{module gmsnn}},
	name={{Module \glsentrytext{gmsnn}}},
	text={{module \glsentrytext{gmsnn}}},
	plural={{modules \glsentrytext{gmsnn}}},
	description={{Ce module est un \glsentrytext{rnn}. C'est l'objet principal du \glsentrytext{project_gmsnn}}}}

\newglossaryentry{tensor}{
	name={{Tenseur}},
	text={{tenseur}},
	first={{tenseur (\foreign{tensor} en anglais, un type de \gls{matrice} spécifique utilisé en \gls{dl})}},
	description={{\defref{def:tensor}}}}

\newglossaryentry{embedding}{
	sort={{embedding}},
	name={{\foreign{Embedding}}},
	text={\foreign{embedding}},
	plural={\foreign{embeddings}},
	description={{\defref{def:embedding}}}}

\newglossaryentry{hidden state}{
	sort={{etat cache}},
	name={{État caché}},
	text={{état caché}},
	plural={{états cachés}},
	description={}}

\newglossaryentry{rnn}{
	sort={{Reseau de neurones recurrents}},
	name={{Réseau de neurones récurrents (RNN)}},
	text={{RNN}},
	plural={{RNN}},
	first={réseau de neurones récurrents (\foreign{Recurrent Neural Network} en anglais, RNN\glsadd{rnn-})},
	description={\defref{def:rnn}}}

\newglossaryentry{rnn-}{type=\acronymtype, name={RNN}, text={RNN},
	description={\foreign{Recurrent Neural Network} (réseau de neurones récurrents)}}

\newglossaryentry{lstm}{
	sort={{Reseau recurrent a memoire a court et long terme}},
	name={{Réseau récurrent à mémoire à court et long terme (LSTM)}},
	text={{LSTM}},
	plural={{LSTM}},
	first={réseau récurrent à mémoire à court et long terme (\foreign{Long Short Term Memory} en anglais, LSTM\glsadd{lstm-})},
	description={\defref{def:lstm}}}

\newglossaryentry{lstm-}{type=\acronymtype, name={LSTM}, text={LSTM},
	description={\foreign{Long Short Term Memory} (réseau récurent à mémoire à court et long terme)}}

\newglossaryentry{gmsnn-}{
	type=\acronymtype,
	name={{GMSNN}},
	text={{GMSNN}},
	description={\foreign{Growing Multi-Scale Recurrent Neural Network} (réseau de neurones récurrents multi-échelles croissant)}}

\newglossaryentry{gmsnn}{
	name={{Réseau de neurones récurrents multi-échelles croissant (GMSNN)}},
	text={{GMSNN}},
	first={{réseau de neurones récurrents multi-échelles croissant (\foreign{Growing Multi-Scale Recurrent Neural Network} en anglais, GMSNN)\glsadd{gmsnn-}}},
	description={\defref{ch:gmsnn_model}}}

%\newglosacr{gmsnn}{GMSNN}
%{{\defref{def:gmsnn}}}
%{Réseau de Neurones Récurrents Multi-Échelles Croissant (\foreign{Growing Multi-Scale Recurrent Neural Network} en anglais, GMSNN)}

%\newglosacr{gru}{GRU}
%{{\defref{def:gru}}}
%{réseau récurent à porte (\foreign{Gated Recurent Unit} en anglais, GRU)}


%%%%%%%%%%%%%%%%%%%%%%%%%%%%%%%%%%%%%%%%%%%%%%%%%%%%%%%%%%%%%%%%%%%%%%%%%%%%%%%%
%   Language Models                                                            %
%%%%%%%%%%%%%%%%%%%%%%%%%%%%%%%%%%%%%%%%%%%%%%%%%%%%%%%%%%%%%%%%%%%%%%%%%%%%%%%%
\newglossaryentry{lm}{
	sort={{modele de la langue}},
	name={{Modèle de la langue}},
	text={{modèle de la langue}},
	plural={{modèles de la langue}},
	first={modèle de la langue (\foreign{Language Model} en anglais)},
	firstplural={modèles de la langue (\foreign{Language Models} en anglais)},
	description={\defref{def:lm}}}

%\newglosacr{char-lm}{Char-LM}
%{%
	%{Un Modèle de la Langue au niveau du Caractère (\foreign{Character-level Language Model} en anglais) est un Modèle de la Langue qui prédit non pas le prochain mot à partir des mots précédents, mais le prochain caractère à partir des caractères précédents}
	%TODO describe LM
%}{Modèle de la Langue au niveau du Caractère (\foreign{Character-level Language Model} en anglais, Char-LM)}

%\newglosacr{char-gmsnn-lm}{Char-GMSNN-LM}
%{%
	%Un Modèle du Langage au niveau du Caractère basé sur un Réseau de Neurones Multi-Échelles Croissant. C'est un modèle du langage basé sur un réseau de neurones artificiels. Ce réseau de neurones est un GMSNN utilisé comme Char-LM.}
%{Modèle du Langage au niveau du Caractère basé sur un Réseau de Neurones Multi-Échelles Croissant (\foreign{Character-level Growing Multi-Scale Language Model} en anglais, Char-LM)}

%%%%%%%%%%%%%%%%%%%%%%%%%%%%%%%%%%%%%%%%%%%%%%%%%%%%%%%%%%%%%%%%%%%%%%%%%%%%%%%%
%   Other technical expressions                                                %
%%%%%%%%%%%%%%%%%%%%%%%%%%%%%%%%%%%%%%%%%%%%%%%%%%%%%%%%%%%%%%%%%%%%%%%%%%%%%%%%
\newglossaryentry{methodes statistiques}{
	type=entity,
	name={Méthodes statistiques},
	text={méthodes statistiques},
	description={}}

\newglossaryentry{automatic differentiation}{
	name={Différentiation automatique},
	text={différentiation automatique},
	description={\defref{def:automatic differentiation}}}

\newglossaryentry{backpropagation}{
	sort={retro-propagation du gradient},
	name={Rétro-propagation du gradient},
	text={rétro-propagation du gradient},
	description={\defref{def:automatic differentiation}}}

\newglossaryentry{matrice}{
	name={Matrice},
	text={matrice},
	plural={matrices},
	description={Objet mathématique similaire à un tableau de valeurs.}
}

%\newglosacr{gpu}{GPU}
%{%
	%Un processeur graphique (\foreign{Graphical Processing Unit} en anglais, GPU) est un composant d'ordinateur spécialisé, qui montre d'excellentes performances dans les calculs impliquant des matrices (ex. : images)\glsadd{matrice}
%}{processeur graphique (\foreign{Graphical Processing Unit} en anglais, GPU)}

\newglossaryentry{gpu-}{
	type=\acronymtype,
	name={GPU},
	text={GPU},
	description={\foreign{Graphical Processing Unit} (processeur graphique)}}

\newglossaryentry{nlp-}{
	type=\acronymtype,
	name={TAL},
	text={TAL},
	description={traitement automatique des langues}}

\newglossaryentry{gpu}{
	name={Processeur graphique (GPU)},
	text={GPU\glsadd{gpu-}},
	description={\defref{def:gpu}}}

\newglossaryentry{nlp}{
	name={Traitement automatique des langues (TAL)},
	text={TAL\glsadd{nlp-}},
	first={traitement automatique des langues (TAL, \foreign{Natural Language Processing} en anglais)},
	description={\defref{def:nlp}}}

\newglossaryentry{formal grammars}{
	type=entity,
	name={Grammaires formelles},
	text={grammaires formelles},
	description={}}

\newglossaryentry{training}{
	name={Entraînement},
	text={entraînement},
	description={\defref{def:training}}
}

\newglossaryentry{learning}{
	name={Apprentissage},
	text={apprentissage},
	description={Voir \glsentrytext{training}}
}

\newglossaryentry{data}{
	name={Données},
	text={donnée},
	description={\defref{def:data}}}
\newglossaryentry{example}{
name={Exemple},
text={exemple},
description={\defref{def:example}}}
\newglossaryentry{epoch}{
	sort={epoque},
name={Époque},
text={époque},
description={\defref{def:epoch}}}
\newglossaryentry{corpus}{
name={Corpus},
text={corpus},
plural={corpus},
description={\defref{def:corpus}}}

\newglossaryentry{weight}{
	name={Poids},
	text={poids},
	description={\defref{def:weight1}}}

\newglossaryentry{computational semantics}{
	type=entity,
	name={Sémantique computationnelle},
	text={sémantique computationnelle},
	description={}}

\newglossaryentry{speech processing}{
	type=entity,
	name={Traitement de la parole},
	text={traitement de la parole},
	description={}}

\newglossaryentry{data centers}{
	sort={centres de calcul},
	name={Centres de calcul},
	text={centres de calcul},
	description={\defref{def:data centers}},
	first={centres de calcul (\foreign{data-centers} en anglais, infrastructures spécialisées regroupant de nombreux serveurs)}}

\newglossaryentry{log}{
	sort={fichier de journalisation},
	name={Fichier de journalisation},
	text={fichier de journalisation},
	plural={fichiers de journalisation},
	description={\defref{def:log}},
	first={fichier de journalisation (\foreign{log file} en anglais)},
	firstplural={fichiers de journalisation (\foreign{log files} en anglais)},}

\newglossaryentry{cloud}{
	sort={cloud},
	name={\foreign{Cloud}},
	text={\foreign{cloud}},
	description={\defref{def:cloud}}}

\newglossaryentry{parameter}{
	sort={parametre},
	name={Paramètre},
	text={paramètre},
	description={\defref{def:parameter}}}

\newglossaryentry{batch}{
	sort={batch},
	name={\foreign{Batch}},
	text={\foreign{batch}},
	plural={\foreign{batches}},
	description={\defref{def:batch}}}

\newglossaryentry{soa}{
	sort={etat de l'art},
	name={État de l'art},
	text={état de l'art},
	plural={états de l'art},
	description={\defref{def:soa}}}

\newglossaryentry{md}{
	type=entity,
	sort={Markdown},
	name={Markdown},
	text={Markdown},
	description={\og Markdown est un langage de balisage léger [dont le] but est d'offrir une syntaxe facile à lire et à écrire. Un document balisé par Markdown peut être lu en l'état sans donner l’impression d'avoir été balisé ou formaté par des instructions particulières.\fg{} D'après Wikipédia\footfullcite{wiki_md}.}}

\newglossaryentry{git_md}{
	type=entity,
	sort={Gitlab Flavoured Markdown},
	name={Gitlab Flavoured Markdown},
	text={Gitlab Flavoured Markdown},
	description={Le Gitlab Flavoured Markdown est une variante du Markdown supportant des fonctionnalités particulières telles que l'intégration d'images et de cases à cocher.}
	}

\newglossaryentry{bug}{
	name={{Bogue}},
	text={{bogue}},
	plural={{bogues}},
	description={Défaut d'un logiciel entraînant des anomalies de fonctionnement.%
		%{Un Bogue (\foreign{bug} en anglais) est \og un défaut de conception d'un programme informatique à l'origine d'un dysfonctionnement.\fg D'après Wikipédia \autocite{bug}}
	}}