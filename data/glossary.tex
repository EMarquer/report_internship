%\newacronym{svm}{SVM}{support vector machine}
%\newacronym[description={description}]{label}{short}}{long}}
%sort={empty set}, long, longplural, plural
%to use it you call \gls{svm}, \glspl{}, \Gls, \Glspl, \acrlong{label}, 
%\acrfull{label}

%\newglossaryentry{potato}{name={potato},plural={potatoes}, 
%	description={starchy tuber}}

%\newglossaryentry{dl}{name={DL},plural={potatoes}, 
%	description={starchy tuber}}
\newglossary[eng]{entity}{end}{edt}{Entités et sigles} % TODO need better name

\makenoidxglossaries

%%%%%%%%%%%%%%%%%%%%%%%%%%%%%%%%%%%%%%%%%%%%%%%%%%%%%%%%%%%%%%%%%%%%%%%%%%%%%%%%
%   Institues, Entities                                                        %
%%%%%%%%%%%%%%%%%%%%%%%%%%%%%%%%%%%%%%%%%%%%%%%%%%%%%%%%%%%%%%%%%%%%%%%%%%%%%%%%

\newglossaryentry{loria}{
	type=entity,
	name={LORIA},
	text={LORIA},
	description={Laboratoire Lorrain d'Informatique et ses Applications},
	first={Laboratoire Lorrain d'Informatique et ses Applications (LORIA)}}

\newglossaryentry{cnrs}{
	type=entity,
	name={CNRS},
	text={CNRS},
	description={Centre National de la Recherche Scientifique},
	first={Centre National de la Recherche Scientifique (CNRS)}}

\newglossaryentry{inria}{
	type=entity,
	name={INRIA},
	text={INRIA},
	description={Institut National de Recherche en Informatique et en Automatique},
	first={Institut National de Recherche en Informatique et en Automatique (INRIA)}}

\newglossaryentry{ul}{
	type=entity,
	name={Université de Lorraine},
	text={Université de Lorraine},
	description={Université de Lorraine},
	first={Université de Lorraine (UL)}}

\newglossaryentry{gitlab}{
	type=entity,
	name={Gitlab},
	text={Gitlab},
	description={Gitlab},
	first={Gitlab}}

\newglossaryentry{synalp}{
	type=entity,
	name={SYNALP},
	text={SYNALP},
	description={SYNALP (\foreign{SYmbolic and statistical NAtural Language Processing}) est une équipe de recherche du département 4 du LORIA},
	first={SYNALP (\foreign{SYmbolic and statistical NAtural Language Processing})},
	see={loria}}

\newglossaryentry{bull}{
	type=entity,
	name={BULL},
	text={BULL},
	description={}}

\newglossaryentry{atos}{
	type=entity,
	name={ATOS},
	text={ATOS},
	description={}}

\newglossaryentry{eureka}{
	type=entity,
	name={EUREKA},
	text={EUREKA},
	description={\og EUREKA est une initiative européenne, intergouvernementale, destinée à renforcer la compétitivité de l’industrie européenne.\fg{} D'après Wikipedia \autocite{wiki_eureka}.}, %https://fr.wikipedia.org/wiki/EUREKA
	first={EUREKA}}

\newglossaryentry{itea3}{
	type=entity,
	name={ITEA3},
	text={ITEA3},
	description={troisième instance d'ITEA (\foreign{Information Technology for European Advancement}), une initiative de recherche, développement et innovation du réseau EUREKA\glsadd{eureka}.},
	first={ITEA (\foreign{Information Technology for European Advancement})},
	see=eureka}

\newglossaryentry{papud}{
	type=entity,
	name={PAPUD},
	text={PAPUD},
	description={\foreign{Profiling and Analysis Platform Using Deep Learning}},
	first={PAPUD (\foreign{Profiling and Analysis Platform Using Deep Learning})}}

\newglossaryentry{areq}{
	type=entity,
	name={AREQ},
	text={AREQ},
	description={Assemblée des Responsables des Équipes},
	first={Assemblée des Responsables des Équipes (AREQ)}}

\newglossaryentry{project_papud}{
	type=entity,
	name={projet PAPUD},
	text={projet PAPUD},
	description={projet ITEA3-PAPUD, cas d'utilisation BULL},
	first={projet ITEA3-PAPUD, cas d'utilisation BULL\glsadd{itea3}\glsadd{papud}},
	see={papud,itea3}}

\newglossaryentry{project_gmsnn}{
	type=entity,
	name={{projet GMSNN}},
	text={{projet GMSNN}},
	description={{projet basé sur une proposition innovante d'architecture de réseau de neurones, faite par M. Christophe Cerisara}},
	first={{projet Réseau de Neurones Récurrents Multi-Échelles Croissant (\foreign{Growing Multi-Scale Recurrent Neural Network} en anglais, GMSNN)}},
	see={papud,itea3}}

%%%%%%%%%%%%%%%%%%%%%%%%%%%%%%%%%%%%%%%%%%%%%%%%%%%%%%%%%%%%%%%%%%%%%%%%%%%%%%%%
%   Machine Learning                                                           %
%%%%%%%%%%%%%%%%%%%%%%%%%%%%%%%%%%%%%%%%%%%%%%%%%%%%%%%%%%%%%%%%%%%%%%%%%%%%%%%%
\newglossaryentry{ml}{
	name={{apprentissage automatique}},
	text={{apprentissage automatique}},
	description={{L'apprentissage automatique (\foreign{Machine Learning} en anglais) est un ensemble de \og méthodes [statistiques] permettant à une machine (au sens large) d'évoluer par un processus systématique, et ainsi de remplir des tâches difficiles ou problématiques par des moyens algorithmiques plus classiques\fg{}. D'après Wikipédia \cite{wiki_ml}.}},
	first={{Apprentissage Automatique (\foreign{Machine Learning} en anglais)}},
	see={papud,itea3}}

\newglossaryentry{dl}{
	name={{apprentissage profond}},
	text={{apprentissage profond}},
	description={{L'apprentissage profond (\foreign{Deep Learning} en anglais), est une méthode d'\glsentrytext{ml} utilisant la technique des réseaux de neurones. %https://fr.wikipedia.org/wiki/Apprentissage_profond
		}},%TODO describe DL
	first={{apprentissage profond (\foreign{Deep Learning} en anglais)}},
	see={ml,nn}}

\newglossaryentry{model}{
	name={{modèle}},
	text={{modèle}},
	plural={{modèles}},
	description={{Un modèle en \glsentrytext{ml} est la représentation du monde construite lors de l'apprentissage afin de répondre au problème à résoudre.}},%TODO describe model
	first={{Apprentissage Profond (\foreign{Deep Learning} en anglais)}},
	see={ml}}

%%%%%%%%%%%%%%%%%%%%%%%%%%%%%%%%%%%%%%%%%%%%%%%%%%%%%%%%%%%%%%%%%%%%%%%%%%%%%%%%
%   Neural Networks                                                            %
%%%%%%%%%%%%%%%%%%%%%%%%%%%%%%%%%%%%%%%%%%%%%%%%%%%%%%%%%%%%%%%%%%%%%%%%%%%%%%%%
\newglossaryentry{nn}{
	name={{Réseau de Neurones}},
	text={{Réseau de Neurones Artificiels}},
	plural={{Réseaux de Neurones}},
	description={{Un Réseau de Neurones Artificiels ou Réseau de Neurones (\foreign{Neural Network} en anglais) est %TODO describe NN %\cite{wikinn}
	}},
	firstplural={{Réseaux de Neurones Artificiels ou Réseaux de Neurones (\foreign{Neural Networks} en anglais)}},
	first={{Réseau de Neurones Artificiels ou Réseau de Neurones (\foreign{Neural Network} en anglais)}},
	see={ml}}

\newglosacrsee{rnn}{RNN}
{%
	Un Réseau de Neurones Artificiels Récurrents, plus 
	simplement Réseau de Neurones Récurrents (\foreign{Recurrent Neural 
	Network} en anglais) est un réseau de neurones artificiels suivant une 
	architecture dite récurrente.
	\newline
	Ce genre de réseau est utilisé pour travailler avec des séquences d'entrées 
	et/ou de sorties; il y a transmission d'information entre chaque élément de 
	la séquence. \cite{wiki_rnn}
	%TODO describe RNN
}{Réseau de Neurones Récurrents (\foreign{Recurrent Neural Network} en anglais, RNN)}
{nn}

\newglosacrsee{msnn}{MSNN}
{%
	%TODO describe MSNN
	%https://en.wikipedia.org/wiki/Language_model
}{Réseau de Neurones Récurrents Multi-Échelles (\foreign{Multi-Scale Recurrent Neural Network} en anglais, MSNN)}
{rnn}

\newglosacrsee{gmsnn}{GMSNN}
{%
	(le nombre de couches augmente selon le nombre d'entrées)
	%TODO describe GMSNN
	%https://en.wikipedia.org/wiki/Language_model
}{Réseau de Neurones Récurrents Multi-Échelles Croissant (\foreign{Growing Multi-Scale Recurrent Neural Network} en anglais, GMSNN)}
{msnn}

\newglosacrsee{gru}{GRU}
{%
	%TODO describe LSTM
	%https://en.wikipedia.org/wiki/Language_model
}{Réseau récurent à porte (\foreign{Gated Recurent Unit} en anglais, GRU)}
{rnn}

\newglosacrsee{lstm}{LSTM}
{%
	%TODO describe LSTM
	%https://en.wikipedia.org/wiki/Language_model
}{Réseau récurent à mémoire à court et long terme (\foreign{Long Short Term Memory} en anglais, LSTM)}
{rnn}

%%%%%%%%%%%%%%%%%%%%%%%%%%%%%%%%%%%%%%%%%%%%%%%%%%%%%%%%%%%%%%%%%%%%%%%%%%%%%%%%
%   Language Models                                                            %
%%%%%%%%%%%%%%%%%%%%%%%%%%%%%%%%%%%%%%%%%%%%%%%%%%%%%%%%%%%%%%%%%%%%%%%%%%%%%%%%
\newglosacr{lm}{LM}
{%
	Un Modèle de la Langue (\foreign{Language Model} en anglais) est une \og distribution de probabilité sur une séquence de mots [ou de caractères]\fg{}, 
	utilisé pour estimer la probabilité d'apparition du prochain mot.
	Autrement dit, c'est une représentation servant à prédire le prochain mot à partir des mots précédents. D'après Wikipédia \cite{wiki_lm}.
	%TODO describe LM
	%https://en.wikipedia.org/wiki/Language_model
}{Modèle de la Langue (\foreign{Language Model} en anglais, LM)}

\newglosacrsee{char-lm}{Char-LM}
{%
	{Un Modèle de la Langue au niveau du Caractère (\foreign{Character-level Language Model} en anglais) est un Modèle de la Langue qui prédit non pas le prochain mot à partir des mots précédents, mais le prochain caractère à partir des caractères précédents}
	%TODO describe LM
}{Modèle de la Langue au niveau du Caractère (\foreign{Character-level Language Model} en anglais, Char-LM)}
{lm}

\newglosacrsee{char-gmsnn-lm}{Char-GMSNN-LM}
{%
	Un Modèle du Langage au niveau du Caractère basé sur un Réseau de Neurones 
	Multi-Échelles Croissant. C'est un modèle du langage basé sur un réseau de 
	neurones artificiels. Ce réseau de neurones est un GMSNN utilisé comme Char-LM.}
{Modèle du Langage au niveau du Caractère basé sur un Réseau de Neurones Multi-Échelles Croissant (\foreign{Character-level Growing Multi-Scale Language Model} en anglais, Char-LM)}
{char-lm,gmsnn}

%%%%%%%%%%%%%%%%%%%%%%%%%%%%%%%%%%%%%%%%%%%%%%%%%%%%%%%%%%%%%%%%%%%%%%%%%%%%%%%%
%   Other technical expressions                                                %
%%%%%%%%%%%%%%%%%%%%%%%%%%%%%%%%%%%%%%%%%%%%%%%%%%%%%%%%%%%%%%%%%%%%%%%%%%%%%%%%
\newglossaryentry{methodes statistiques}{
	name={méthodes statistiques},
	text={méthodes statistiques},
	description={}}

\newglossaryentry{matrice}{
name={matrice},
text={matrice},
plural={matrices},
description={}
}

\newglosacrsee{gpu}{GPU}
{%
	Un processeur graphique (\foreign{Graphical Processing Unit} en anglais) est un composant d'ordinateur spécialisé, qui montre d'excellentes performances dans les calculs impliquant des matrices (ex. : images)\glsadd{matrice}
}{processeur graphique (\foreign{Graphical Processing Unit} en anglais, GPU)}
{matrice}

\newglosacr{nlp}{NLP}
{%
	Le Traitement Automatique des Langues (\foreign{Natural Language Processing} en anglais, NLP) est une discipline qui s'intéresse au traitement des informations langagières par des moyens formels ou informatiques}
{Traitement Automatique des Langues (\foreign{Natural Language Processing} en anglais, NLP)}

\newglossaryentry{formal grammars}{
	name={grammaires formelles},
	text={grammaires formelles},
	description={}}

\newglossaryentry{computational semantics}{
	name={sémantique computationnelle},
	text={sémantique computationnelle},
	description={}}

\newglossaryentry{speech processing}{
	name={traitement de la parole},
	text={traitement de la parole},
	description={}}

\newglossaryentry{data centers}{
	sort={data centers},
	name={\foreign{data-centers}},
	text={\foreign{data-centers}},
	description={},
	first={\foreign{data-centers} (infrastructures spécialisées regroupant de nombreux serveurs)}}

\newglossaryentry{cloud}{
	sort={cloud},
	name={\foreign{cloud}},
	text={\foreign{cloud}},
	description={}}

\newglossaryentry{soa}{
	sort={etat de l'art},
	name={état de l'art},
	text={état de l'art},
	description={}}

\newglossaryentry{md}{
	sort={Markdown},
	name={Markdown},
	text={Markdown},
	description={\og Markdown est un langage de balisage léger créé en 2004 par John Gruber avec Aaron Swartz. Son but est d'offrir une syntaxe facile à lire et à écrire. Un document balisé par Markdown peut être lu en l'état sans donner l’impression d'avoir été balisé ou formaté par des instructions particulières.\fg{}\autocite{wiki_md}}}

\newglossaryentry{git_md}{
	sort={Gitlab Flavoured Markdown},
	name={Gitlab Flavoured Markdown},
	text={Gitlab Flavoured Markdown},
	description={Le Gitlab Flavoured Markdown est une variante du Markdown supportant des fonctionnalités particulières telles que l'intégration d'images et de cases à cocher}
	see={gitlab,md}}