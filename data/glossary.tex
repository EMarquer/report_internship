%\newacronym{svm}{SVM}{support vector machine}
%\newacronym[description={description}]{label}{short}}{long}}
%sort={empty set}, long, longplural, plural
%to use it you call \gls{svm}, \glspl{}, \Gls, \Glspl, \acrlong{label}, 
%\acrfull{label}

%\newglossaryentry{potato}{name={potato},plural={potatoes}, 
%	description={starchy tuber}}

%\newglossaryentry{dl}{name={DL},plural={potatoes}, 
%	description={starchy tuber}}
\newglossary[eng]{entity}{end}{edt}{Entités, projets et sigles} % TODO need better name

\makenoidxglossaries

%%%%%%%%%%%%%%%%%%%%%%%%%%%%%%%%%%%%%%%%%%%%%%%%%%%%%%%%%%%%%%%%%%%%%%%%%%%%%%%%
%   Institues, Entities                                                        %
%%%%%%%%%%%%%%%%%%%%%%%%%%%%%%%%%%%%%%%%%%%%%%%%%%%%%%%%%%%%%%%%%%%%%%%%%%%%%%%%
\newglossaryentry{loria}{
	type=entity,
	name={LORIA},
	text={LORIA},
	description={Laboratoire Lorrain d'Informatique et ses Applications},
	first={Laboratoire Lorrain d'Informatique et ses Applications (LORIA)}}

\newglossaryentry{cnrs}{
	type=entity,
	name={CNRS},
	text={CNRS},
	description={Centre National de la Recherche Scientifique},
	first={Centre National de la Recherche Scientifique (CNRS)}}

\newglossaryentry{inria}{
	type=entity,
	name={INRIA},
	text={INRIA},
	description={Institut National de Recherche en Informatique et en Automatique},
	first={Institut National de Recherche en Informatique et en Automatique (INRIA)}}

\newglossaryentry{ul}{
	type=entity,
	name={Université de Lorraine},
	text={Université de Lorraine},
	description={Université de Lorraine},
	first={Université de Lorraine (UL)}}

\newglossaryentry{gitlab}{
	type=entity,
	name={Gitlab},
	text={Gitlab},
	description={Gitlab},
	first={Gitlab}}

\newglossaryentry{synalp}{
	type=entity,
	name={SYNALP},
	text={SYNALP},
	description={SYNALP (\foreign{SYmbolic and statistical NAtural Language Processing}) est une équipe de recherche du département 4 du LORIA},
	first={SYNALP (\foreign{SYmbolic and statistical NAtural Language Processing})},
	}

\newglossaryentry{bull}{
	type=entity,
	name={BULL},
	text={BULL},
	description={}}

\newglossaryentry{atos}{
	type=entity,
	name={ATOS},
	text={ATOS},
	description={}}

\newglossaryentry{eureka}{
	type=entity,
	name={EUREKA},
	text={EUREKA},
	description={\og EUREKA est une initiative européenne, intergouvernementale, destinée à renforcer la compétitivité de l’industrie européenne.\fg{} D'après Wikipedia \footfullcite{wiki_eureka}.}, %https://fr.wikipedia.org/wiki/EUREKA
	first={EUREKA}}

\newglossaryentry{itea3}{
	type=entity,
	name={ITEA3},
	text={ITEA3},
	description={troisième instance d'ITEA (\foreign{Information Technology for European Advancement}), une initiative de recherche, développement et innovation du réseau EUREKA\glsadd{eureka}.},
	first={ITEA (\foreign{Information Technology for European Advancement})},
	see=eureka}

\newglossaryentry{papud}{
	type=entity,
	name={PAPUD},
	text={PAPUD},
	description={\foreign{Profiling and Analysis Platform Using Deep Learning}},
	first={PAPUD (\foreign{Profiling and Analysis Platform Using Deep Learning})}}

\newglossaryentry{areq}{
	type=entity,
	name={AREQ},
	text={AREQ},
	description={Assemblée des Responsables des Équipes},
	first={Assemblée des Responsables des Équipes (AREQ)}}

\newglossaryentry{project_papud}{
	type=entity,
	name={projet PAPUD},
	text={projet PAPUD},
	description={projet ITEA3-PAPUD, cas d'utilisation BULL},
	first={projet ITEA3-PAPUD, cas d'utilisation BULL\glsadd{itea3}\glsadd{papud}},
	}

\newglossaryentry{project_gmsnn}{
	type=entity,
	name={{projet GMSNN}},
	text={{projet GMSNN}},
	description={{projet basé sur une proposition innovante d'architecture de réseau de neurones, faite par M. Christophe Cerisara}},
	first={{projet Réseau de Neurones Récurrents Multi-Échelles Croissant (\foreign{Growing Multi-Scale Recurrent Neural Network} en anglais, GMSNN)}},
	}

%%%%%%%%%%%%%%%%%%%%%%%%%%%%%%%%%%%%%%%%%%%%%%%%%%%%%%%%%%%%%%%%%%%%%%%%%%%%%%%%
%   Machine Learning                                                           %
%%%%%%%%%%%%%%%%%%%%%%%%%%%%%%%%%%%%%%%%%%%%%%%%%%%%%%%%%%%%%%%%%%%%%%%%%%%%%%%%
\newglossaryentry{ml}{
	name={{Apprentissage automatique}},
	text={{apprentissage automatique}},
	description={{\defref{def:ml}}},
	first={{apprentissage automatique (\foreign{Machine Learning} en anglais)}}}

\newglossaryentry{dl}{
	name={{Apprentissage profond}},
	text={{apprentissage profond}},
	description={{\defref{def:dl}}},
	first={{apprentissage profond (\foreign{Deep Learning} en anglais)}}}

\newglossaryentry{preprocessing}{
	name={{Prétraitement des données}},
	text={{prétraitement}},
	description={{\defref{def:preprocessing}}},
	first={{prétraitement des données (\foreign{preprocessing} en anglais)}}}


\newglossaryentry{model}{
	name={{Modèle}},
	text={{modèle}},
	plural={{modèles}},
	description={{\defref{def:model}}}}

\newglossaryentry{module}{
	name={{Module}},
	text={{module}},
	plural={{modules}},
	description={{\defref{def:module}}}}

\newglossaryentry{architecture}{
	name={{Architecture}},
	text={{architecture}},
	description={{\defref{def:architecture}}}}

%%%%%%%%%%%%%%%%%%%%%%%%%%%%%%%%%%%%%%%%%%%%%%%%%%%%%%%%%%%%%%%%%%%%%%%%%%%%%%%%
%   Neural Networks                                                            %
%%%%%%%%%%%%%%%%%%%%%%%%%%%%%%%%%%%%%%%%%%%%%%%%%%%%%%%%%%%%%%%%%%%%%%%%%%%%%%%%
\newglossaryentry{nonlinearite}{
	name={{Non-linéarité}},
	text={{non-linéarité}},
	description={{\defref{def:nonlinearite}}}}

\newglossaryentry{nn}{
	sort={{reseau de neurones}},
	name={{Réseau de neurones}},
	text={{réseau de neurones artificiels}},
	plural={{réseaux de neurones}},
	description={{\defref{def:nn}}},
	firstplural={{réseaux de neurones artificiels, ou plus simplement réseaux de neurones (\foreign{Neural Networks} en anglais)}},
	first={{réseau de neurones artificiels ou réseau de neurones (\foreign{Neural Network} en anglais)}}}

\newglossaryentry{module_gmsnn}{
	sort={{module gmsnn}},
	name={{Module \glsentrytext{gmsnn}}},
	text={{module \glsentrytext{gmsnn}}},
	plural={{modules \glsentrytext{gmsnn}}},
	description={{Ce module est un \glsentrytext{rnn}. C'est l'objet principal du \glsentrytext{project_gmsnn}}}}

\newglossaryentry{tensor}{
	name={{Tenseur}},
	text={{tenseur}},
	first={{tenseur (\foreign{tensor} en anglais, un type de \gls{matrice} spécifique utilisé en \gls{dl})}},
	description={{Un tenseur en \glsentrytext{dl} est un type de \glsentrytext{matrice} particulier adapté à la technique de la \glsentrytext{automatic differentiation}}}}

\newglossaryentry{embedding}{
	sort={{embedding}},
	name={{\foreign{embedding}}},
	text={\foreign{embedding}},
	plural={\foreign{embeddings}},
	description={{Un \foreign{embedding} est un \glsentrytext{tensor} particulier produit par un module éponyme. Il est la représentation d'un caractère ou d'un mot.}}}

\newglossaryentry{hidden state}{
	sort={{etat cache}},
	name={{État caché}},
	text={{état caché}},
	plural={{états cachés}},
	description={}}

\newglossaryentry{rnn}{
	sort={{RNN}},
	name={{RNN}},
	text={{RNN}},
	plural={{RNN}},
	first={réseau de neurones récurrents (\foreign{Recurrent Neural Network} en anglais, RNN\glsadd{rnn-})},
	description={\defref{def:rnn}}}

\newglossaryentry{rnn-}{type=\acronymtype, name={RNN}, text={RNN},
	description={réseau de neurones récurrents (\foreign{Recurrent Neural Network} en anglais)}, see=[Glossaire~:]{rnn}}

\newglossaryentry{lstm}{
	sort={{LSTM}},
	name={{LSTM}},
	text={{LSTM}},
	plural={{LSTM}},
	first={réseau récurent à mémoire à court et long terme (\foreign{Long Short Term Memory} en anglais, LSTM\glsadd{lstm-})},
	description={\defref{def:lstm}}}

\newglossaryentry{lstm-}{type=\acronymtype, name={LSTM}, text={LSTM},
	description={réseau récurent à mémoire à court et long terme (\foreign{Long Short Term Memory} en anglais)}, see=[Glossaire~:]{lstm}}

\newglossaryentry{gmsnn}{
	name={{GMSNN}},
	text={{GMSNN}},
	first={{réseau de neurones récurrents multi-échelles croissant (\foreign{Growing Multi-Scale Recurrent Neural Network} en anglais, GMSNN)}},
	description={}}

%\newglosacr{gmsnn}{GMSNN}
%{{\defref{def:gmsnn}}}
%{Réseau de Neurones Récurrents Multi-Échelles Croissant (\foreign{Growing Multi-Scale Recurrent Neural Network} en anglais, GMSNN)}

%\newglosacr{gru}{GRU}
%{{\defref{def:gru}}}
%{réseau récurent à porte (\foreign{Gated Recurent Unit} en anglais, GRU)}


%%%%%%%%%%%%%%%%%%%%%%%%%%%%%%%%%%%%%%%%%%%%%%%%%%%%%%%%%%%%%%%%%%%%%%%%%%%%%%%%
%   Language Models                                                            %
%%%%%%%%%%%%%%%%%%%%%%%%%%%%%%%%%%%%%%%%%%%%%%%%%%%%%%%%%%%%%%%%%%%%%%%%%%%%%%%%
\newglossaryentry{lm}{
	sort={{modele de la langue}},
	name={{Modèle de la langue}},
	text={{modèle de la langue}},
	plural={{modèles de la langue}},
	first={modèle de la langue (\foreign{Language Model} en anglais, LM)},
	firstplural={modèles de la langue (\foreign{Language Models} en anglais, LM)},
	description={}}

%\newglosacr{char-lm}{Char-LM}
%{%
	%{Un Modèle de la Langue au niveau du Caractère (\foreign{Character-level Language Model} en anglais) est un Modèle de la Langue qui prédit non pas le prochain mot à partir des mots précédents, mais le prochain caractère à partir des caractères précédents}
	%TODO describe LM
%}{Modèle de la Langue au niveau du Caractère (\foreign{Character-level Language Model} en anglais, Char-LM)}

%\newglosacr{char-gmsnn-lm}{Char-GMSNN-LM}
%{%
	%Un Modèle du Langage au niveau du Caractère basé sur un Réseau de Neurones Multi-Échelles Croissant. C'est un modèle du langage basé sur un réseau de neurones artificiels. Ce réseau de neurones est un GMSNN utilisé comme Char-LM.}
%{Modèle du Langage au niveau du Caractère basé sur un Réseau de Neurones Multi-Échelles Croissant (\foreign{Character-level Growing Multi-Scale Language Model} en anglais, Char-LM)}

%%%%%%%%%%%%%%%%%%%%%%%%%%%%%%%%%%%%%%%%%%%%%%%%%%%%%%%%%%%%%%%%%%%%%%%%%%%%%%%%
%   Other technical expressions                                                %
%%%%%%%%%%%%%%%%%%%%%%%%%%%%%%%%%%%%%%%%%%%%%%%%%%%%%%%%%%%%%%%%%%%%%%%%%%%%%%%%
\newglossaryentry{methodes statistiques}{
	name={Méthodes statistiques},
	text={méthodes statistiques},
	description={}}

\newglossaryentry{automatic differentiation}{
	name={Différentiation automatique},
	text={différentiation automatique},
	description={}}

\newglossaryentry{matrice}{
	name={Matrice},
	text={matrice},
	plural={matrices},
	description={}
}

\newglosacr{gpu}{GPU}
{%
	Un processeur graphique (\foreign{Graphical Processing Unit} en anglais, GPU) est un composant d'ordinateur spécialisé, qui montre d'excellentes performances dans les calculs impliquant des matrices (ex. : images)\glsadd{matrice}
}{processeur graphique (\foreign{Graphical Processing Unit} en anglais, GPU)}

\newglossaryentry{nlp}{
	name={Traitement automatique des langues},
	text={TAL},
	first={traitement automatique des langues (TAL, \foreign{Natural Language Processing} en anglais)},
	description={\defref{def:nlp}}}

\newglossaryentry{formal grammars}{
	name={Grammaires formelles},
	text={grammaires formelles},
	description={}}

\newglossaryentry{training}{
	name={Entraînement},
	text={entraînement},
	description={\defref{def:weight1,def:weight2}}
}

\newglossaryentry{learning}{
	name={Apprentissage},
	text={apprentissage},
	description={Voir \glsentrytext{training}}
}

\newglossaryentry{data}{
	name={Données},
	text={donnée},
	description={\defref{def:data}}}
\newglossaryentry{example}{
name={Exemple},
text={exemple},
description={\defref{def:example}}}
\newglossaryentry{epoch}{
	sort={epoque},
name={Époque},
text={époque},
description={\defref{def:epoch}}}
\newglossaryentry{corpus}{
name={Corpus},
text={corpus},
plural={corpus},
description={\defref{def:corpus}}}

\newglossaryentry{weight}{
	name={Poids},
	text={poids},
	description={\defref{def:weight1,def:weight2}}}

\newglossaryentry{computational semantics}{
	name={Sémantique computationnelle},
	text={sémantique computationnelle},
	description={}}

\newglossaryentry{speech processing}{
	name={Traitement de la parole},
	text={traitement de la parole},
	description={}}

\newglossaryentry{data centers}{
	sort={data centers},
	name={\foreign{Data-centers}},
	text={\foreign{data-centers}},
	description={},
	first={\foreign{data-centers} (infrastructures spécialisées regroupant de nombreux serveurs)}}

\newglossaryentry{cloud}{
	sort={cloud},
	name={\foreign{Cloud}},
	text={\foreign{cloud}},
	description={}}

\newglossaryentry{parameter}{
	sort={parametre},
	name={Paramètre},
	text={paramètre},
	description={}}

\newglossaryentry{batch}{
	sort={batch},
	name={\foreign{batch}},
	text={\foreign{batch}},
	plural={\foreign{batchs}},
	description={}}

\newglossaryentry{soa}{
	sort={etat de l'art},
	name={état de l'art},
	text={état de l'art},
	description={}}

\newglossaryentry{md}{
	sort={Markdown},
	name={Markdown},
	text={Markdown},
	description={\og Markdown est un langage de balisage léger [dont le] but est d'offrir une syntaxe facile à lire et à écrire. Un document balisé par Markdown peut être lu en l'état sans donner l’impression d'avoir été balisé ou formaté par des instructions particulières.\fg{} D'après Wikipédia\footfullcite{wiki_md}.}}

\newglossaryentry{git_md}{
	sort={Gitlab Flavoured Markdown},
	name={Gitlab Flavoured Markdown},
	text={Gitlab Flavoured Markdown},
	description={Le Gitlab Flavoured Markdown est une variante du Markdown supportant des fonctionnalités particulières telles que l'intégration d'images et de cases à cocher.}
	}

\newglossaryentry{bug}{
	name={{Bogue}},
	text={{bogue}},
	plural={{bogues}},
	description={{Un Bogue (\foreign{bug} en anglais) est \og un défaut de conception d'un programme informatique à l'origine d'un dysfonctionnement.\fg D'après Wikipédia \autocite{bug}}}}