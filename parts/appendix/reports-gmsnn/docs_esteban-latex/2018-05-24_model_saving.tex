\section*{Analysis and implementation of job save and
restart}

Implementation report

by E. Marquer, 2018/05/24\\
Synalp and Université de Lorraine

\subsection{Abstract}

As the jobs are taking longer than an hour per epoch, it has become
necessary to keep an image of the model, the training parameters and the
current state of the model to be able to interrupt training when needed.

Multiple choices are available:
\begin{itemize}
\item an easy but heavy serialisation (pickle) of the whole system;
\item a ``full'' save of the model and parameters, excluding everything that can be recomputed easily;
\item a ``partial'' save of the model, removing a part of the less important
information.
\end{itemize}

\subsection{\texorpdfstring{\textbf{End solution: saving using pytorch
utilities}}{End solution: saving using pytorch utilities}}

The saving solution currently implemented is the
\lstinline!torch.save(trainer, file)! and
\lstinline!trainer = torch.load(file)! and the alternatie version
\lstinline!trainer = torch.load(file, map_location=lambda storage, loc: storage)!
allowing the loading of a CUDA model out of CUDA.

This solution posed a number of problems when it was first tested
(internal non-parameter atributes where not correctly saved and loaded),
that is why it was not considered at first.

But a more recent try resulted in a perfect result, replacing the need
of a custom serialisation system.

\subsection{Full serialisation}

This kind of serialisation is done thanks the \lstinline!pickle!
package, on the whole Trainer object.

\begin{lstlisting}[language=Python]
import pickle

def load(filename: str) -> object:
    with open(filename, 'rb') as f:
        return pickle.load(f)

def save(filename: str, trainer: object) -> None:
    with open(filename, 'wb') as f:
        pickle.dump(trainer, f)
\end{lstlisting}

\subsection{Full save}

\subsubsection{Elements to save}

\begin{itemize}
\item
  Corpus file name
\item
  \lstinline!cuda_on!
\item
  \lstinline!batch_size!
\item
  Trainer

  \begin{itemize}
  \item
    Model (see specific points)
  \item
    \lstinline!self.epoch! current epoch
  \item
    \lstinline!self.batch! and \lstinline!self.i! current position in
    training epoch
  \item
    \lstinline!self.start_time! needs a litle work: storing the elapsed
    time, and when loading, remove elapsed time from load time
  \item
    \lstinline!self.log_interval!
  \item
    \lstinline!self.save_interval!
  \item
    \lstinline!self.bptt!
  \item
    \lstinline!self.nwords!
  \item
    \lstinline!self.repackage_interval!
  \item
    \lstinline!self.repackage_strategy!
  \item
    \lstinline!self.reset_growth!
  \item
    \lstinline!self.reset_hidden!
  \item
    \lstinline!self.epochs!
  \item
    \lstinline!self.save_folder!
  \end{itemize}
\item
  Model (MSNN)

  \begin{itemize}
  \item
    \lstinline!self.training!
  \item
    input layer (embeddings) using \lstinline!torch.save(self.emb, f)!
    and \lstinline!torch.load(f)!
  \item
    MSNN (see specific points)
  \item
    output layer (linear)
  \end{itemize}
\item
  MSNN

  \begin{itemize}
  \item
    Final

    \begin{itemize}
    \item
      \lstinline!self.input_size!
    \item
      \lstinline!self.hidden_size!
    \item
      \lstinline!self.growth_factor!
    \item
      \lstinline!self.batch_size!
    \item
      \lstinline!self.cuda_on!
    \item
      \lstinline!self.layer_id!
    \item
      \lstinline!self.max_detach!
    \item
      \lstinline!self.repackage_strategy!
    \item
      \lstinline!self.max_layers!
    \end{itemize}
  \item
    Next MSNN
  \item
    \lstinline!self.detach_count!
  \item
    \lstinline!self.rnn!
  \item
    \lstinline!self.hidden!
  \item
    \lstinline!self.seq_count!
  \item
    \lstinline!self.detach_count!
  \item
    \lstinline!self.transmitted_output!
  \item
    \lstinline!self.transmitted_hidden!
  \end{itemize}
\end{itemize}

\subsubsection{Elements to forget and
recreate}

\begin{itemize}
\item
  Trainer

  \begin{itemize}
  \item
    Corpus

    \begin{itemize}
    \item
      Corpus file name is needed
    \end{itemize}
  \item
    Optimizer: reasign learning rate, weight\_decay and model's
    parameters (create after model is loaded) using
    \lstinline!torch.optim.SGD(model.parameters(), lr=args.lr, weight_decay=args.weight_decay)!
  \item
    Criterion: \lstinline!criterion = torch.nn.CrossEntropyLoss()!
  \item
    \lstinline!self.layers! using
    \lstinline!self.model.msnn.get_layers()! (create after model is
    loaded)
  \item
    \lstinline!self.train_data! using
    \lstinline!batchify(corpus.train, batch_size, cuda_on)!
  \item
    \lstinline!self.val_data! using
    \lstinline!batchify(corpus.valid, batch_size, cuda_on)!
  \item
    \lstinline!self.plotdata! using
    \lstinline!{"Epoch": None, "Layer": None, "Frac": None}! and
    \lstinline!self.init_plotter()! (perhaps with a new file, otherwise
    in append mode without cleaning the file)
  \item
    \lstinline!self.msnn_backup!, sould be empty if the new backup
    system (using the \lstinline!with! statement) is implemented
  \end{itemize}
\item
  Model

  \begin{itemize}
  \item
    None
  \end{itemize}
\item
  MSNN

  \begin{itemize}
  \item
    \lstinline!self.tensors!
  \end{itemize}
\end{itemize}

\subsection{Partial save}

The elements to keep are the same as with the Full save, except: -
hidden states and transmitted output are to be detached

\subsection{Other things that need to be added to interrupt and resume
job}

\subsubsection{Methods to interrupt and resume epoch loop and training
loop}

\paragraph{Interruption strategy}

Interruption can be done by:
\begin{itemize}
\item saving the state at specific timestep;
\item saving automatically when shutting down.
\end{itemize}

The second option isn't realy realistic, as an interuption wouldn't
allow a big enough margin to save the model. Even thought, it could be
added to allow manual interruption at certain timesteps (smaller than
the ones implemented in the first option).

As the first option is the only viable one, multiple timesteps are
possible, from the shortest to the longest:
\begin{itemize}
\item after each operation (example: after forward pass, and after backward pass, and after optimization \ldots{});
\item after each sequence;
\item after \emph{n} sequences;
\item  after each epoch;
\item  after \emph{n} epochs;
\item  after the whole training.
\end{itemize}

Option {[}1.{]} is not viable, as it would consume a lot of time
relative to each computation for the sole writing process. Option
{[}3.{]}, of which option {[}2.{]} is a specific case, allows both a
fine granulation with only a minimal loss if anything were to occur, and
a reduced burden computation-wise. From option {[}4.{]} onward, the save
time is negligible, and even if the granulation is mediocre, it offers
fine milestones for specific usage (usage of an already trained mode for
example).

Currently, at creation time, with no history, the model save file is
about 770MB.

What will be implemented is thoe third option, saving the model after
\emph{n} sequences.

\paragraph{Resume strategy}\label{resume-strategy}

It will be necessary to adapt the training loodps a little to allow
resuming at any sequence in any epoch.

\subsection{Other methods implemented}\label{other-methods-implemented}

\subsubsection{\texorpdfstring{Method using \texttt{with} keyword for
backup system, during
evaluation}{Method using with keyword for backup system, during evaluation}}\label{method-using-with-keyword-for-backup-system-during-evaluation}

\begin{lstlisting}[language=Python]
class BackupContextManager:
    def __init__(self, model: DetRNN):
        self.model = model
        
    def __enter__(self):
        # Backing up training linked data
        self.msnn_backup = self.model.msnn.backup()

    def __exit__(self, type, value, traceback):
        # Restoring training linked data
        self.model.msnn.restore(self.msnn_backup)


with BackupContextManager(model):
    """Do computations"""
\end{lstlisting}

