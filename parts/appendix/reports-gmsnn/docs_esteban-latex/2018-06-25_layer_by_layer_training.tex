% !TeX spellcheck = en_US
\section*{Layer by layer training}

Analysis report

by E. Marquer, 2018/06/25, Synalp and Université de Lorraine

\subsection{Abstract}

Multiple advanced training algorithms use layer ``freezing'', meaning
that the layer will not be trained.

As it would be interesting to use those algorithms to get the most out
of the current architecture, a layer ``freezing'' will be implemented.
To do so, a dummy algorithm will be implemented. This algorithm is a
naive layer by layer training.

\subsection{Layer by layer training}

This training is used to see if convergence can be sped up, if
performance can be improved, and if the layered architecture is of any
use.

\subsubsection{Principle}

The general principle is to train each layer individually, and to
fine-tune them together frequently.

An iterative presentation would be:

\begin{enumerate}
\def\labelenumi{\arabic{enumi}.}
\item
  Create a layer
\item
  Train the layer alone
\item
  Train all the layers together
\item
  Restart from {[}1.{]}
\end{enumerate}

Another way to explain this algorithm is that it is a variation of the
IM algorithm, were storing the output of the training of a layer is
replaced by recomputing those results. It removes the drawback of the
increase in memory usage, while speeding up training(time-wise at least,
convergence-wise at best).

\newpage
\paragraph{Example for a 4 layered
MSNN}

\begin{enumerate}
\def\labelenumi{\arabic{enumi}.}
\item
  We train for n epochs the first layer. It is expected to learn a
  maximum of things of a low level of abstraction and time dependency.
\item
  Then, we freeze this layer, and start training a second one over n
  epochs. It is expected to learn a higher level of abstraction and time
  dependency form the representations in the first layer's hidden state.
\item
  We train those two layers together to fine-tune them.
\item
  We then add a third layer, freeze the first two layers, and train the
  third layer of information extracted from the second layer.
\item
  We train the three layers together.
\item
  We add a new layer and train it alone.
\item
  We train the four layers together.
\item
  We train all the layers until the ends of epochs.
\end{enumerate}

\subsubsection{Speeding up convergence and improving
performance}

Training each layer individually requires less computation during
backpropagation. Moreover, by pushing each layer to learn the maximum of
information it can learn, we can expect each layer to specialize at
their scale.

A more detailed way to understand the intuition behind that is that a
layer closer to the data has to learn very basic features. Step by step,
every layer is constrained to its respective scale (by its memory span
due to the architecture). Each layer has a minimal set of knowledge to
learn before benefiting to the whole network. Even if a part of this
learning is shared over the layers, at least over the first epochs there
would be nothing to gain but noise by training all the layers together.

Globally, by skipping the ``noisy'' part of the training, a reduction of
training time (the real computation time, not the number of epochs
needed) is to be expected. A bonus effect would be a small acceleration
of convergence, as training would be less noisy.

\subsubsection{Use of the layered
architecture}

By training the model layer by layer, we can expect to see if training a
single layer with equivalent parameters would have the same effect (if
the individual training time is big enough).
