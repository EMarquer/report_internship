% !TeX spellcheck = en_US
\section*{Analysis and implementation of improved Truncated-BPTT training
algorithm}

Implementation report

by E. Marquer, 2018/05/29\\
Synalp and Université de Lorraine

\subsection{Abstract}

Last update (implementation of explicit history to solve memory leaks
problem) solved memory problems, leaving the time consumption problem
(hundreds of hours for a single epoch).

As the most time-consuming process is the backpropagation, the most
evident way to reduce time consumption is to improve the training
strategy.

One of the possible optimization is the Truncated Backpropagation
Through Time (Truncated-BPTT or TBPTT).

\subsection{Notations}

The following notations are from {an introduction article on BPTT} \autocite{bptt-intro}:

\begin{quote}
\begin{itemize}
\item
  \textbf{TBPTT(n,n)}: Updates are performed at the end of the sequence
  across all timesteps in the sequence (e.g. classical BPTT).
\item
  \textbf{TBPTT(1,n)}: timesteps are processed one at a time followed by
  an update that covers all timesteps seen so far (e.g.~classical TBPTT
  by Williams and Peng).
\item
  \textbf{TBPTT(k1,1)}: The network likely does not have enough temporal
  context to learn, relying heavily on internal state and inputs.
\item
  \textbf{TBPTT(k1,k2)}, where k1\textless{}k2\textless{}n: Multiple
  updates are performed per sequence which can accelerate training.
\item
  \textbf{TBPTT(k1,k2)}, where k1=k2: A common configuration where a
  fixed number of timesteps are used for both forward and backward-pass
  timesteps (e.g.~10s to 100s).
\end{itemize}
\end{quote}

The base implementation of the model, using the
\lstinline!(sequence_length, batch_size, values)! model for the inputs
(and outputs), is already an implementation of the \textbf{TBPTT(n,n)}
algorithm.

What would improve a lot time efficiency is to implement a
\textbf{TBPTT(k1,k2)} algorithm.

\subsection{Algorithm}\label{algorithm}

The algorithm can decompose into 4 steps: 1. Present a sequence of
\emph{k1} timesteps of input and output pairs to the network. 2. Compute
loss across the \emph{k2} last timesteps. 3. Backpropagate loss 4.
Update weights

\subsection{Pseudo-python code}\label{pseudo-python-code}

\subsubsection{Old algorithm}\label{old-algorithm}

\begin{lstlisting}[language=Python]
for sequence in sequences:
    # 1. Present a sequence of *n* timesteps of input to the network.
    output, hidden = model.forward(sequence.input, hidden)
    
    # 2. Compute loss across the *n* timesteps.
    loss = criterion(output, sequence.targets)
    
    # 3. Backpropagate loss
    loss.backward()
    
    # 4. Update weights
    optimizer.step()
\end{lstlisting}

\subsubsection{New algorithm}\label{new-algorithm}

\begin{lstlisting}[language=Python]
for sequence in sequences:
    # 1. Present a sequence of *k1* timesteps of input to the network.
    output, hidden = model.forward(sequence.input, hidden)
    
    # 2. Compute loss across the *k2* last timesteps.
    loss = criterion(output[:-k2], sequence.targets[:-k2])
    
    # 3. Backpropagate loss
    loss.backward()
    
    # 4. Update weights
    optimizer.step()
\end{lstlisting}

