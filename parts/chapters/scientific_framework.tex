\chapter[Cadre théorique]{Cadre théorique (terminologie et concepts fondamentaux) \label{ch:sci_framework}}

\section{Introduction} % quoi que comment le contexte
% "le cadre théorique et définition des concepts de base de notre travail."
Ce chapitre est dédié à la présentation et l'explication des théories, termes et concepts nécessaires à la compréhension de ce rapport.

Dans un premier temps, le domaine scientifique dans lequel s'est déroulé le stage sera expliqué.
Dans un second temps, les concepts et termes fondamentaux utilisés dans ce mémoire seront définis.
Enfin, pour situer le stage dans son contexte scientifique actuel, un aperçu de l'état de l'art dans la littérature sera donné.

\section{Éléments théoriques généraux} % définitions du domaine global
Ce stage s'inscrit dans deux principaux domaines :
\begin{itemize}
	\item l'\gls{dl};
	\item le \gls{nlp}.
\end{itemize}

\subsection{\Glsentrytext{dl}}
L'\gls{dl} représente un ensemble de techniques de \gls{ml}, basés sur ce que l'on appelle des \glspl{nn}.

\glsdesc{ml}

Faisant partie des méthodes du \gls{ml}, l'\gls{dl} regroupe à la fois les méthodes de création, d'entraînement, d'optimisation et d'utilisation des modèles basés sur des \glspl{nn}.

\subsection{\Glsentrytext{nlp}}

\section{Concepts et termes fondamentaux} % définition spécifique
\begin{description}
	\item[\Gls{nn}] \glsdesc{nn}
	\item[\Gls{model}] \glsdesc{model} Dans le cadre de l'\gls{dl}, le modèle correspond au réseau de neurone.
\end{description}


\subsection{\Glsentrytext{lm}}
\section{État de l'art\label{sec:soa}}
\subsection{\Glsentrytext{project_gmsnn}}
\subsection{\Glsentrytext{project_papud}}

--> next parts
