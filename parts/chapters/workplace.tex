%\chapter{Présentation de l'entreprise}
%il est indispensable de positionner le stage dans son contexte, c’est-à-dire de présenter l’entreprise et le service dans lequel il a eu lieu. Il faut éviter les « copier-coller » des plaquettes promotionnelles de l’entreprise, les organigrammes souvent illisibles, en revanche il peut être utile de concevoir des schémas qui montrent exactement ce qui est utile pour situer le stage.

\chapter{Présentation du laboratoire et de l'équipe}
\section[SYNALP]{L'équipe SYNALP}
L'équipe \gls{synalp} est une équipe de recherche affiliée à la fois au \gls{cnrs} et à l'\gls{ul}.

\gls{synalp} est dans le 

Des informations détaillées sont disponible sur le site de l'équippe (en Anglais) \cite{about_synalp}
%es/about-synalp/}{WEB}{name}{text}
%{http://synalp.loria.fr/pages/about-synalp/}

\section[LORIA]{Présentation du LORIA}
Le \glsfirst{loria} est une Unité Mixte de Recherche (UMR 7503), commune à plusieurs établissements : le \gls{cnrs}, l’\gls{ul} et l'\gls{inria}.

Le Loria a pour mission la recherche fondamentale et appliquée en sciences informatiques et ce, depuis sa création, en 1997.

\section[ITEA3/PAPUD]{Présentation du projet ITE3-PAPUD, cas d'utilisation BULL}
\subsection*{Initiative ITEA}

\subsection*{Projet PAPUD}

\subsection*{Cas d'utilisation BULL}

Par la suite, nous désignerons ce projet par "Projet PAPUD"
