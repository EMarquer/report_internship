\chapter{About ITEA}

\emph{ITEA is a transnational and industry-driven Research, Development and Innovation (R\& D\& I) programme in the domain of software innovation. ITEA is a Cluster programme of EUREKA, an intergovernmental network for R\& D\& I cooperation, involving over 40 countries globally. ITEA is the home of software innovation, enabling an international and knowledgeable community to collaborate in funded projects that turn innovative ideas into new businesses, jobs, economic growth and benefits for society.}

\section{Our vision}

In a rapidly changing society where societal challenges are omnipresent, digitisation is no longer an option but should be regarded as the opportunity to create innovative solutions. Digital technology will be applied in all aspects of society, touching every element of people’s lives. Software innovation is a core component for mastering this Digital Transition and this is the main focus of ITEA. Digital Transition is not a one-step process. It has many dimensions whereby development must be continuous. The more that digitisation is enabled, the more penetrative the transition becomes and the need for more solutions increases. To achieve this continuous process in a smooth way, a fertile and collaborative environment is needed in which there are innovative ideas and knowledgeable people that are eager to share, to inspire and be inspired by each other.

Digital Technology has a major role to play in mastering the changes. And it is within this domain of Digital Technology that ITEA is addressing innovation in Software, IT Services, Internal IT and Embedded Software, collectively denoted as ‘Software innovation’. For Europe, an industry strong in Software Innovation is a prerequisite for maintaining global competitiveness and in securing high-value jobs in Digital Technology and in other, more traditional industries that are dependent on Digital Technology.

\section{Our mission}

It is ITEA’s mission to enable businesses, with the involvement of their customers, to create innovative solutions that master the Digital Transition and tackle the major challenges in a way that really helps bring society forward. ITEA encourages its global Community to create impact and value through R\&D\&I projects in the area of Software Innovation with the knowledge of industry and the capability of national financing.

\section{ITEA is...}

\begin{itemize}
    \item[Global and trusted cooperation in an industrial community:]
    ITEA stimulates innovation projects in a global community of large industry, small and medium-sized enterprises (SMEs), start-ups, academia and customer organisations. ITEA’s bottom-up project creation ensures that the project ideas are industry-driven and based on actual customer needs. ITEA provides a trusted framework for cooperation in which standard project collaboration agreements are available, including the complex domains of confidentiality and intellectual property. ITEA is managed by and for industry in close cooperation with the national public authorities.
    
    \item[Project financing through national public and private funding:]
    The ITEA programme is publicly funded on a national level. Each ITEA project partner can apply for funding from their own national Public Authority. An early dialogue between project teams and public authorities supports alignment with national priorities and the best possible opportunities for funding that lead to high success rates.
    
    \item[Commercialisation of research results:]
    ITEA enables organisations to create actual commercial results from research projects. Impact is one of the core values in ITEA; impact on business, economy and society. Impact is central during the project lifecycle: in proposal evaluation, monitoring, closure and in communication of the results.
    
    \item[Focus on high-quality process and support:]
    ITEA follows a flexible and supportive approach towards its project consortia in order for them to maximise the results of their efforts and the project’s impact. Each year, ITEA issues one Call for projects starting with a two-day brokerage event. Each Call follows a two-step procedure, in which industrial experts evaluate the quality of the project proposal in terms of innovation, impact and consortium. During the project lifetime, ITEA provides full-cycle project monitoring in a peer-to-peer mode with digital reports and physical reviews to improve quality and value creation of projects. ITEA has an ISO 9001 certified Quality Management System (by DEKRA).

\end{itemize}