\chapter{Exemples Listings}

Il est aisé d'insérer du code dans un rapport. Il suffit de définir le langage, la légende à afficher et enfin un Label pour pouvoir y faire référence. Le résultat est donnée dans le listing \ref{lst:premierExemple}. Il est également possible de changer les couleurs, pour cela il faut éditer le lstset dans la classe tnreport.cls.

\begin{lstlisting}[language=c++, caption={Premier Exemple}, label={lst:premierExemple}]
void CEquation::IniParser()
{
if (!pP){ //if not already initialized...
pP = new mu::Parser;

pP->DefineOprt("%", CEquation::Mod, 6); //deprecated
pP->DefineFun("mod", &CEquation::Mod, false);
pP->DefineOprt("&", AND, 1); //DEPRECATED
pP->DefineOprt("and", AND, 1);
pP->DefineOprt("|", OR, 1); //DEPRECATED
pP->DefineOprt("or", OR, 1);
pP->DefineOprt("xor", XOR, 1);
pP->DefineInfixOprt("!", NOT);
pP->DefineFun("floor", &CEquation::Floor, false);
pP->DefineFun("ceil", &CEquation::Ceil, false);
pP->DefineFun("abs", &CEquation::Abs, false);
pP->DefineFun("rand", &CEquation::Rand, false);
pP->DefineFun("tex", &CEquation::Tex, false);

pP->DefineVar("x", &XVar);
pP->DefineVar("y", &YVar);
pP->DefineVar("z", &ZVar);
}
}
\end{lstlisting}
\clearpage
Il est également possible d'afficher du code directement depuis un fichier source, le résultat de cette opération est visible dans le listing \ref{lst:fromSrc}
%\lstinputlisting[language=c++,caption={Affichage depuis le fichier source},label={lst:fromSrc}]{figures/sourceCode.cpp}

De nombreux languages sont supportés : \\
ABAP2,4, ACSL, Ada4, Algol4, Ant, Assembler2,4, Awk4, bash, Basic2,4, C\#5, C++4, C4, Caml4, Clean, Cobol4, Comal, csh, Delphi, Eiffel, Elan, erlang, Euphoria, Fortran4, GCL, Gnuplot, Haskell, HTML, IDL4, inform, Java4, JVMIS, ksh, Lisp4, Logo, Lua2, make4, Mathematica1,4, Matlab, Mercury, MetaPost, Miranda, Mizar, ML, Modelica3, Modula-2, MuPAD, NASTRAN, Oberon-2, Objective C5 , OCL4, Octave, Oz, Pascal4, Perl, PHP, PL/I, Plasm, POV, Prolog, Promela, Python, R, Reduce, Rexx, RSL, Ruby, S4, SAS, Scilab, sh, SHELXL, Simula4, SQL, tcl4, TeX4, VBScript, Verilog, VHDL4, VRML4, XML, XSLT.
\clearpage
Il est néanmoins possible de définir le sien, il faudra alors ajouter dans la classe tnreport.cls du code resemblant au listing \ref{lst:defLang}. On y définit les différents mots-clés, ainsi que les délimiteurs des chaines de caractère et des commentaires.
\begin{lstlisting}[language=Tex, caption={Syntaxe définition d'un langage}, label={lst:defLang}]
\lstdefinelanguage{amf}
{keywords=
{
xml,
amf,
volume,
material,
coordinates,
vertices,
vertex,
triangle,
x,
y,
z,
v1,
v2,
v3,
mesh,
object,
constellation,
metadata,
color,
texmap,
texture,
utex1,
utex2,
utex3,
instance,
deltax,
deltay,
deltaz,
r,
g,
b,
rx,
ry,
rz,
composite
},
sensitive=false,
morestring=[b]",
comment=[s]{<!--}{-->}
}
\end{lstlisting}