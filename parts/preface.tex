\section*{Avant-propos} % (optionnel)
\addcontentsline{toc}{chapter}{Avant-propos} % (optionnel)
La lecture du présent rapport ne nécessite pas de connaissance préalable en traitement automatique de la langue, en apprentissage automatique ou en apprentissage profond.

Il est cependant préférable d'avoir quelques connaissances en informatique pour comprendre les enjeux du stage.

Par ailleurs, de nombreux termes techniques sont utilisés.
%La plupart est abrégée ou sous forme de sigle.
%
%Bien qu'habituellement rencontrés sous leur forme anglaise dans le domaine, dans ce rapport tous les termes qui possèdent une traduction française ont été traduits.
%Les seules exceptions sont certains sigles, qui sont utilisés tels quels dans les textes français.
%
% TODO duplicat
Bien qu'habituellement rencontrés sous leur forme anglaise dans le domaine, ces termes ont été traduits en français, à l'exceptions sont certains sigles utilisés tels quels dans la littérature française.

Tous les termes techniques seront donc introduits en français, accompagnées d'une explication, de la traduction anglaise et du sigle anglais si nécessaire.
%
%{D'une part, ces abréviation sont habituellement en anglais dans la littérature du domaine.
%Aussi, pour maintenir la lisibilité pour les lecteurs initiés au domaine, les abréviations utilisées seront sous leur forme anglaise.
%Pour maintenir une cohérence dans les termes utilisés, les expression dont les versions anglaises et françaises 
%Enfin, les termes techniques seront toujours introduits en français, accompagnées d'une explication, de la traduction anglaise et du sigle anglais. Par exemple : Exemple de Terme Technique (\foreign{Technical Expression Example} en anglais, TEE).} % TODO ne sera peut-etre plus nécessaire, si les termes en anglais se révèle peu nombreux, ils seront traduits
%
De plus, une partie du rapport est dédiée à l'explication des termes et concepts utilisés, et un glossaire est présent en fin d'ouvrage.
%
%pas besoin de connaisances en ML ou en DL / NN pour comprendre ce rapport, bien que préférable
%abrev en englais à cause de l'usage (+ autres explications), mais tjs introduites en francais, avec la traduction anglais et le sigle
%
%Enfin, il est à noter que les commentaires de mes programmes ainsi que les interactions avec l’utilisateur ont été écrits en anglais par soucis d’une réutilisation future afin que l’ensemble de mon travail soit compréhensible par tous.
%
Enfin, quelques notes de bas de page fournissent des précisions ponctuelles sur certains concepts utilisées. Elles sont notées en exposant. 

Les références aux sources sont marquées entres crochets, et numérotées par ordre d'utilisation.

Ce rapport a été réalisé avec \LaTeX{} et les diagrammes présentés ont été produits par nos soins avec l'outil Ti\textit{k}Z, ce qui explique l'absence de source pour les figures.