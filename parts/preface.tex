\section*{Avant-propos} % (optionnel)
\addcontentsline{toc}{chapter}{Avant-propos} % (optionnel)
La lecture du présent rapport ne nécessite pas de connaissance préalable en traitement automatique de la langue, en apprentissage automatique ou en apprentissage profond.

Il est cependant préférable d'avoir quelques connaissances en informatique pour comprendre les enjeux du stage.

Par ailleurs, de nombreux termes techniques sont utilisés.
Bien qu'habituellement rencontrés sous leur forme anglaise dans le domaine, ces termes ont été traduits en français, à l'exception de certains sigles utilisés tels quels dans la littérature française.

Tous les termes techniques seront donc introduits en français, accompagnés d'une explication, de la traduction anglaise et du sigle anglais si nécessaire.
De plus, une partie du rapport est dédiée à l'explication des termes et concepts utilisés, et un glossaire est présent en fin d'ouvrage (\autopageref{gls}).
Enfin, quelques notes de bas de page fournissent des précisions ponctuelles sur certains concepts utilisés. Elles sont notées en exposant. 

Les références aux sources (consultables en fin d'ouvrage, \autopageref{bib}) sont indiquées entre crochets, et numérotées par ordre d'utilisation.

Ce rapport a été réalisé avec \LaTeX{} et les diagrammes présentés ont été produits par nos soins avec l'outil Ti\textit{k}Z, ce qui explique l'absence de source pour les figures.

Conformément aux consignes données, les rapports intermédiaires ont été inclus dans les annexes du rapport.

Pour la rédaction de ce rapport, nous avons choisi d'utiliser la première personne du pluriel, afin de maintenir autant que possible neutralité et recul par rapport aux éléments rapportés. Cependant, les conclusions personnelles du rapport ne nécessitant pas cette neutralité, elles ont été rédigées à la première personne du singulier.
