\section*{Avant-propos} % (optionnel)
\addcontentsline{toc}{chapter}{Avant-propos} % (optionnel)
La lecture du présent mémoire ne nécessite aucune connaissance préalable en Apprentissage Automatique ou en Apprentissage Profond.

Cependant, de nombreux termes techniques sont utilisés.
La plupart est abrégée ou sous forme de sigle.

{D'une part, ces abréviation sont habituellement en anglais dans la littérature du domaine.
Aussi, pour maintenir la lisibilité pour les lecteurs initiés au domaine, les abréviations utilisées seront sous leur forme anglaise.
Pour maintenir une cohérence dans les termes utilisés, les expression dont les versions anglaises et françaises 
Enfin, les termes techniques seront toujours introduits en français, accompagnées d'une explication, de la traduction anglaise et du sigle anglais. Par exemple : Exemple de Terme Technique (\foreign{Technical Expression Example} en anglais, TEE).} % TODO ne sera peut-etre plus nécessaire, si les termes en anglais se révèle peu nombreux, ils seront traduits

D'autre part, une partie du rapport est dédiée à l'explication des termes et concepts utilisés, et un glossaire est présent en fin d'ouvrage.

%pas besoin de connaisances en ML ou en DL / NN pour comprendre ce rapport, bien que préférable
%abrev en englais à cause de l'usage (+ autres explications), mais tjs introduites en francais, avec la traduction anglais et le sigle

%Enfin, il est à noter que les commentaires de mes programmes ainsi que les interactions avec l’utilisateur ont été écrits en anglais par soucis d’une réutilisation future afin que l’ensemble de mon travail soit compréhensible par tous.