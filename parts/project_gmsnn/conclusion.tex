\chapter{Conclusions sur le \glsentrytext{project_gmsnn}}
% ccl plus pb mémoire & tps, opti z'oignons (XD) même, 5 min c'est balèze
\section{Retour sur le travail effectué}
Ce projet nous à permit d'implémenter une architecture innovante de \gls{nn}, à partir du squelette d'un modèle \gls{soa}.
Nous avons pus élaborer un prototype suivant les concepts clés de l'architecture proposée, avant de l'améliorer et de l'optimiser.

Pour cela nous avons étudié un domaine technique dans lequel nous avions peu de connaissances~; nous avons manipulé une librairie qui nous était inconnue~;
nous avons géré des tests durant de plusieurs heures à plusieurs jours sur des machines distantes~; nous avons, enfin, affronté un des obstacles les plus importants dans le développement de \gls{nn}, le problème de l'optimisation.

Bien que le l'architecture \gls{gmsnn} n'ai pas atteint les performances espérées, le modèle produit est robuste, rapide, et peu volumineux.
De plus, l'algorithme présenté \autoref{subsec:optilbl} à démontré une faiblesse majeure de l'architecture \gls{gmsnn}.
Enfin, les problèmes rencontrés dans ce projet ont permit de tirer des conclusions très utiles pour de prochains projets~:
\begin{itemize}
	\item les \gls{rnn} sont très lents à entraîner~;
	\item la maîtrise du pré-traitement est fondamentale pour obtenir des bons résultats~;
	\item pour utiliser une architecture multi-échelle comme celle proposée, il vaut mieux entraîner un modèle simple en premier lieu.
\end{itemize}\hspace{1em}

C'est d'un commun accord avec notre maître de stage que nous avons décidé de basculer sur le \gls{project_papud}.

En conclusion, le projet à abouti sur le rejet de l'architecture proposée.
Néanmoins, ce résultat à permit de cerner les principaux écueils de la réalisation d'un \gls{lm} multi-échelle, et à ainsi permit un meilleur déroulement du projet suivant.


%%%%
\section{Apport personnel du projet}
La réalisation de ce projet nous à permit d'approfondir largement nos connaissances en \gls{dl}, et de nous habituer aux problématique de la création et de l'utilisation de \glspl{nn}.

\section{Discussion et perspectives}
\subsection{Pertinence des choix et possibilités d'exploration}
Il est important de relever que dans beaucoup des situations rencontrées, nous avons dû faire des choix. De même dans l'ordre de priorité des optimisations à effectuer. La pertinence de nos choix est d'autant plus douteuse que notre niveau d'expertise du domaine est faible.

Si à posteriori nous sommes capables d'envisages d'autres pistes pour poursuivre ce projet, en aucun cas nous ne regrettons les choix effectués, en particulier la décision d'abandonner le projet.

Par exemple, nous aurions pu optimiser le taux d'apprentissage, comme fait dans le \gls{project_papud} (voir \autoref{lr_opti_papud}).

Parmi les très nombreuse possibilités envisagées, on trouve aussi~:
\begin{itemize}
	\item l'utilisation d'un \gls{rnn} pour interpréter les informations des différentes échelles~;
	\item la poursuite de l'utilisation de l'algorithme couche par couche, en poussant chaque échelle au maximum de ses capacités avant d'intégrer de nouvelles couches~;
	\item le changement de l'architecture de pyramidale à parallèle, c'est à dire que chaque échelle serait indépendante, similairement à l'article \autocite{Xiao2018Jan}.
\end{itemize}

\subsection{Travaux similaires}
Dans un article datant du 27 Juillet 2018 \autocite{hierachical_rnn}, soit quelques jours avant la fin du stage, une architecture extrêmement similaire à celle du \gls{gmsnn} est développée.

Contrairement a notre stage, le modèle de l'article montre des performances supérieures à celles des autres architectures auxquelles il est comparé.

En écho à la partie précédente, cela peut être dû à des choix de développement différent, comme à un travail plus poussé et plus expert sur le sujet.
