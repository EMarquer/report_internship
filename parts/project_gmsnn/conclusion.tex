\chapter{Conclusions sur le \glsentrytext{project_gmsnn}}
% ccl plus pb mémoire & tps, opti z'oignons (XD) même, 5 min c'est balèze
\section{Retour sur le travail effectué}
Ce projet nous à permit d'implémenter une architecture innovante de \gls{nn}, à partir du squelette d'un modèle \gls{soa}.
Nous avons pus élaborer un prototype suivant les concepts clés de l'architecture proposée, avant de l'améliorer et de l'optimiser.

Pour cela nous avons étudié un domaine technique dans lequel nous avions peu de connaissances~; nous avons manipulé une librairie qui nous était inconnue~;
nous avons géré des tests durant de plusieurs heures à plusieurs jours sur des machines distantes~; nous avons, enfin, affronté un des obstacles les plus importants dans le développement de \gls{nn}, le problème de l'optimisation.

Bien que le l'architecture \gls{gmsnn} n'ai pas atteint les performances espérées, le modèle produit est robuste, rapide, et peu volumineux.
De plus, l'algorithme présenté \autoref{subsec:optilbl} à démontré une faiblesse majeure de l'architecture \gls{gmsnn}.
Enfin, les problèmes rencontrés dans ce projet ont permit de tirer des conclusions très utiles pour de prochains projets~:
\begin{itemize}
	\item les \gls{rnn} sont très lents à entraîner~;
	\item la maîtrise du pré-traitement est fondamentale pour obtenir des bons résultats~;
	\item pour utiliser une architecture multi-échelle comme celle proposée, il vaut mieux entraîner un modèle simple en premier lieu.
\end{itemize}\hspace{1em}

C'est d'un commun accord avec notre maître de stage que nous avons décidé de basculer sur le \gls{project_papud}.

En conclusion, le projet à abouti sur le rejet de l'architecture proposée.
Néanmoins, ce résultat à permit de cerner les principaux écueils de la réalisation d'un \gls{lm} multi-échelle, et à ainsi permit un meilleur déroulement du projet suivant.


%%%%
\section{Apport personnel du projet}
La réalisation de ce projet nous à permit d'approfondir largement nos connaissances en \gls{dl}, et de nous habituer aux problématique de la création et de l'utilisation de \glspl{nn}.

\section{Discussion et perspectives}
%TODO
