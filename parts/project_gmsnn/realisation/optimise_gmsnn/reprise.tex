\subsection{Sauvegarde, interruption et reprise de l'entraînement}\label{subsec:gmsnn_save}
Une fonctionnalité s'est très vite imposée comme essentielle~: la sauvegarde du \gls{model} et l'interruption et reprise de l'entraînement.

En effet, avec des entraînements très lents et donc longs, il était nécessaire de pouvoir suspendre l'entraînement afin de répartir le temps d'entraînement sur plusieurs sessions de plusieurs heures. De plus, les sauvegardes permettent de conserver le modèle une fois qu'il est entraîné.

Le système mis en œuvre permet d'effectuer cycliquement des sauvegardes du modèle ainsi que de l'état de l'entraînement, permettant ainsi une reprise en l'état de l'entraînement.

Pour la réalisation du système, le principal obstacle a été le dysfonctionnement initial des outils fournis par \gls{pytorch}.
La conception d'un système de sauvegarde personnalisé est alors apparue nécessaire, mais malheureusement complexe.
Cependant, une mise à jour majeure de la librairie a résolu le problème, et c'est finalement avec les outils de \gls{pytorch} que le système de sauvegarde a été mis en œuvre.

Plus de détails sur le système de sauvegarde sont disponibles dans le rapport de l'annexe \ref{subsec:save} (\autopageref{subsec:save}).