\subsection{Sauvegarde, interruption et reprise de l'entraînement}\label{subsec:gmsnn_save}
Une fonctionnalité s'est très vite détachée comme essentielle~: la sauvegarde du \gls{model}, et l'interruption et la reprise de l'entraînement.

En effet, avec des entraînements très lents et donc longs, il était nécessaire de pouvoir suspendre l'entraînement afin de répartir le temps d'entraînement sur plusieurs sessions de plusieurs heures. De plus, les sauvegardes permettaient de conserver le modèle une fois qu'il est entraîné.

Le système implémenté permet d'effectuer cycliquement des sauvegarde du modèle ainsi que de l'état de l'entraînement, permettant ainsi une reprise en l'état de l'entraînement.

Pour la réalisation du système, le principal obstacle a été le malfonctionnement initial des outils fournis par \gls{pytorch}.
Cela a poussé à la conception d'un système de sauvegarde personnalisé mais malheureusement assez complexe.
Cependant, la mise à jour majeure de la librairie qui c'est déroulé à point nommé à résolu le problème, et c'est avec les outils de \gls{pytorch} que le système de sauvegarde à finalement été implémenté.

Plus de détail sur le système de sauvegarde sont disponibles dans le rapport de l'annexe \ref{subsec:save} (\autopageref{subsec:save}).