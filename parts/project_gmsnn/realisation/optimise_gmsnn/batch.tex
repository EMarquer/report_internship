
\newpage
\subsection{Entraînement par exemples simultanés} \label{subsec:optibatch}
Une fois le problème des fuites de mémoire résolu, la première optimisation mise en place est l'utilisation de \glspl{batch} parallèles.

\subsubsection{\Gls{batch}} \label{def:batch}
Un \gls{batch} (anglais pour lot), est un paquet d'exemples successifs.

Le découpage des données en \glspl{batch} permet de répartir l'apprentissage tout au long de l'étude des données.
Cela permet d'atteindre de meilleures performances.
L'algorithme basé sur ce principe est appelé \foreign{mini-batch} \autocite{batch}.

L'utilisation de cet algorithme est une optimisation répandue pour l'entraînement de \glspl{nn} \autocite{batch}.
Elle est souvent couplée à un entraînement simultané sur plusieurs \glspl{batch}, décrit dans la partie suivante.

\subsubsection{\Glspl{batch} parallèles}
Un entraînement par \glspl{batch} parallèles permet de calculer le résultat de plusieurs exemples simultanément.
On calcule ensuite la différence de chaque résultat avec le résultat attendu correspondant.
Enfin, on met à jour le \gls{model} en fonction des différences observées sur l'ensemble des \glspl{batch}.
Au final, les calculs des résultats sont parallélisés, et le coût de la mise à jour est mis en commun entre les \glspl{batch}.

Le temps de calcul est ainsi réduit drastiquement et la qualité de l'entraînement est augmentée, au prix d'une plus grande utilisation de la mémoire et de la puissance de calcul.

La version de l'algorithme de parallélisation utilisée est similaire à celle décrite dans l'article~\autocite{batch_parallel}. Elle est gérée nativement par \gls{pytorch}.

\subsubsection{Conflit entre les \glspl{batch} parallèles et l'architecture \gls{gmsnn}}
Cependant, le découpage en \glspl{batch} pose un problème majeur avec l'architecture \gls{gmsnn}~: cette dernière est basée sur la continuité des exemples fournis, et l'utilisation de \glspl{batch} brise la continuité en introduisant un parallélisme.

Une analyse approfondie a permis d'établir une méthode pour résoudre ou écarter la majorité des aspects du problème. Après consultation, nous avons décidé d'utiliser l'entraînement par \glspl{batch} malgré les problèmes non encore résolus.

Le rapport de l'annexe \ref{subsec:batch} (\autopageref{subsec:batch}) contient les détails de l'analyse des problèmes théoriques de l'utilisation de \glspl{batch} avec l'architecture \gls{gmsnn}. Les rapports des tests de la solution retenue suite à l'analyse sont contenus dans les annexes \ref{subsec:batch_1} (\autopageref{subsec:batch_1}), \ref{subsec:batch_2} (\autopageref{subsec:batch_2}), \ref{subsec:batch_3} (\autopageref{subsec:batch_3}) et \ref{subsec:batch_4} (\autopageref{subsec:batch_4}).

L'annexe \ref{subsec:memuse} (\autopageref{subsec:memuse}) contient les détails de l'impact de la taille des \glspl{batch} et du nombre de \glspl{batch} sur la consommation de mémoire.