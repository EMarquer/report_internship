%nouvelle tentative augmenter params
%aucun impact probant malgré tentative variées
\subsection{Augmentation du nombre de paramètres}\label{subsec:optiparam}
Pour rappel, les \glspl{parameter} du modèle sont des valeurs qui varient au long de l'entraînement du \gls{model}.

Le fait d'augmenter ces \glspl{parameter} augmente la qualité de l'apprentissage, et la précision du modèle appris. Mais cela se fait au prix d'un volume plus important du \gls{model}, et d'une augmentation des calculs nécessaires pour utiliser et entraîner le \gls{model}. Cela se manifeste par un entraînement plus lent et une consommation mémoire plus élevée.

Cependant, grâce aux optimisations mises en place durant la résolution des fuites mémoires (voir \autoref{subsec:optimem}), ces coûts restent raisonnables.

Il existe plusieurs façon de mettre en place cette optimisation~:
\begin{itemize}
	\item augmenter le nombre de neurones par couches~;
	\item augmenter le nombre de couches dans le \glspl{rnn} qui compose chaque \og échelle\fg{}.
\end{itemize}

%\subsubsection{Conclusion}
Ces deux pratiques ont été testées, et aucune n'a apporté d'amélioration de qualité de l'apprentissage, tout en multipliant le temps d'entraînement.
En résumé, ces améliorations apportaient des coûts supplémentaires sans aucun bénéfices. Par conséquent, aucune de ces améliorations n'a été conservée.