\subsection[Agrégation des sorties des couches]{Agrégation des sorties des couches~: d'une stratégie additive à une concaténation}\label{subsec:addcat_}
La première optimisation à été de changer la façon de regrouper les informations de toutes les \og échelles\fg{} avant de les transmettre au module produisant la distribution de probabilité.

Initialement, les sorties de toutes les \og échelles\fg{} étaient sommées. Cela permettait de maintenir des \glspl{tensor} de dimensions uniformes quel que soit le nombre d'\og échelle\fg{} (voir figure \ref{fig:add}).

Après discussion avec notre maître de stage, la stratégie d'agrégation à été changé en une concaténation des sorties.

Comme montré dans la figure \ref{fig:cat}, la taille du \gls{tensor} concaténé change en fonction du nombre d'entrées.
La manipulation de \glspl{tensor} de taille non fixée est très ardue dans ce cas précis, bien que nous ne développerons pas plus avant les raisons de cette difficulté.

Cela à nécessité l'abandon de la propriété de croissance à l'infini de l'architecture (décrite \autoref{inf_growth}), au profit d'un nombre maximal d'échelles défini à l'avance ou déterminée à l'aide d'une formule en fonction des données disponibles (décrite \mbox{\autoref{growth_formula}}).

\begin{figure}[ht]
	\begin{subfigure}{0.45\textwidth}
		\centering
		\scalebox{1}{\def\layersep{5em}
\begin{tikzpicture}[shorten >=1pt,->,draw=black, node distance=\layersep]
\tikzstyle{every pin edge}=[<-,shorten <=1pt]
\tikzstyle{block}=[minimum size=2em];
\tikzstyle{value}=[rectangle, fill=green!50,block];
\tikzstyle{operation}=[block, circle,inner sep=0pt, fill=red!50];
\tikzstyle{nonlinearity}=[rectangle,block, fill=blue!50];
\tikzstyle{annot} = [text width=6em, text centered]

% Draw the input layer nodes
\foreach \name / \y in {1,...,3}
% This is the same as writing \foreach \name / \y in {1/1,2/2,3/3,4/4}
\node[value, label={[]north:{\'{E}chelle \y}}] (I-\name) at (0,-2*\y) {\y};
\node[value, label={[]north:{\'{E}chelle 4}}, opacity=.5] (I-4) at (0,-8) {4};

% Draw the output layer node
\node[operation, right of=I-2] (ope) {{\Large +}};
\node[value, right of=ope, label={[]north:$1+2+3+4$}](cat){};
% Draw the output layer node
%\node[nonlinearity, right of=cat] (lin) {Lin};
%\node[annot, right of=lin, text width=7em,xshift=2em ] (out) {Distribution de probabilit\'{e}s};

% Connect every node in the input layer with every node in the
% hidden layer.
\foreach \source in {1,...,3}
\path (I-\source.east) edge (ope);
\path (I-4.east) edge[dashed, opacity=.5] (ope);
\path (ope) edge (cat);
%\path (cat) edge (lin);
%\path (lin) edge (out);
\end{tikzpicture}}
		\caption[Stratégie d'agrégation additive]{Stratégie d'agrégation additive. Les sorties sont additionnées afin de former un nouveau \gls{tensor}.}\label{fig:add}
	\end{subfigure}
	\begin{subfigure}{0.45\textwidth}
		\centering
		\scalebox{1}{\input{plots/cat}}
		\caption[Stratégie d'agrégation par concaténation]{Stratégie d'agrégation par concaténation. Les sorties sont mises côte-à-côte afin de former un nouveau \gls{tensor}.}\label{fig:cat}
	\end{subfigure} 
	\caption{Stratégies d'agrégation}
\end{figure}

La stratégie par concaténation est plus lente en terme de temps de calcul que la stratégie additive, cependant pour le même temps de calcul elle permet d'obtenir de meilleurs résultats (voir figure \ref{fig:addcat}).

Voir l'annexe \ref{subsec:addcat}, \autopageref{subsec:addcat}, pour plus de détail sur le choix de la stratégie d'agrégation. 

\begin{figure}[H]
	\centering
	\includegraphics[width=\textwidth]{parts/appendix/reports-gmsnn/docs_esteban-latex/test_reports/comparative-bpc-msnn-det-msnn-cat.png}
	\caption[Performances comparées des stratégies additive et par concaténation]{Performances comparées des stratégies par concaténation \textit{(msnn-cat)} et additive \textit{(msnn-add)}. Le temps de calcul alloué est identique. Avec la concaténation on entraîne le modèle sur 1/4 des données, avec l'addition on l'entraîne 5 fois sur l'ensemble des données. Avec la concaténation, on obtient une BPC de 3.5, alors qu'on obtient une BPC de 4 avec l'addition.}\label{fig:addcat}
\end{figure}