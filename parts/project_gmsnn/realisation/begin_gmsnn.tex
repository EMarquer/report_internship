\section{Implémentation du nouveau modèle}
\subsection{Travail effectué}\label{def:lstm_2}
La troisième partie du projet a été la réalisation d'un prototype de l'architecture \gls{gmsnn}, basé sur la ré-implémentation simplifiée du \gls{model} \gls{soa}.

L'architecture du \gls{gmsnn} est celle du \gls{model} ré-implémenté, mais le \gls{rnn} y est remplacé par le \gls{module_gmsnn} (voir figure \ref{fig:reimplement_gmsnn}, \autopageref{fig:reimplement_gmsnn}). C'est sur ce nouveau \gls{module_gmsnn} que le reste du travail au cours du \gls{project_gmsnn} a été effectué.

De la même façon qu'avec la version ré-implémentée, le modèle prend en entrée des caractères, et produit des probabilités sur le caractère qui apparaîtra ensuite.

Ce prototype a permis de mettre en place les mécanismes de base du modèle.

Durant cette étape, nous avons mis en place l'architecture multi-échelle avec deux mécanismes fondamentaux, la transmission de l'information d'une échelle à l'autre et l'agrégation de l'information de toutes les échelles.

Chaque échelle qui compose le \gls{module_gmsnn} est un \gls{lstm}, qui est le \gls{rnn} utilisé dans le \gls{model} d'origine.

\begin{figure}[ht]
	\centering
	\scalebox{1}{\usetikzlibrary{calc}

\begin{tikzpicture}[shorten >=2pt,->,draw=black, node distance=7em]
    \tikzstyle{every pin edge}=[<-,shorten <=2pt]
	\tikzstyle{module}=[minimum size=3em, fill=gray!50!]
	\tikzstyle{char}=[module, circle, fill=green!50!]%, label={below:\usebox\mydictbox}]
	\tikzstyle{text label}=[rectangle, text centered, text width=7em, node distance=5em]
	\tikzstyle{nn}=[rectangle, module, fill=orange!50!]
	\tikzstyle{rnn}=[nn, fill=red!50!]

	\node[char] (char) {\$};
	\node[nn, right of=char] (emb) {Emb};
	\node[rnn, right of=emb] (rnn) {GMSNN};
	\draw[->] (rnn.north) to [out=90,in=90, looseness=2] ($(rnn.west) + (-0.5,0)$) to [out=-90,in=-90, looseness=2] (rnn.south);
	\node[nn, right of=rnn] (lin) {Lin};
	\node[char, right of=lin] (out) {P(\$$_{+1}$)};
	\draw[->] (char) to (emb);
	\draw[->] (emb) to (rnn);
	\draw[->] (rnn) to (lin);
	\draw[->] (lin) to (out);
	%\node[char] (char) {\$};


    \node[text label, above of=char] (charl) {Caract\`{e}re};
    \node[text label, above of=emb] (embl) {Encodage du caract\`{e}re};
    \node[text label, above of=rnn] (rnnl) {RNN};
    \node[text label, above of=lin] (linl) {Transformation en distribution de probabilit\'{e}};
    \node[text label, above of=out] (outl) {Probabilit\'{e}s pour chaque caract\`{e}re d'apparaitre};
\end{tikzpicture}}
	\caption[Architecture du \glsentrytext{model} GMSNN]{Architecture du \glsentrytext{model} GMSNN}\label{fig:reimplement_gmsnn}
\end{figure} 

\pagebreak
\subsection{Transmission de l'information}
Pour rappel, la transmission de l'information se fait d'une couche donnée à la couche immédiatement supérieure.
Cette transmission se fait périodiquement, en fonction d'un nombre appelé fréquence de transmission.

%Par exemple, pour fréquence de transmission de $3$~: 
%\begin{itemize}
%	\item toutes les $3$ entrées de la couche $n-1$, la couche $n$ reçoit de l'information de la couche $n-1$~;
%	\item toutes les $3$ entrées de la couche $n$ (soit toutes les $3^2$ entrées de la couche $n-1$), la couche $n+1$ reçoit de l'information de la couche $n$.
%\end{itemize}
%\vspace{1em}

Dans un premier temps, il a fallu choisir quelle information transmettre d'une échelle à l'échelle supérieure. En effet, les \glspl{rnn} produisent à la fois une sortie et un \gls{tensor} contenant leur mémoire. L'utilisation de l'\gls{embedding} a été écartée initialement, car elle n'est pas en accord avec le principe d'abstraction de l'architecture proposée.

%\pagebreak
Le choix s'est porté sur le \gls{tensor} contenant la mémoire, qui contient donc les informations abstraites par l'échelle, contrairement à la sortie qui  contient uniquement les informations pour prédire le caractère suivant.

%\begin{figure}[h]
%	\begin{subfigure}{0.45\textwidth}
%		\centering
%%		\scalebox{1}{\usetikzlibrary{calc}

\begin{tikzpicture}[shorten >=2pt,->,draw=black!50, node distance=7em]
    \tikzstyle{every pin edge}=[<-,shorten <=2pt]
	\tikzstyle{module}=[minimum size=3em, fill=gray!50!]
	\tikzstyle{char}=[module, circle, fill=green!50!]
	\tikzstyle{text label}=[rectangle, text centered, text width=7.5em, node distance=7em]
	\tikzstyle{nn}=[rectangle, module, fill=orange!50!]
	\tikzstyle{rnn}=[nn, fill=red!50!]

	\node[rnn, pin={[pin edge={<-}, pin distance=3em]south:Entr\'{e}e}] (rnn1) {RNN};
	\node[rnn, above of=rnn1] (rnn2) {RNN};
	\node[rnn, above of=rnn2, pin={[pin edge={->,dashed}, pin distance=3em]north:}] (rnn3) {RNN};

	\foreach \n in {1,...,3}
		\draw[->] (rnn\n.east) to [out=0,in=0, looseness=2] ($(rnn\n.south) + (0,-.5)$) to [out=180,in=180, looseness=2] (rnn\n.west);

	\path[->,dashed] (rnn1)  edge coordinate (@aux) (rnn2);
	\path [late options={name=@aux, pin={[pin edge={-}, text width=10.5em, pin distance=5em]0:Transmission toutes les $n$ entr\'{e}es de l'\'{e}chelle 1}}];
	\path[->,dashed] (rnn2)  edge coordinate (@aux) (rnn3);
	\path [late options={name=@aux, pin={[pin edge={-}, text width=10.5em, pin distance=5em]0:Transmission toutes les $n$ entr\'{e}es de l'\'{e}chelle 2}}];

	\foreach \n in {1,...,3}
	    \node[text label, left of=rnn\n] (rnn\n l) {\'{E}chelle \n};
	\node[right of=rnn3, node distance=10.7em, text width=11em, yshift=3em] (rnn l) {$n$ est la fr\'{e}quence de transmission};
\end{tikzpicture}}
%		\caption{Architecture en blocs simples}
%	\end{subfigure}
%	\begin{subfigure}{0.45\textwidth}
%		\centering
%%		\scalebox{1}{\input{plots/base_gmsnn_unfolded}}
%		\caption{Architecture en blocs dépliés}
%	\end{subfigure} 
%	\caption[Architecture fondamentale du \glsentrytext{gmsnn}]{Architecture fondamentale de \gls{module_gmsnn}}\label{fig:module_gmsnn_base}
%\end{figure}

L'annexe \ref{subsec:testms} (\autopageref{subsec:testms}) présente le rapport sur le prototype. 