\chapter{Présentation du laboratoire et de l'équipe}
\section{Généralités}
C'est dans le laboratoire du \gls{loria} que le stage s'est déroulé, au sein de l'équipe \gls{synalp} dirigée par M. Christophe Cerisara, notre maître de stage.

\section{Le \glsentrytext{loria}}
Le \glsfirst{loria} est une Unité Mixte de Recherche (UMR 7503), commune à plusieurs établissements : le \gls{cnrs}, l’\gls{ul} et l'\gls{inria}.
Depuis sa création en 1997, le \gls{loria} se concentre sur les science informatiques, que ce soit par la recherche fondamentale ou appliquée.

\subsection*{Structure administrative du \glsentrytext{loria}}
Il est dirigé par quatre instances\autocite{organisation_loria}:
\begin{itemize}
	\item \textbf{l'équipe de direction}~: composée du directeur, de son adjoint, de la responsable administrative, et de l'assistante de direction~; assiste le directeur dans la prise et la mise en œuvre des décisions~;
	\item \textbf{le conseil scientifique}~: composé du directeur du laboratoire, des deux directeurs adjoints et des scientifiques responsables des cinq départements du laboratoire composée de membres élus pour 4 ans et de membres nommés~; assiste le directeur dans la prise et la mise en œuvre des décisions~;
	\item \textbf{le conseil de laboratoire}~: composé de membres élus pour 4 ans et de membres nommés~; émet des avis et conseille le directeur sur toutes les questions concernant l’UMR~;
	\item \textbf{l'\gls{areq}}.
\end{itemize}

\subsection*{La recherche au sein du \glsentrytext{loria}}
Le \gls{loria} est l'établissement qui héberge l'équipe \gls{synalp}, parmi de nombreuses autres équipes.

Ce laboratoire regroupe 28 équipes de recherche, structurées en 5 départements en fonction de leur domaine d'étude.

\begin{figure}[h]
	\centering
	%\rotatebox{90}{\scalebox{0.9}{\usetikzlibrary{arrows,shapes,positioning,shadows,trees}

\tikzset{
  basic/.style  = {draw, rounded corners=2pt, thin, text width=10em, drop shadow, rectangle},
  root/.style   = {basic, align=center,
                   fill=blue!30},
  level 1/.style={sibling distance=12em, level distance=5em},
  level 2/.style = {basic, align=center, fill=green!30},
  level 3/.style = {basic, rounded corners=2pt, thin, align=left, fill=pink!60, text width=16em, yshift=-2em}
}

\begin{tikzpicture}[
  edge from parent path={[->](\tikzparentnode.south) -- ++(0,-0.5em)
			-| (\tikzchildnode.north)},
  >=latex]

% root of the the initial tree, level 1
\node[root](loria) {LORIA}
% The first level, as children of the initial tree
  child {node[level 2] (c1) {D\'{e}partement 1:\\Algorithmique, calcul, image et g\'{e}om\'{e}trie}}
  child {node[level 2] (c2) {D\'{e}partement 2:\\M\'{e}thodes formelles}}
  child {node[level 2] (c3) {D\'{e}partement 3:\\R\'{e}seaux, syst\`{e}mes et services}}
  child {node[level 2, fill=green!60!] (c4) {D\'{e}partement 4:\\Traitement automatique des langues et des connaissances}}
  child {node[level 2] (c5) {D\'{e}partement 5:\\Syst\`{e}mes complexes, intelligence artificielle et robotique}};

% The second level, relatively positioned nodes
\begin{scope}[every node/.style={level 3}]
% \node [below of = c1, xshift=1em] (c11) {Setting shape};
% \node [below of = c11] (c12) {Choosing color};
% \node [below of = c12] (c13) {Adding shading};
\node [below of = c4, xshift=5em, yshift=-0.75em] (c41) {CELLO:\\\textit{Computational Epistemic Logic in LOrraine}};
\node [below of = c41] (c42) {MULTISPEECH:\\Analyse, perception et reconnaissance automatique de la parole};
\node [below of = c42] (c43) {ORPAILLEUR:\\Extraction et représentation de connaissances};
\node [below of = c43] (c44) {READ:\\Reconnaissance de l’écriture \& analyse de documents};
\node [below of = c44] (c45) {SMarT:\\\textit{Statistical Machine Translation \& Speech Modelization and Text}};
\node [below of = c45, yshift=0.5em] (c46) {SEMAGRAMME:\\Linguistique computationnelle};
\node [below of = c46, fill=red!40] (c47) {SYNALP:\\Traitement automatique des langues naturelles par méthode statistique et symbolique};
\end{scope}

% lines from each level 1 node to every one of its "children"
% \foreach \value in {1,...,5}
%   \draw[->] (loria.south) -| (c\value.north);
\foreach \value in {1,...,7}
  \draw[->] (c4.208) |- (c4\value.west);
\end{tikzpicture}}}
	\scalebox{1}{\usetikzlibrary{arrows,shapes,positioning,shadows,trees}

\tikzset{
  basic/.style  = {draw, rounded corners=2pt, thin, text width=10em, drop shadow, rectangle},
  root/.style   = {basic, align=center,
                   fill=blue!30},
  level 1/.style={level distance=5em},
  level 2/.style = {basic, align=center, fill=green!30, yshift=-2em},
  level 3/.style = {basic, rounded corners=2pt, thin, align=left, fill=pink!60, text width=16em, yshift=-2em}
}

\begin{tikzpicture}[
  edge from parent path={[->](\tikzparentnode.south) -- ++(0,-0.5em)
			|- (\tikzchildnode.west)},
  >=latex]

% root of the the initial tree, level 1
\node[root](loria) {LORIA}
% The first level, as children of the initial tree
  child {node[level 2, , xshift=14em, yshift=2em] (c1) {\textbf{D\'{e}partement 1}\\Algorithmique, calcul, image et g\'{e}om\'{e}trie}}
  child {node[level 2, below of = c1] (c2) {\textbf{D\'{e}partement 2}\\M\'{e}thodes formelles}}
  child {node[level 2, below of = c2] (c3) {\textbf{D\'{e}partement 3}\\R\'{e}seaux, syst\`{e}mes et services}}
  child {node[level 2, below of = c3, fill=green!60!, yshift=-1.25em] (c4) {\textbf{D\'{e}partement 4}\\Traitement automatique des langues et des connaissances}}
  child {node[level 2, below of = c4, yshift=-2em] (c5) {\textbf{D\'{e}partement 5}\\Syst\`{e}mes complexes, intelligence artificielle et robotique}};

% The second level, relatively positioned nodes
\begin{scope}[every node/.style={level 3}]
% \node [below of = c1, xshift=1em] (c11) {Setting shape};
% \node [below of = c11] (c12) {Choosing color};
% \node [below of = c12] (c13) {Adding shading};
\node [below of = c4, xshift=18em, yshift=22em] (c41) {\textbf{CELLO}\\\textit{Computational Epistemic Logic in LOrraine}};
\node [below of = c41] (c42) {\textbf{MULTISPEECH}\\Analyse, perception et reconnaissance automatique de la parole};
\node [below of = c42] (c43) {\textbf{ORPAILLEUR}\\Extraction et représentation de connaissances};
\node [below of = c43] (c44) {\textbf{READ}\\Reconnaissance de l’écriture \& analyse de documents};
\node [below of = c44] (c45) {\textbf{SMarT}\\\textit{Statistical Machine Translation \& Speech Modelization and Text}};
\node [below of = c45, yshift=0.5em] (c46) {\textbf{SEMAGRAMME}\\Linguistique computationnelle};
\node [below of = c46, fill=red!40] (c47) {\textbf{SYNALP}\\Traitement automatique des langues naturelles par méthode statistique et symbolique};
\end{scope}

% lines from each level 1 node to every one of its "children"
% \foreach \value in {1,...,5}
%   \draw[->] (loria.south) -| (c\value.north);
\foreach \value in {1,...,7}
  \draw[->] (c4.east)  -- ++(2em,0em) |- (c4\value.west);
\end{tikzpicture}}
	\caption{Organigramme des départements du \gls{loria}, et des équipes du département 4}
	\label{fig:dep_org}
\end{figure}

La structure générale du \gls{loria} en départements et plus en détail du département 4 est représentée sur l'organigramme de la \autoref{fig:dep_org}. % (\autopageref{fig:dep_org}). % TODO remove pageref
Les thématiques générales de chaque département et des équipes du département 4 y sont présentées brièvement.
Un organigramme complet du \gls{loria} est disponible sur {le site du laboratoire \autocite{org_loria}}.

%\subsection*{Généralités}
\section{L'équipe \glsentrytext{synalp}}
L'équipe \glsfirst{synalp} est une équipe de recherche affiliée à la fois au \gls{cnrs} et à l'\gls{ul}.
Elle fait partie, avec 6 autre équipes, du département 4, dédié au traitement automatique des langues (\gls{nlp}) et des connaissances.

\subsection*{Membres}
L'équipe \gls{synalp} est sous la direction de M. Christophe Cerisara, et comporte actuellement 12 membres permanents, une dizaine de doctorants et d'ingénieurs, et approximativement 6 stagiaires à l'heure de l'écriture de ce mémoire.

\subsection*{Thématiques de recherche}
La recherche dans \gls{synalp} se concentre sur les approches hybrides, symboliques et statistiques du \gls{nlp}, ainsi que sur les applications de ces approches.
% TODO expliquer, ou renvoyer vers le cadre scientifique

Ainsi, les principaux sujets de recherche de l'équipe sont les \gls{lm}, les \gls{formal grammars}, la \gls{computational semantics}, le \gls{speech processing}, et les outils et ressources utilisés en \gls{nlp}.
% TODO expliquer, ou renvoyer vers le cadre scientifique

Ce stage s'inscrit en particulier dans la réalisation de \gls{lm}, et l'élaboration d'outils et ressources utilisés en \gls{nlp}. Nous verrons en détail pourquoi dans le chapitre~\ref{ch:Projet}.

\section*{Pour en savoir plus}
Des informations plus détaillées sur le \gls{loria} sont disponible sur {le site du laboratoire \autocite{about_loria}}.
Par ailleurs, la liste complète des membres de l'équipe, ainsi que des informations plus détaillées sont disponible sur {le site de \gls{synalp} (en anglais) \autocite{about_synalp}}.