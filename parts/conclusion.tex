%un bilan professionnel du travail de stage doit être exposé. Celui-ci peut être complété par un bilan personnel en termes d’aptitudes améliorées, découvertes ou acquises.

%une partie, dite fermée, qui reformule l’ensemble de ce qui a précédé et qui montre que des réponses ont été apportées aux questions posées lors de l’introduction,

%une partie, dite ouverte, qui place le travail dans une perspective à moyen ou long terme, elle ne doit pas être de pure forme et éviter les poncifs et les banalités.



%Conclusion générale sur le déroulement du stage.
%	généralités, reprise des deux autres ccl
%Conclusion professionnelle sur le stage.
%	compétences nécessaires/possédées -> acquis + impact du non possédé (citer 'c'est pas la bonne facon d'apprendre le DL')
%Conclusion personnelle sur le stage.
%	objectif perso (recherche/labo, techiniques de l'IA, TAL)
%	plaisir, turfu, adaptation du stage à moi
%
%%%%%%%%%%%
%
%Technique vs scientifique
%
%%%%%%%%%%%
%Conclusion
%Discussion et perspectives
%	Apport personnel
%	Placement dans les Sciences-Cognitives
%	
%	
%%%%%%%%%%%%%%%%%%%%%%%%%%%%%%
%-----------------------------
%
%Conclusion
%	Généralité
%		-> généralités, reprise des deux autres ccl
%une partie, dite fermée, qui reformule l’ensemble de ce qui a précédé et qui montre que des réponses ont été apportées aux questions posées lors de l’introduction,
%
%	Professionnelle
%		-> compétences nécessaires/possédées -> acquis + impact du non possédé
%		(citer 'c'est pas la bonne facon d'apprendre le DL')
%un bilan professionnel du travail de stage doit être exposé. Celui-ci peut être complété par un bilan personnel en termes d’aptitudes améliorées, découvertes ou acquises.
%
%	Personnelle
%		-> objectif perso (recherche/labo, techiniques de l'IA, TAL)
%	plaisir, turfu, adaptation du stage à moi
%
%
%Discussion et perspectives
%	Apport personnel
%	Placement dans les Sciences-Cognitives
%	Poursuite du projet
%	
%%%%%%%%%%%%%%%%%%%%%%%%%%%%%%
%-----------------------------
%Conclusion
%	Généralité
%		-> généralités, reprise des deux autres ccl
%	Professionnelle
%un bilan professionnel du travail de stage doit être exposé. Celui-ci peut être complété par un bilan personnel en termes d’aptitudes améliorées, découvertes ou acquises.
%		-> compétences nécessaires/possédées -> acquis + impact du non possédé
%		(citer 'c'est pas la bonne facon d'apprendre le DL')
%Discussion et perspectives
%une partie, dite ouverte, qui place le travail dans une perspective à moyen ou long terme, elle ne doit pas être de pure forme et éviter les poncifs et les banalités.



%regret de ne pas avoir résolu space à l'avance
%à posteriori, solutions pour continuer awd

%\chapter{Rétrospective sur le stage}
%%généralités, reprise des deux autres ccl
%\section{Bilan sur }
%\section{Bilan professionnel}
%%un bilan professionnel du travail de stage doit être exposé. Celui-ci peut être complété par un bilan personnel en termes d’aptitudes améliorées, découvertes ou acquises.
%%compétences nécessaires/possédées -> acquis + impact du non possédé
%%(citer 'c'est pas la bonne facon d'apprendre le DL')
%\section{Bilan personnel}
%%objectif perso (recherche/labo, techiniques de l'IA, TAL)
%%plaisir d'avoir fait le stage, turfu, adaptation du stage à moi
%\chapter{Discussion et perspectives}
%%	Placement dans les Sciences-Cognitives
%\section{Poursuite du projet PAPUD}
%%	Poursuite du projet







%%%%%%%%%%%%%%%%%%%%%%%%%%%%%%%%%%%%%%%%%%%%%%%%%%%%%%%%%%%%%%%%%%%%%%%%%%%%%%%%
\chapter{Rétrospective sur le stage}
\section{Bilan sur le travail effectué}
% GMSNN
% PAPUD
% En qoi réponse à Intro
Durant ce stage, nous avons travaillés sur deux projets différents, le \gls{project_gmsnn} et le \gls{project_papud}.
L'un était une initiation aux techniques et technologies du \gls{dl}, ainsi qu'une exploration du potentiel d'une idée~; l'autre était une mise en pratique des connaissances et compétences acquises.

Même si le premier projet à été interrompu, le principal objectif qui était de sonder le potentiel de l'architecture à été rempli dans une certaine mesure. De plus, le second projet qui est la continuation du premier, à été mené à bien.

Ainsi, les éléments développés durant la première partie du stage ont été essentiels au succès de la dernière.

Au final, les résultats et outils fourni pour le \gls{project_papud} sont les fruits de tous le travail effectué durant ce stage.

\section{Bilan professionnel}
Ces deux projets nécessitaient pour leur réalisation de bonnes compétences en programmation Python, une connaissance théorique des mécanismes et théories principales du \gls{dl}, ainsi qu'une certaine maîtrise de PyTorch et de Grid5000.

Nous avions l'habitude du Python et de ses subtilités~; nous avions aussi une vague connaissance des théories et des mécanismes utilisés en \gls{dl}, de par notre formation et la lecture de quelques articles. % TODO cite lecun

Conscients de nos lacunes, nous avons dédié une période au début du stage à l'acquisition des connaissances nécessaires, que nous avons complétées tout au long du stage.

Nous avons aussi appris à maîtriser PyTorch et Grid5000, à gérer des entraînements de modèles d'\gls{dl}, ainsi qu'à  manipuler un des modèles réputé parmi les plus complexes de l'\gls{dl}, le \gls{rnn}.

La disponibilité limitée de notre maître de stage, qui est aussi le chef de l'équipe \gls{synalp}

L'essentiel du stage s'est déroulé en autonomie, avec la possibilité de consulter notre maître de stage. Nous avons ainsi eu une grande liberté de mouvement dans notre travail, dont il nous semble nous avons su tirer parti.

Nous avons été menés à rédiger beaucoup de comptes rendus, ainsi que présenter à des profanes les résultats de nos travaux. Cela nous à permit de développer nos capacités d'analyse, de synthèse et de vulgarisation.

En conclusion, durant ce stage, nous avons acquis un panel de connaissances et de savoirs-faire liées à la conception et à l'utilisation de \gls{nn}, ainsi qu'une méthodologie expérimentale typique de l'\gls{dl}.

%compétences nécessaires/possédées -> acquis + impact du non possédé
\section{Bilan personnel}
Ce stage était une aubaine pour nous, qui cherchions à la fois un stage permettant de découvrir le travail de recherche, et une occasion de découvrir les domaines de l'intelligence artificielle ou du \gls{nlp}, afin de décider de la poursuite de nos études.

Ce stage nous à permis d'avoir un aperçu du travail de chercheur. D'une part durant les premier mois, pendant lesquels nous avons explorer une problématique jamais explorée. D'autre part en travaillant avec l'équipe du \gls{project_papud}, qui nous à montré un aspect du travail de recherche plus appliqué.

Nous tenons à souligner qu'il à été très agréable de voir nos avis considérés et nos idées bien reçues tout au long projet.

Enfin, ce stage à été l'occasion de laisser libre court à un de nos vices cachés~: l'optimisation compulsive du code produit.

En définitive, ce stage à été une expérience extrêmement enrichissante, et ce fut un réel plaisir de manipuler le code, d'apprendre et de relever les différent défis qui se sont présentés, en compagnie de notre maître de stage, des collaborateurs du \gls{project_papud} et de nos camarades stagiaires et doctorants.

%%objectif perso (recherche/labo, techiniques de l'IA, TAL)
%%plaisir d'avoir fait le stage, turfu, adaptation du stage à moi








