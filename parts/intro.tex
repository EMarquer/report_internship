\chapter{Introduction}
%il faut placer l’objet du stage dans son contexte et ses enjeux, puis annoncer
%la mission et ses problématiques spécifiques. Le plan du rapport doit être 
%présenté et justifié.

\section[Contexte \& Enjeux]{Contexte et enjeux du stage}
%-> nn \& tal -> ML
%-> big amount of data

D'une part, depuis quelques années, l'\gls{dl} et les \glspl{nn} ont connu une explosion de popularité.
Ce qui se cache derrière cet engouement est la combinaison de théories relativement anciennes et d'avancées technologiques permettant la mise en œuvre desdites théories.

D'autre part, nous sommes à l'ère du \foreign{\og Big Data\fg{}}, et les quantités de \glspl{data} produites de nos jours sont bien au-delà de ce que nous pouvons gérer sans l'aide d'outils spécialisés.
Afin de produire des outils adaptés aux échelles actuelles, nous nous intéressons dans ce rapport à des grands volumes de \glspl{data} bien au-delà des volumes habituellement utilisés en \gls{dl}.

%Un des intérêt de ces méthodes d'\gls{ml} est la capacité à apprendre à partir de données restreintes.
Un des domaines exploitant les performances de ces nouveaux outils est le \gls{nlp}.
En particulier, nous nous intéresserons dans ce rapport à la création de \glspl{lm}.
%L'application au \gls{nlp} des \glspl{nn} qui nous intéresse particulièrement dans ce rapport est la création de \glspl{lm}.

Ainsi, l'axe principal de ce rapport est l'application des méthodes de l'\gls{dl} sur de grands volumes de \glspl{data}, par la réalisation de \glspl{lm}.
Cela implique des problématiques relativement classiques en développement de \gls{nn}~: le choix de l'architecture du réseau, de l'algorithme d'entraînement, mais aussi des questions plus pragmatiques d'optimisation liées au volume de \glspl{data}.

\newpage
\section[Objectifs]{Objectifs du stage}
%Les objectifs du stage sont doubles:
%- découvrir et apprendre à manipuler les NN
%- par la même occasion, explorer une idée d'architecture inovante: le Char-GMSNN-LM
%- au vu des résultats et de l'évolution du contexte, mission évolue aussi: fournir une première approche et un premier retour pour le projet PAPUD cas d'utilisation BULL
Deux objectifs se sont succédé durant le stage.

% Version résumée
L'objectif initial du stage était d'explorer une idée d'architecture innovante de \gls{nn}, imaginée par Mr.~Cerisara, le maître de stage. Il s'agit du \glsunset{project_gmsnn}\gls{project_gmsnn} (réseau de neurones récurrents multi-échelles croissant, \foreign{Growing Multi-Scale Recurrent Neural Network} en anglais, voir \autoref{ch:project_gmsnn}, \autopageref{ch:project_gmsnn}).
%Par la même occasion, ce stage est l'opportunité pour moi d'apprendre à manipuler les \gls{nn}.

Cependant, à l'issue du deuxième mois du stage, nous avons changé d'objectif.

Le nouvel objectif a été la réalisation d'un \gls{nn} et des outils nécessaires à son utilisation, en mettant à profit les connaissances acquises durant la première partie du stage.
Cette réalisation doit servir de base technique pour une partie du \glsunset{project_papud}\gls{project_papud} (\foreign{Profiling and Analysis Platform Using Deep Learning}, voir \autoref{ch:project_papud}, \autopageref{ch:project_papud}).

%Les tenants et aboutissants des deux phases, ainsi que l'explication des noms des projets, seront expliqués en détails dans le \autoref{ch:project_gmsnn} (\autopageref{ch:project_gmsnn}) et le \autoref{ch:project_papud} (\autopageref{ch:project_papud}).

%%\section{Terminologie et concepts fondamentaux}
\chapter{Terminologie et concepts fondamentaux}
Cette section est dédiée à la présentation et l'explication des théories, termes et concepts nécessaires à la compréhension du présent rapport.

L'objectif n'est pas de fournir des explications approfondies, mais de fournir les connaissances minimales nécessaires à la compréhension du contenu et des enjeux du stage.

Les termes présentés sont répertoriés dans le glossaire, mais aucun rappel de leur présence dans celui-ci n'est donné au long du texte.

%\subsection{\Glsentrytext{ml}, modèle, entraînement et données}
\section{\Glsentrytext{ml}, modèle, entraînement et données}

%\subsubsection{\Glsentrytext{ml}}
\subsection{\Glsentrytext{ml}}
\label{subsec:ml} \label{def:ml}
L'\glsfirst{ml} est un ensemble de \og méthodes [statistiques] permettant à une machine (au sens large) d'évoluer par un processus systématique, et ainsi de remplir des tâches difficiles ou problématiques par des moyens algorithmiques plus classiques\fg{}, d'après Wikipédia \autocite{wiki_ml}.

%\subsubsection{\Glsentrytext{model}}
\subsection{\Glsentrytext{model}} \label{def:model}
Un \gls{model} en \gls{ml} est la représentation du monde construite afin de répondre au problème à résoudre.

On parle d'entrées pour désigner les \glspl{data} fournies au \gls{model}, et de sorties pour désigner les \glspl{data} produites par ce \gls{model}.

%\subsubsection{Entraînement du \glsentrytext{model}}
\subsection{Entraînement du \glsentrytext{model}}
\label{def:training}
L'\gls{training} du \gls{model}, aussi appelé apprentissage, est le processus par lequel on adapte le \gls{model} de façon à mieux résoudre le problème.

%\subsubsection{Données d'entraînement}
\subsection{Données d'entraînement}
\label{def:preprocessing} \label{def:corpus} \label{def:data} \label{def:example} \label{def:epoch}
Pour entraîner un \gls{model}, il faut lui fournir des \glspl{data}.
%
Voici quelques termes courants se référant aux \glspl{data} d'\gls{training}~:
\begin{itemize}
	\item le \gls{corpus}~: l'ensemble des \glspl{data} d'\gls{training}~;
	\item un \gls{example}~: un fragment du \gls{corpus} utilisé pour entraîner un \gls{model}~;
	\item une \gls{epoch}~: un cycle complet d'\gls{training} sur le \gls{corpus}.
\end{itemize}
\vspace{1em}

Généralement, on effectue un \gls{preprocessing} pour les préparer. Cela peut consister à retirer les données erronées, à en adapter le format, à les anonymiser, ou encore à associer le résultat attendu aux données correspondantes.

%\subsection{\Glsentrytext{dl} et \glsentryplural{nn}}
\section{\Glsentrytext{dl} et \glsentryplural{nn}}
%\subsubsection{\Glsentrytext{dl}}
\subsection{\Glsentrytext{dl}}
\label{subsec:dl} \label{def:dl}
L'\gls{dl} représente un ensemble de techniques d'\gls{ml} consacrées aux \glspl{nn}.

Faisant partie des méthodes d'\gls{ml}, l'\gls{dl} regroupe à la fois les méthodes de création, d'entraînement, d'optimisation et d'utilisation des modèles basés sur des \glspl{nn}.

%\subsubsection{\Glsentrytext{nn}}
\subsection{\Glsentrytext{nn}}
\label{def:nn}
%Dans le cadre de l'\gls{dl}, un \gls{model} correspond généralement au \gls{nn}.

Un \glsfirst{nn} est un \gls{model} mathématique composé d'éléments interconnectés nommés neurones formels, % dont le comportement est vaguement semblable aux neurones biologiques.
par analogie lointaine avec le fonctionnement des neurones biologiques.
%par analogie avec les neurones biologiques dont le fonctionnement est proche.

Un \gls{nn} prend en entrée un \gls{tensor}, % (un type de \gls{matrice} spécifique utilisé en \gls{dl})
et produit un \gls{tensor} en sortie.
Toutes les valeurs contenues dans ces \glspl{tensor} sont des nombres.

%Un \glsfirst{nn} est un \og ensemble de neurones formels interconnectés permettant la résolution de problèmes complexes tels que la reconnaissance des formes ou le traitement du langage naturel, grâce à l'ajustement des coefficients de pondération dans une phase d'apprentissage.\fg{} D'après Futura\autocite{futura_nn}.

%\subsubsection{Architecture et modules}
\subsection{Architecture et modules}
\label{def:module} \label{def:architecture}
Pour des raisons de concision, nous considérons les \glspl{nn} comme des \glspl{module}, c'est-à-dire des boites noires,
sans nous intéresser à leur conception interne.

Nous parlons d'\gls{architecture} pour désigner à la fois la façon dont sont conçus les \glspl{nn} et la façon dont sont assemblés les \glspl{module} pour former un \gls{model}.

Pour décrire les \glspl{architecture}, nous utilisons des diagrammes considérant les \glspl{module} comme des blocs.

%\subsubsection{\Glsentrytext{parameter}}
\subsection{\Glsentrytext{parameter}}
\label{def:parameter}
Un \gls{parameter} pour un \gls{module} ou un \gls{model} est une valeur qui varie au cours de l'\gls{training}. Généralement, plus un \gls{model} possède de \glspl{parameter}, plus son \gls{training} consomme de ressources mais plus la qualité de l'apprentissage est élevée.

%\subsubsection{Principaux modules complémentaires}
%\paragraph{Dictionnaire}
%Afin de rendre les mots ou les caractères utilisables par un \gls{nn}, on les associe à des nombres. L'outil qui stocke la correspondance entre les nombres et les mots ou caractères s'appelle un dictionnaire
%
%\paragraph{Encodage des caractères}
%%TODO describe embedding
%Le module d'encodage des caractères, appelé \foreign{embeding layer} en anglais (littéralement \og couche d'inclusion\fg{}), produit une représentation apprise de chaque caractère sous forme de \gls{tensor}. Ce \gls{tensor} est appelé \gls{embedding}. Ce module, entraîné, peut apprendre des propriétés spécifiques à chaque caractère. Par exemple ce module peut apprendre que tel caractère est une consonne et que tel autre est un caractère de ponctuation. \label{def:embeding}

%Le module produisant la distribution de probabilité est un module linéaire\footnote{\og Module linéaire \fg{} est le nom donné aux modules composé d'un \gls{nn} intégralement connecté. Un réseau de neurones intégralement connécté signifie que chaque neurone d'une couche est connecté avec tous les neurones de la couche précédente. Il s'agit de l'architecture de \gls{nn} la plus simple.\label{def:fully_connected}\label{def:lin_module}}.
%Il transforme les informations produites par le réseau de neurones en probabilité pour chaque caractère d'être le prochain caractère de la séquence.

%\subsubsection{Principales \glsentryplural{architecture} de \glsentryplural{nn}}
\subsection{Principales \glsentryplural{architecture} de \glsentryplural{nn}}
\label{def:lstm} \label{def:rnn}

La liste suivante présente les principales \glspl{architecture} de \glspl{nn} utilisées dans ce rapport, classées en ordre croissant de complexité et de consommation de ressources.

\begin{itemize}
	\item Le \gls{nn} intégralement connecté ou module linéaire est un des  plus simples. Il est emblématique des \glspl{nn}.
	
	\item Le \gls{rnn} est une \gls{architecture} de \glspl{nn} particulièrement adaptée au traitement de séquences. Le \gls{rnn} traite des éléments les uns après les autres, et stocke dans une \og mémoire \fg{} des informations au fur-et-à-mesure. Il utilise les informations stockées pour améliorer la façon dont il traite les éléments suivants. 
	
	\item Le \gls{lstm} est un \gls{rnn} particulier. Équipé d'une mémoire supplémentaire, il plus puissant mais plus coûteux à entraîner qu'un \gls{rnn} classique.
\end{itemize}


%\paragraph{\Glsentrytext{nn} intégralement connecté ou module linéaire}
%Il s'agit d'un des types de \glspl{nn} les plus simples. Il est emblématique des \glspl{nn}.
%
%\paragraph{\Glsentrytext{rnn}} \label{def:rnn}
%Le \glsfirst{rnn} est une \gls{architecture} de \glspl{nn} particulièrement adaptée au traitement de séquences.
%Il traite des éléments les uns après les autres, et stocke dans une \og mémoire \fg{} des informations au fur-et-à-mesure.
%Il utilise les informations stockées pour améliorer la façon dont il traite les prochains éléments.
%
%\paragraph{\Glsentrytext{lstm}} \label{def:lstm}
%Le \glsfirst{lstm} est un \gls{rnn} particulier équipé d'une mémoire supplémentaire.
%\subsubsection{Mémoire et contexte}

%On peut noter qu'un \gls{lstm} consomme plus de ressources qu'un \gls{rnn} classique, qui en consomme plus qu'un \glspl{nn} intégralement connecté.

%On peut noter qu'un \glspl{nn} intégralement connecté consomme moins de ressources qu'un \gls{rnn} qui en consomme moins qu'un \gls{lstm}.

%\subsection{Traitement automatique des langues (\glsentrytext{nlp}) et \glsentrytext{lm}}
\section{Traitement automatique des langues (\glsentrytext{nlp}) et \glsentrytext{lm}}
%\subsubsection{Traitement automatique des langues (\glsentrytext{nlp})}
\subsection{Traitement automatique des langues (\glsentrytext{nlp})}
\label{subsec:nlp}\label{def:nlp}
Le \glsfirst{nlp} est une discipline qui s'intéresse au traitement des informations langagières par des moyens formels ou informatiques.

%\subsubsection{\Glsentrytext{lm}}
\subsection{\Glsentrytext{lm}}
Un \glsfirst{lm} est une \og distribution de probabilité sur une séquence de mots [ou de caractères]\fg{} (d'après Wikipédia \autocite{wiki_lm})
utilisée pour estimer la probabilité d'apparition du prochain mot ou caractère.

Autrement dit, c'est une représentation servant à prédire le mot suivant à partir des mots précédents (ou le caractère suivant à partir des caractères précédents).

\pagebreak
%\subsubsection{Contexte et dépendances}
\subsection{Contexte et dépendances}
Dans le cadre d'un \gls{lm}, le contexte est l'ensemble des informations disponibles hors du mot ou caractère à prédire.

On parle aussi de dépendances entre d'une part les mots ou caractères et d'autre part l'élément du contexte correspondant.

Parmi les informations contenues dans le contexte d'un mot, on peut trouver aussi bien le sens des mots environnants que la structure syntaxique de la phrase, ou encore des informations plus générales comme le fait que les chats sont des mammifères.

%Les informations contenues dans le contexte peuvent être aussi bien les mots environnants que la structure de la phrase ou des informations comme le genre du héro d'un livre.

On considère que, plus on a d'informations contextuelles, plus le \gls{lm} est précis. Par exemple, si on nous dit \og un chat \fg{}, il sera plus difficile de prédire la couleur du chat que si on nous dit \og un chat de couleur sombre \fg{}.

%\subsection{Performance et mesure}
\section{Performance et mesure}
Le dernier concept important du rapport est celui de performance des modèles produits.

La performance d'un modèle est évalué par trois composantes~:
\begin{itemize}
	\item la qualité maximale des résultats produits~; dans le cas d'un \gls{lm}, il s'agit de la qualité de la prédiction~;
	\item le temps d'entraînement nécessaire pour atteindre cette qualité (nombre d'\glspl{epoch}, durée, etc.)~;
	\item la consommation de ressources nécessaire pour atteindre cette qualité (mémoire, puissance de calcul, \dots).
\end{itemize}
\vspace{1em}

Afin d'évaluer la performance, des mesures ont été définies~:
\begin{itemize}
	\item pour mesurer la qualité du résultat, on utilise l'écart entre le résultat produit et le résultat attendu~; une mesure définie dans la littérature et utilisée dans les annexes est le BPC\footnote{Le BPC (\foreign{Bits Per Character} en anglais) \autocite{BPC} est originalement une mesure de la qualité de compression de texte, mais plusieurs papiers ont détourné cette mesure en s'en servant d'estimation de la qualité d'un \gls{lm} au niveau du caractère. Dans cet usage, le BPC est assimilable à une mesure de la précision des résultats du \gls{lm}. Plus le BPC est proche de 0 plus la qualité du \gls{lm} est élevé.}~;
	\item pour mesurer le temps d'entraînement, on le nombre d'\gls{epoch} et le temps nécessaire par \gls{epoch}, la durée totale en heures, etc.~;
	\item pour mesurer la consommation de ressources, on évalue l'espace mémoire occupée (en MiB ou GiB), la puissance de calcul utilisée (pourcentage de la puissance disponible), etc.
\end{itemize}
\vspace{1em}

Un \gls{model} optimal serait un modèle qui atteint d'excellents résultats  (l'écart entre ce qui est attendu et ce qui est produit le plus faible possible), le plus rapidement possible, en consommant le moins de ressources possibles.



\section[Plan]{Plan du rapport}

%%1. intro
%Dans un premier temps, nous avons présenté à la fois le contexte, les enjeux, et les objectif généraux du stage.
%
%%2. scientific framework
%Dans un second temps, nous étudierons plus en détail le cadre théorique du rapport, afin de définir les termes et concepts principaux utilisés dans ce rapport.
%
%%3. workplace
%Dans un troisième temps, nous allons nous attarder plus en détail sur les différentes entités impliquées, en particulier sur l'équipe \gls{synalp} et ses entités parentes, ainsi que sur le \gls{project_papud}.
%
%%4. project
%%4.1. awd-lstm-lm
%%4.2. papud-bull-nn
%%4.3. gobal (or 3.0.)
%Dans un quatrième temps, nous allons décrire plus en détail les deux aspects du stage: l'idée d'architecture de \gls{nn} et l'intérêt d'une telle architecture d'un côté, et le \gls{project_papud}, ses implications et la portée du stage dans ce projet. Nous verrons aussi en quoi le \gls{project_papud} est dans la continuité de la première partie du stage.
%
%%5. realisation
%%5.1. recherche doc
%%5.2. patch "reimplement"
%%5.3. build "growing"
%%5.4. optimise "growing"
%%5.5. conclusions on "growing" and "awd"
%%5.6. preparing "papud" \& basic model
%%5.7. understanding strange comportment \& preparing corpus
%%5.8. conclusions on "papud"
%Dans un cinquième temps, nous exposerons le déroulement pas-à-pas du stage, avec les obstacles rencontrés, la façon de les surmonter, et en quoi chaque résultat entraîne l'étape suivante. 
%
%%6. conclusion, additional internship
%Dans un sixième et dernier temps, nous ferons une rétrospective sur l'avancement des objectifs, la qualité des résultats obtenus, et les apports du stage.

%0. intro
%1. scientific framework
%2. workplace
%3. project
%3.1. awd-lstm-lm
%3.2. papud-bull-nn
%3.3. gobal (or 3.0.)
%4. realisation
%4.1. recherche doc
%4.2. patch "reimplement"
%4.3. build "growing"
%4.4. optimise "growing"
%4.5. conclusions on "growing" and "awd"
%4.6. preparing "papud" \& basic model
%4.7. understanding strange comportment \& preparing corpus
%4.8. conclusions on "papud"
%5. conclusion, additional internship

%1. intro
%2. scientific framework
%3. Projet GMSNN
%4. Projet PAPUD
%5. Conclusion
%6. Annexes

%%1. intro
%Dans un premier temps, nous avons présenté à la fois le contexte, les enjeux, et les objectif généraux du stage. Nous avons aussi présenté les principaux termes et concepts utilisés au long du rapport.
%
%%2. scientific framework
%Dans un second temps, nous définirons les termes et concepts principaux utilisés dans ce rapport.
%
%Dans un second temps nous allons développer les deux parties du stage, c'est-à-dire le \gls{project_gmsnn} et le \gls{project_papud}.
%
%Notamment, pour chacun d'eux, nous présenterons le contexte, les enjeux, et les entités impliquées dans le projet. Nous décrirons ensuite le \gls{model} réalisé avant de rapporter le travail effectué. Enfin, une conclusion résumera les points majeurs du projet.
%
%Dans un dernier temps, nous ferons un bilan de l'ensemble du travail réalisé pur en tirer les apports principaux.

%%%%%%%%%%

Nous avons présenté à la fois le contexte, les enjeux, et les objectif généraux du stage.

Tout d’abord, nous définirons les termes et concepts principaux utilisés dans ce rapport.

La complexité du stage est qu'il est composé de deux projets, le \gls{project_gmsnn} et le \gls{project_papud}, l'un étant la continuation de l'autre. En effet, les conclusions tirées du premier projet ont servi de base pour le second. C'est pourquoi, pour simplifier la lecture du présent rapport, chaque projet sera traité suivant le même plan.

Pour chacun d'eux, nous présenterons le contexte, les enjeux, et les entités impliquées dans le projet. Nous décrirons ensuite le \gls{model} réalisé avant de rapporter le travail effectué. Enfin, une conclusion résumera les points majeurs du projet.

Enfin, nous ferons un bilan de l'ensemble du travail réalisé pour en tirer les apports principaux.



