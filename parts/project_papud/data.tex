\chapter{Données disponibles}\label{ch:data_papud}
Les données disponibles sont des \glspl{log}. \label{def:log}

Ces fichiers sont composés de lignes de texte en anglais extrêmement structurées (comme on peut le voir sur le \autoref{cde:traindata}, \autopageref{cde:traindata}), générées par les programmes en cours d'exécution et qui donnent des informations sur les évènements qui surviennent sur la machine.
Ce sont par exemple des messages d'information, tels que \og le programme A a tenté de faire B\fg{}, ou \og l'utilisateur C a changé son mot de passe\fg{}.

Ces messages sont précédés d'informations contextuelles comme l'instant d'écriture du message et le programme d'où il provient.

De plus, ces \glspl{log} contiennent un nombre très grand de messages, et la quantité de données disponibles pour le projet dépasse les 400~GiB de texte brut. En comparaison, les données utilisées pour le projet précédent pesaient au maximum 100~MiB, soit 4~000 fois moins.

Un échantillon de 9~MiB des données disponibles a été utilisé pendant le projet.
Cela correspond à 70~131 lignes de \glspl{log}.

\section{Extrait des données d'entraînement}%dummy example
Les données utilisées sont confidentielles.
Ainsi, pour les protéger, l'extrait présenté ici a été modifié.
Entre autres, la date et le \foreign{timestamp} (nombre correspondant à la date) ont été remplacés par la date de début du stage et le \foreign{timestamp} correspondant, et l'utilisateur a été renommé \og OOOOOO \fg{}.

Dans le \autoref{cde:traindata} (\autopageref{cde:traindata}), on a, de gauche à droite~: le \foreign{timestamp}, la date, l'heure, l'utilisateur (OOOOOO), le processus (authpriv), le type de message (info), et le message.

\begin{lstlisting}[caption={Exemple de ligne extraite d'un des fichiers de journalisation},label=cde:traindata]
	1524463200 2018 Apr 23 08:00:00 OOOOOO authpriv info access granted for user root (uid=0)
\end{lstlisting}

\pagebreak
\section{Prétraitement des données}\label{def_dict_papud}
Le prétraitement des données consiste en~:
\begin{itemize}
	\item la suppression du \foreign{timestamp}, de la date, de l'heure, et de l'utilisateur~;
	\item le découpage du texte restant à une certaine longueur~;
	\item le remplissage des lignes n'atteignant pas cette longueur par un caractère spécifique nommé caractère de remplissage~;
	\item la transformation de tous les caractères en nombres, à l'aide d'un dictionnaire~; les caractères apparaissant peu fréquemment ou n'apparaissant pas dans le dictionnaire sont tous remplacés par un caractère spécial nommé \og caractère inconnu\fg{}.
\end{itemize}

\subsection{Traitement des codes hexadécimaux}\label{hex}
Par la suite, le remplacement de tout code hexadécimal\footnote{Le système hexadécimal est un système de numérotation en base 16, très utilisé en informatique. Les codes hexadécimaux sont écrits avec les chiffres de 0 à 9 et les lettres de A à F. On utilise généralement le suffixe \lstinline|0x| pour indiquer qu'un nombre est codé en hexadécimal.} comme \lstinline|0x005f| par un caractère spécifique du dictionnaire a été ajouté (voir \autoref{sec:spike}, \autopageref{sec:spike}).
De cette façon, \lstinline|0x005f| et \lstinline|0x0007| sont remplacés par le caractère  \lstinline|<hexX4>|, tandis que  \lstinline|0000005f| et  \lstinline|89abcdef| sont remplacés par  \lstinline|<hex8>|.
Cela permet de s'abstraire de la valeur du code sans perdre l'information liée à sa présence.

