\chapter{Conclusions sur le \glsentrytext{project_papud}}

\section{Retour sur le travail effectué}
Le \gls{project_papud} s'est déroulé sans accroc, et l'intégralité des objectifs ont été remplis.

Nous avons réalisé deux principales optimisations, outre le prétraitement des codes hexadécimaux et le choix du nombre de \glspl{batch}~:
\begin{itemize}
	\item la réalisation d'un outil d'optimisation du taux d'apprentissage~;
	\item la réalisation d'un outil multi-fichiers multi-processus de gestion du corpus.
\end{itemize} % TODO add refs
\hspace{1em}

De plus, nous avons porté une attention particulière à la documentation du code.
En effet, le travail effectué était le début de la réalisation du cas d'utilisation BULL du projet PAPUD.
Il est donc normal que nous ayons laissé une base de code propre et intégralement documentée pour notre successeur sur le projet.

%%%%%%%
En conclusion, les objectifs du projet ont été remplis. Le prototype réalisé a permis de fournir des premiers résultats encourageants pour la suite du \gls{project_papud}. De plus de nombreux outils ont été réalisées, qui seront réutilisables pour la suite du projet.

\section{Apport personnel du projet}
La réalisation de ce projet nous a permit de mettre en application l'ensemble des connaissances accumulées durant le projet précédent.

Nous avons ainsi mettre nos compétences à bon usage dans un projet de grande ampleur.

De plus, le travail avec une équipe de projet nous a permit de découvrir certains aspects de la réalisation d'un projet de recherche, cela a été une expérience enrichissante.

Enfin, la documentation du code et des outils, en particulier les derniers jours du projet qui y étaient dédiés, nous a permit de nous perfectionner dans l'art de la documentation en Python.
Par là nous entendons surtout les techniques de typage du code \autocite{pep483,pep484} et les pratiques de rédaction de \textit{docstring} (le format sous lequel est rédigé la documentation en Python).

\section{Discussion et perspectives}
\subsection{Continuation du travail}
Le projet s'est très bien déroulé, et de notre point de vue les résultats ont été satisfaisant.

Mais ce n'est que le début du projet, et il serait très intéressant de poursuivre le travail et de continuer à construire sur les bases que nous avons posé, forts de notre connaissance du projet et de notre expérience.

La réalisation d'un stage l'an prochain en serait l'opportunité idéale.

Parmi les pistes pour continuer le projet, on peut compter~:
\begin{itemize}
	\item l'amélioration du système de gestion de corpus~;
	\item l'étude de l'utilisation encore faible des ressources et l'optimisation de cet usage~;
	\item l'entraînement et le perfectionnement du modèle sur l'ensemble des données disponibles, jusqu'à atteindre les limites du modèle~;
	\item le début du travail sur la deuxième grande étape du projet (décrites \autoref{sec:solution}, \autopageref{sec:solution}).
\end{itemize}
