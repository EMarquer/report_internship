\chapter{Conclusions sur le \glsentrytext{project_papud}}

\section{Retour sur le travail effectué}
Le \gls{project_papud} s'est déroulé sans accroc, et l'intégralité des objectifs a été atteinte.

Nous avons réalisé deux principales optimisations, en plus du prétraitement des codes hexadécimaux et du choix du nombre de \glspl{batch}~:
\begin{itemize}
	\item un outil d'optimisation du taux d'apprentissage~;
	\item un outil multi-fichiers multi-processus de gestion du corpus.
\end{itemize} % TODO add refs
\hspace{1em}

De plus, nous avons porté une attention particulière à la documentation du code.
En effet, le travail effectué est la première phase de la réalisation du cas d'utilisation \gls{bull} du \gls{project_papud}. Nous avons donc laissé une base de code propre et intégralement documentée pour faciliter la poursuite du projet.

%%%%%%%
En conclusion, les objectifs du projet ont été atteints. Le prototype réalisé a permis de fournir des premiers résultats encourageants pour la suite du \gls{project_papud}. De plus, de nombreux outils ont été réalisés, qui seront réutilisables pour la suite du projet.

\section{Apport personnel du projet}
La réalisation de ce projet m'a permis d'appliquer l'ensemble des connaissances accumulées au cours du projet précédent.

J'ai ainsi utilisé mes compétences dans un projet de grande ampleur.

De plus, le travail avec une équipe de projet m'a permis d'avoir un aperçu du déroulement d'un projet de recherche, ce qui a été une expérience enrichissante.

Enfin, la documentation du code et des outils, en particulier pendant les derniers jours du projet qui y étaient dédiés, m'a permis de me perfectionner dans les techniques de documentation en Python.
Par là j'entends surtout les techniques de typage du code \autocite{pep483,pep484} et les pratiques de rédaction de \textit{docstring} (le format sous lequel est rédigée la documentation en Python).

\section{Discussion et perspectives}
\subsection{Poursuite du travail}
Le projet s'est très bien déroulé et, de notre point de vue, les résultats ont été satisfaisant.

Mais ce n'est que le début du projet, et il serait très intéressant de poursuivre le travail et de continuer à construire sur les bases que nous avons posé, forts de notre connaissance du projet et de notre expérience.

La réalisation d'un stage l'an prochain en serait l'opportunité idéale.

Parmi les pistes pour continuer le projet, on peut citer~:
\begin{itemize}
	\item l'amélioration du système de gestion de corpus~;
	\item l'étude de l'utilisation encore faible des ressources et l'optimisation de cet usage~;
	\item l'entraînement et le perfectionnement du modèle sur l'ensemble des données disponibles, jusqu'à atteindre les limites du modèle~;
	\item le début du travail sur la deuxième grande étape du projet, c'est à dire adapter le modèle à des dépendances plus longues (décrite \autoref{sec:solution}, \autopageref{sec:solution}).
\end{itemize}
