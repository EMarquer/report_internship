\section{Implémentation du modèle et transfert des outils de base}
L'étape suivante a été l'implémentation d'une version de base du \gls{model}, ainsi que des outils nécessaire pour l'utiliser.

Ces outils sont~:
\begin{itemize}
	\item un outil de prétraitement et de manipulation de \gls{corpus}~;
	\item un algorithme d'\gls{training} sur ces données, élaboré lors du projet précédent~;
	\item un outil d'analyse et de traçage des performances du modèle, qui est une version améliorée, plus fluide et plus puissante, des outils de visualisation du projet précédent (voir \autoref{sec:gmsnn_track}, \autopageref{sec:gmsnn_track})~;
	\item un système de sauvegarde et d'interruption de l'entraînement, similaire à celui du projet précédent (voir \autoref{subsec:gmsnn_save}, \autopageref{subsec:gmsnn_save})
\end{itemize}
\vspace{1em}

Les premiers résultats du modèle sont encourageants, autant sur sa rapidité que sur ses performance.
Ils sont présentés dans l'annexe \ref{anx:first} (\autopageref{anx:first}).
