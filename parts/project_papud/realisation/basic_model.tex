\section{Implémentation du modèle et transfert des outils de base}
Une fois le modèle défini, l'étape suivante à été l'implémentation d'une version de base, ainsi que des outils nécessaire pour les utiliser.

Ces outils sont~:
\begin{itemize}
	\item un premier outil de prétraitement de manipulation du \gls{corpus}~;
	\item un algorithme d'\gls{training} sur ces données, récupéré du projet précédent~;
	\item un outil d'analyse et de traçage des performances du modèle, qui est une version améliorée, plus fluide et plus puissante, des outils de visualisation du projet précédent (voir \autoref{sec:gmsnn_track})~;
	\item un système de sauvegarde et d'intéruption de l'entraînement similaire à celui du projet précédent (voir \autoref{subsec:gmsnn_save})
\end{itemize}

\subsection{Premiers résultats}
Les premiers résultats du modèle sont encourageants, autant sur la rapidité du modèle que sur ses performance.
Ils sont disponibles dans l'annexe \ref{anx:first}.
