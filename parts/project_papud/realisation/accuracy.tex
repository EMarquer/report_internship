\section{Changement de métrique pour évaluer la qualité du modèle}
Cette partie du projet n'est pas dédie à une optimisation, mais à la rectification d'une métrique inadaptée.

Jusqu'à cette étape, les performance étaient évaluées à partir du score utilisé pour mettre à jour le modèle. Ce score est difficile à utiliser pour se faire une idée réelle des performance du modèle. % TODO add ref formule

Il à donc été remplacé par une métrique nommée \og précision\fg{}, qui est le taux de caractères correctement prédits (le bon caractère au bon endroit).

\subsection{Précision de base}
Ce changement de métrique à été l'occasion de déterminer ce que l'on appelle la précision de base (\foreign{baseline accuracy} en anglais).

Cette valeur représente la précision d'un modèle qui répondrait toujours la même chose.
Elle permet de déterminer si le modèle produit apprend réellement de l'information, et dans quelle mesure.

Le détail de ces résultats est disponible dans les rapports des annexes \ref{anx:meeting} et \ref{anx:info}.
