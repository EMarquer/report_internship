\section{Changement d'unité de mesure pour évaluer la qualité du modèle}
Cette partie du projet n'est pas dédie à une optimisation, mais à la rectification d'une unité de mesure inadaptée.

Jusqu'à cette étape, les performance étaient évaluées à partir du score utilisé pour mettre à jour le \gls{model}. Ce score est difficile à utiliser pour se faire une idée réelle des performance du modèle. 

Il à donc été remplacé par une mesure nommée \og précision\fg{}, qui correspond au taux de caractères correctement prédits (le bon caractère au bon endroit).

\subsection{Précision de base}
Ce changement d'unité de mesure à été l'occasion de déterminer ce que l'on appelle la précision de base (\foreign{baseline accuracy} en anglais).

Cette valeur représente la précision d'un \gls{model} qui répondrait toujours la même chose. Elle sert de base de comparaison pour les performances du modèle.
Cela permet de déterminer si le \gls{model} produit apprend réellement \gls{model} l'information, et dans quelle mesure.

Le détail de ces résultats est disponible dans les rapports des annexes \ref{anx:info} (\autopageref{anx:info}) et \ref{anx:meeting} (\autopageref{anx:meeting}).
