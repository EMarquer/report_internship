\section{Optimisation du taux d'apprentissage}
La troisième optimisation mise en place pour le modèle et le réglage d'un des paramètres de l'algorithme d'apprentissage, est une valeur nommée \og taux d'apprentissage\fg{}.

Ce paramètre permet de déterminer la vitesse d'apprentissage du modèle.
Cependant, un mauvais réglage entraîne des conséquences catastrophiques sur le modèle, comme un apprentissage extrêmement lent, voir une divergence de l'apprentissage.

Il est donc nécessaire de correctement choisir ce paramètre.
La procédure classique en \gls{dl} repose sur l'essai-erreur~: on entraîne le modèle avec différentes valeurs pour le taux d'apprentissage, et on choisi celles qui à donné le meilleur résultat pour les entraînements à venir.

Un module permettant la détermination du taux d'apprentissage idéal à donc été implémenté, et utilisé.

La nouvelle valeur définie pour le à permit d'augmenter largement la vitesse de convergence du modèle. % TODO donner valeur

Le détail du processus de choix et des résultats est disponible dans les annexes \ref{anx:lr_1} et \ref{anx:lr_2}.
