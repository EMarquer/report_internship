\section{Mise en place d'un système de gestion de corpus plus puissant}\label{sec:papud_mulitiqueue}
La dernière optimisation est la transformation du système de gestion de corpus en une version plus puissante. Le nouvel outil créé a été intégralement doté de tests unitaires et testé.

\subsection{Fichiers multiples}
L'étude des \glspl{data} disponibles a révélé une structure en nombreux fichiers successifs.
Ainsi, à la demande du maître de stage, nous avons donné au nouvel outil la capacité d'utiliser plusieurs fichiers comme les fragments d'une unique source de \glspl{data}.
Cette fonctionnalité a été optimisée quant au temps de calcul et à l'utilisation de mémoire.

\subsection{Prétraitement modulaire}
En outre, durant le déroulement du projet, nous avons remarqué que l'ajout ou la modification d'une étape de \gls{preprocessing} était complexe avec l'ancien outil.
Pour simplifier les modifications futures, nous avons développé un système de \gls{preprocessing} modulaire, dans lequel chaque étape du traitement est séparée et interchangeable.
Cette architecture permet d'ajouter ou de retirer à volonté des étapes au traitement.
C'est d'ailleurs lors de la réalisation de ces modules que le \gls{preprocessing} mentionné \autoref{sec:spike} (\autopageref{sec:spike}) et \autoref{hex} (\autopageref{hex}) a été intégré.

\subsection{Processus multiples}
Une autre amélioration, tirant profit de l'organisation modulaire du traitement, est l'utilisation de multiples processus en parallèle.
Chacun d'entre eux est dédié à une étape du \gls{preprocessing}.
Ceci permet de répartir la charge de travail sur les différents processus, à la façon d'un travail à la chaîne.
Ainsi, l'outil est beaucoup plus rapide que dans sa version précédente.

\subsection{Performances}
Le nouvel outil est largement plus rapide et efficace que son prédécesseur.
De plus, il augmente la facilité d'utilisation et d'entretien.
Il reste cependant plus lent que la version adaptée aux petits fichiers.

La description des performances et le détail du fonctionnement de l'outil sont présentés dans le rapport de l'annexe \ref{anx:multi_process} (\autopageref{anx:multi_process}).
\pagebreak