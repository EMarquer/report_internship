\section{Mise en place d'un système de gestion de corpus plus puissant}
La dernière optimisation mise en place est la transformation du système de corpus en une version plus puissante.

\subsection{Fichiers multiples}
L'étude des données disponibles a révélé une structure en nombreux fichiers successifs.
Ainsi, à la demande de notre maître de stage, nous avons donné au nouvel outil la capacité à utiliser plusieurs fichiers comme source de données continue.
Cette fonctionnalité à été implémentée de façon optimisée en temps de calcul et en espace optimisé.

\subsection{Prétraitement modulaire}
D'autre part, durant le déroulement du projet nous avons remarqué que l'ajout d'une étape de prétraitement était ardue avec l'ancien outil.
En gardant cela à l'esprit, nous avons développé un système de prétraitement modulaire, dans lequel chaque étape du traitement est séparée et interchangeable.
Cette architecture permet d'ajouter ou de retirer à volonté des étapes au traitement.
C'est d'ailleurs lors de l'implémentation de ces modules que le prétraitement mentionnée \autoref{} et \autoref{} à été intégré.

\subsection{Processus multiples}
Une autre amélioration, tirant profit de l'aspect modulaire du traitement, est l'utilisation de multiples processus en parallèle.
Chacun d'entre eux est dédié à une étape du prétraitement.
Cela permet de répartir la charge de travail sur les différents processus, à la façon d'un travail à la chaîne.
Ainsi, l'outil est beaucoup plus rapide que la version précédente.

\subsection{Performances}
Le nouvel outil est largement plus rapide et efficace que son prédécesseur.
De plus, il augmente la facilité d'utilisation et d'entretient.

La description des performances et du fonctionnement en détail de l'outil sont disponibles dans le rapport de l'annexe \ref{anx:multi_process}.
