\section{Adaptation du système de gestion de corpus au volume de données}
L'étape suivant l'implémentation du modèle fonctionnel à été l'adaptation du système de gestion de corpus aux grands volumes de données à traiter dans le futur.

Pour cela, un premier prototype à été réalisé.
Il utilise des mécanismes optimisés du langage Python pour la lecture de fichier.
L'objectif de ce prototype est de maintenir uniquement une fiable quantité de données en mémoire.

Le principe de fonctionnement est simple~:
\begin{enumerate}
	\item on charge en mémoire une portion des données~; \label{itm:1}
	\item on applique le prétraitement sur ces données~;
	\item on transfert les données traitées à l'algorythme d'entrainement~;
	\item on recommence de l'étape \ref{itm:1} avec la portion suivante des données.
\end{enumerate}
\vspace{1em}

L'intégration de cet outil s'est faite en parallèle de la suite du projet.