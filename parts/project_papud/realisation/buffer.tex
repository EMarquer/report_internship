\section{Adaptation du système de gestion de corpus au volume de données}
Nous avons ensuite adapté le système de gestion de corpus aux grands volumes de \glspl{data} à traiter dans le futur.

Pour cela, un premier prototype a été réalisé.
Conçu pour limiter la quantité de \glspl{data} en mémoire,
il utilisait des mécanismes optimisés du langage Python pour la lecture de fichier, couplés à un principe de fonctionnement itératif simple.

\begin{enumerate}
	\item Premièrement, on charge en mémoire une portion des \glspl{data}. \label{itm:1}
	\item Ensuite, on applique le \gls{preprocessing} sur ces \glspl{data}.
	\item On transfère les \glspl{data} traitées à l'algorithme d'\gls{training}.
	\item Enfin, on recommence à partir de l'étape \ref{itm:1} avec la portion suivante des \glspl{data}.
\end{enumerate}
\vspace{1em}

L'intégration de cet outil à l'algorithme d'\gls{training} s'est faite simultanément à la suite du projet.