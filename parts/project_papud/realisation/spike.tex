\section{Analyse d'une variation anormale de performance dans l'apprentissage} \label{sec:spike}
La seconde optimisation apportée au modèle était dédiée à la réduction des sources d'erreur dans les données.

Dans les courbes d'apprentissage du modèle basique, d'importantes baisses de performances sont visible, toujours sur la même portion des données.

Il a été décidé que ce phénomène devait être analysé avant de poursuivre les travaux.

L'étude du phénomène a révélé une forte présence de codes hexadécimaux (comme \lstinline|0x005f| ou \lstinline|4A6D005F|) dans le fragment des données concerné, codes qui semblaient être la cause de la baisse de performance.

Il a donc été décidé, comme décrit dans la \autoref{hex} (\autopageref{hex}), d'intégrer au prétraitement des données la gestion de ces codes.
Cette opération a été intégrée durant la réalisation de la nouvelle version du gestionnaire de données, décrite \autoref{sec:papud_mulitiqueue} (\autopageref{sec:papud_mulitiqueue}).

Le déroulement de l'analyse et les conclusions détaillées sont rapportés dans l'annexe \ref{anx:spike} (\autopageref{anx:spike}).
