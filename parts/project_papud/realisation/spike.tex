\section{Analyse d'une étrange variation de performance dans l'apprentissage} \label{sec:spike}
La seconde optimisation apportée au modèle était dédiée à réduire les sources d'erreur dans les données.

Dans les courbes d'apprentissage du modèle basique, d'étranges baisses de performances sont visible, toujours sur la même portion des données.

Il à été décidé de consacrer du temps à l'analyse de ce phénomène.

L'étude du phénomène à révélé une forte présence de codes hexadécimaux (comme \lstinline|0x005f| ou \lstinline|4A6D005F|) dans le fragment des données mis en cause, codes qui semblent être la cause de la baisse de performance.

Il à donc été décidé, comme décrit \autoref{hex}, d'intégrer au prétraitement des données la gestion de ces codes.
Cette intégration à cependant dù attendre la fin de la réalisation de la nouvelle version du gestionnaire de données (décrite ??). % TODO add ref

Le déroulement de l'analyse et les conclusions détaillées sont disponibles dans l'annexe \ref{anx:spike}.
