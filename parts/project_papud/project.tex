\chapter{\Glsentrytext{nn} pour la prédiction de pannes\label{ch:project_papud}}
\section{Contexte}
Nous avons vu que, parmi les secteurs d'activité de \gls{bull}, l'utilisation de serveurs et autres systèmes de traitement de gros volumes de \glspl{data} sont très présents.

Ces machines tombent rarement en panne mais, quand ils le font, cela occasionne des pertes très importantes pour l'entreprise.

Il serait donc très intéressant de mettre au point un système de prédiction des pannes, afin de pouvoir les gérer.

Les \glspl{data} disponibles pour remplir cette tâche sont des \glspl{log}  de très grande taille (ils sont décrits en détail dans le \autoref{ch:data_papud}, \autopageref{ch:data_papud}).
Ils contiennent de nombreuses informations sur les évènements se déroulant dans les machines.

\section{Solution}\label{sec:solution}
%\textit{D'après les documents de travail officiels (en particulier le fichier README.md du dépôt de code officiel du projet).} % TODO remove dat shit

Le cas d'utilisation \gls{bull} du \glsname{project_papud} est destiné à répondre à cette problématique, en fournissant un système détectant les signes de panne dans les \glspl{log}.

Pour cela, il a été décidé de modéliser le comportement normal (sans panne) reflété dans ces journaux.

Ils sont composés de lignes de textes en anglais. Il est donc possible de produire un \gls{lm} capable de prédire la prochaine ligne.
Par la suite, le modèle sera augmenté d'un système prenant en compte le contexte de la ligne à prédire pour améliorer sa précision.
Par contexte, on entend ici des dépendances avec des lignes plus anciennes que la ligne seule précédente.

Ainsi, le plan général des opérations est composé de 2 parties.
\begin{enumerate}
	\item Initialement, on suppose que la structure en dépendances entre les lignes est simplissime~: une ligne dépend uniquement de la ligne précédente~; on cherche donc à établir un \gls{lm} capable de modéliser au mieux cette dépendance.
	\item Une fois le modèle simple établi, on abandonne le postulat précédent, et on cherche à établir, à partir du modèle créé, un modèle capable de modéliser des dépendances à la fois plus complexes et sur plus d'une ligne de distance.
\end{enumerate}
\hspace{1em}

Pour ce qui est du modèle simple, il a été décidé de ne pas utiliser de \gls{rnn}, bien trop lent pour la quantité de \glspl{data} à traiter. On y préférera un \gls{nn} basique.

On peut noter que les conclusions du \gls{project_gmsnn} ont été appliquées, autant pour le déroulement du projet que pour le type de modèle à utiliser.


\section{\Glsentrytext{project_papud}}
La tâche qui nous a été confiée est la réalisation du modèle simple.

Plus exactement, étant donné qu'il était évident que la durée restante du stage serait insuffisante pour réaliser le modèle simple et l'optimiser au mieux, nos objectifs étaient la réalisation d'un prototype du modèle, et la mise en place des outils nécessaires à l'entraînement.
Ceux-ci sont principalement les outils de gestion et de \gls{preprocessing} des \glspl{data}, les outils d'évaluation des performances du modèle, et l'algorithme d'entraînement du modèle.

La description des caractéristiques du modèle est disponible dans le \autoref{ch:papud_model} (\autopageref{ch:papud_model}).

Par la suite, nous désignerons ce projet par \og \gls{project_papud}\fg{}.

\section{Organisation du travail}
Contrairement au \gls{project_gmsnn}, d'autres membres participaient à cette équipe de projet (voir \autoref{sec:papud_colabo}, \autopageref{sec:papud_colabo}).
Étant, avec le maître de stage, les seuls habitués à manipuler des réseaux de neurones, nous avons travaillé individuellement durant ce projet.

Le \gls{project_gmsnn} s'étant bien déroulé, une organisation similaire a été mise en place afin de tenir ces autres membres de l'équipe de projet informés de l'avancement et des conclusions de notre travail.
Plus exactement, nous avons continué de rédiger des rapports d'avancement et nous avons présenté la progression du travail au cours de réunions hebdomadaires.

Les rapports de ce projet sont également disponibles en annexe (voir l'annexe \ref{anx:papud}, \autopageref{anx:papud}).
Le rapport d'une des réunion est disponible pour l'exemple à l'annexe \ref{anx:meeting} (\autopageref{anx:meeting}). % TODO enlever ?

\pagebreak
\subsection{Organisation initiale du travail}
Le \gls{project_papud} s'est déroulé sur la base de cycles de développement.
C'est durant les réunions hebdomadaires que les objectifs intermédiaires étaient fixés.

En effet, contrairement au \gls{project_gmsnn} pour lequel il a été simple de définir des périodes réservées aux grandes phases du projet, ce \gls{project_papud} s'est déroulé dans un temps très court.

%\pagebreak
Cependant, il a été possible de définir les délivrables suivants, classés par priorité décroissante~:
\begin{enumerate}
	\item la réalisation d'un prototype fonctionnel~;
	\item la mise en place d'un algorithme d'entraînement basique~;
	\item la mise en place de moyens d'évaluer le modèle et l'obtention de premières performances~;
	\item le \gls{preprocessing} des \glspl{data} et la préparation de la gestion des très gros volumes de \glspl{data} à venir.
\end{enumerate}
\hspace{1em}
%
De plus, si du temps supplémentaire était disponible, il serait dédié à l'amélioration des performances du prototype.

Une extension de la durée du stage d'une semaine a été décidée, pour augmenter le temps dédié au projet, en considérant les disponibilités de chaque membre de l'équipe de projet.

La durée totale de travail sur le projet a donc été de 5 semaines.

\subsection{Déroulement du projet}
Tous les délivrables ont été fournis, et deux améliorations notables des performances ont été mises en place.

La répartition du travail du projet est représentée dans la figure \ref{fig:papud_time} (\autopageref{fig:papud_time}). Une case de la figure correspond à un jour de travail.

\begin{figure}[ht]
	\centering
	\definecolor{red1}{RGB}{195,0,0}
\definecolor{red2}{RGB}{246,136,93}
\definecolor{yellow1}{RGB}{247,175,47}
\definecolor{yellow2}{RGB}{255,192,96}
\definecolor{yellow3}{RGB}{255,255,96}
\definecolor{green1}{RGB}{214,249,121}
\definecolor{green2}{RGB}{113,158,65}

% vertical separation between timeline and text boxes
\def\TextShift{15pt}

\tikzset{
	myrect/.style={
		rectangle split, 
		rectangle split horizontal,
		rectangle split parts=#1,
		draw,
		anchor=west,
		minimum height=2 em,
		inner sep=.42 em,
	},
	mytext/.style={
		arrow box,
		draw=#1!70!black,
		fill=#1,
		align=center,
		line width=0pt,
		font=\sffamily
	},
	mytextb/.style={
		mytext=#1,
		anchor=north,
		arrow box arrows={north:0.5cm}  
	},
	mytexta/.style={
		mytext=#1,
		anchor=south,
		arrow box arrows={south:0.5cm}  
	}
}

\newcommand\AddTextA[4][]{
	\node[mytexta=#2,#1] at #3 {#4};
}
\newcommand\AddTextB[4][]{
	\node[mytextb=#2,#1] at #3 {#4};
}
\newcommand\AddText[5][]{
	\if#5l\relax
	\node[mytextb=#2,yshift=-\TextShift,#1] 
	at (part#4.south west) {\strut#3\strut};
	\fi
	\if#5L\relax
	\node[mytexta=#2,yshift=\TextShift,#1] 
	at (part#4.north west) {\strut#3\strut};
	\fi
	\if#5m\relax
	\node[mytextb=#2,yshift=-\TextShift,#1] 
	at ( $ (part#4.south west)!0.5!(part#4.south east) $ ) {\strut#3\strut};
	\fi
	\if#5M\relax
	\node[mytexta=#2,yshift=\TextShift,#1] 
	at ( $ (part#4.north west)!0.5!(part#4.north east) $ ) {\strut#3\strut};
	\fi
	\if#5r\relax
	\node[mytextb=#2,yshift=-\TextShift,#1] 
	at (part#4.south east) {\strut#3\strut};
	\fi
	\if#5R\relax
	\node[mytexta=#2,yshift=\TextShift,#1] 
	at (part#4.north east) {\strut#3\strut};
	\fi
}

\newcommand\TimeLine[1]{%
	\coordinate (part0);  
	\foreach \Longitud/\Color/\Texto [count=\ti] in {#1}
	{
		\node[
		myrect=\Longitud,
		fill=\Color,
		draw=\Color!70!black,
		right=of part\the\numexpr\ti-1\relax
		] 
		(part\ti)
		{};
		\node (upper\the\numexpr\ti-1\relax) at  ($(part\ti.west) + (0,-2em)$) {};
		\node (lower\the\numexpr\ti-1\relax) at  ($(part\ti.west) + (0,2em)$) {};
		\node[text width=\Longitud em,text centered]
		at (part\ti.center) {\baselineskip=10pt\Texto\par};  
		\gdef\lastpart{\ti}
	}
	\node (upper\lastpart) at  ($(part\lastpart.east) + (0,-2em)$) {};
	\node (lower\lastpart) at  ($(part\lastpart.east) + (0,2em)$) {};
	
%	\foreach \Nodo in {1,...,3}
%	{
%		\ifodd\Nodo\relax
%		\draw[decoration={brace,amplitude=4pt,mirror},decorate] 
%		(lower\Nodo) -- (lower\the\numexpr\Nodo-1\relax);
%		\else
%		\draw[decoration={brace,amplitude=4pt},decorate,minimum height=5pt] 
%		(upper\Nodo) -- (upper\the\numexpr\Nodo-1\relax);
%		\fi    
%	}
}


\colorlet{col1}{yellow2}
\colorlet{col2}{yellow3}
\colorlet{col3}{green!30!}
\colorlet{col4}{blue!30!}
\colorlet{col5}{purple!30!}


\newenvironment{timeline}[1][]
{\begin{tikzpicture}[node distance=0pt and -\pgflinewidth,#1]}
{\end{tikzpicture}}

\begin{timeline}
\TimeLine{%
	1/col1/{1},%
	4/col2/{4 jours},%
	10/col3/{10 jours},%
	6/col4/{6 jours},%
	4/col5/{4 jours}
  }

%\AddText[text=white]{red!60!black}{D\'{e}but \\ du projet}{1}{l}
\AddText{col1!50!}{Choix du \\ mod\`{e}le}{1}{M}
\AddText{col2!50!}{Impl\'{e}mentation \\ du prototype}{2}{m}
%\AddText{col2!50!}{Premiers r\'{e}sultats}{2}{R}
\AddText{col3!50!}{Am\'{e}lioration du prototype : \\recherche du taux  d'apprentissage, \\ analyse du pic de performance,  \dots}{3}{M}
\AddText{col4!50!}{Corpus multi-processus}{4}{m}
\AddText{col5!50!}{Documentation}{5}{M}
\AddText[text=white]{purple!60!black}{D\'{e}but de la \\r\'{e}daction\\ du rapport}{5}{r}

% \AddText[text=white]{red1}{Initial \\ meeting}{2}{L}
% \AddText{red2}{List \\ property}{2}{m}
% \AddText{yellow1}{Listing \\ period}{3}{M}
% \AddText{yellow2}{Offer \\ received}{4}{L}
% \AddText{yellow2}{Offer \\ signed}{4}{m}
% \AddText{yellow3}{File under \\ review}{5}{M}
% \AddText[xshift=-3pt]{green1}{Negotiator \\ assigned}{6}{L}
% \AddText{green1}{Offer in final \\ review}{6}{m}
% \AddText[xshift=3pt]{green2}{Short sale\\ approved}{7}{L}
% \AddText{green2}{Under \\ contract}{7}{m}
% \AddText[text=white]{green2!60!black}{Vacate \& \\ close}{7}{R}
\end{timeline}\caption[Répartition du travail sur le projet PAPUD]{Répartition du travail sur le projet PAPUD}\label{fig:papud_time}
\end{figure}

\section{Outils}
\subsection{Gestion du code}
Le code et les rapports ont été gérés avec le système de gestion de version Git, et ont été stockés sur les serveurs Gitlab de l'\gls{inria}. Les membres de l'équipe de projet ont accès à l'ensemble de ces \glspl{data}, qui ont été mises à jours tout au long du projet.

\subsection{Python, PyTorch et Grid5000}
Nous avons décidé de continuer à travailler avec Python, PyTorch et Grid5000, pour trois raisons principales~:
\begin{itemize}
	\item la première partie du projet s'est déroulée avec ces outils, nous étions donc habitués à leur utilisation et à leurs spécificités~;
	\item la base de code accumulée jusque là reposait sur ces outils, et pouvait être réutilisée sans trop d'efforts~;
	\item il était possible de basculer ultérieurement vers une autre librairie que  PyTorch, cette dernière possédant une fonctionnalité de conversion d'un modèle vers les autres principales librairies d'\gls{dl}.
\end{itemize}