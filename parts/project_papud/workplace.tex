\chapter{Présentation du \glsentryfirst{project_papud}}
La seconde partie du stage s'intègre dans le \gls{project_papud}{projet \gls{itea3}-\gls{papud}, en particulier dans le cas d'utilisation \gls{bull}}. Nous verrons en détail les objectifs du projet dans la \autoref{sec:projet:papud} (\autopageref{sec:projet:papud}).

Le \glsfirst{project_papud} est un projet de l'initiative \gls{itea3} du réseau \gls{eureka}.

\section{Collaborateurs}\label{sec:papud_colabo}
Les personnes avec lesquelles nous avons collaboré durant ce projet sont notre maître de stage M. Christophe Cerisara, ainsi que deux autres chercheurs de l'équipe \gls{synalp}, Mme. Nadia Bellalem et M. Samuel Cruz-Lara.
Une quatrième chercheuse, Mme. Christine Fay-Varnier, nous à rejoint vers la fin du stage.

\section{\Glsentrytext{eureka} et \glsentrytext{itea3}}
\og\gls{eureka} est une initiative européenne, intergouvernementale, destinée à renforcer la compétitivité de l’industrie européenne.\fg{} D'après Wikipedia \autocite{wiki_eureka}.

\gls{itea3} est la troisième itération d'un programme du réseau \gls{eureka} nommé \glsfirst{itea3}.
ITEA est un programme de recherche, développement et innovation basé sur un partenariat public / privé, et fonctionnant par appels de projet.
Ces appels à projets se concentrent sur des problématiques des technologies de l'information et de la communication, et ce dans une perspective industrielle.

\gls{itea3} implique plus de 40 pays, ainsi que de nombreuses entreprises.

\section{\Glsentrytext{project_papud} et cas d'utilisation \glsentrytext{bull}}
C'est lors de la troisième vague d'appels à projets d'\gls{itea3} que le  \gls{project_papud} à été accepté.

L'objectif du projet \glsfirst{papud} est l'élaboration d'une série d'outils basés sur les techniques de l'\gls{dl}.
La plateforme ainsi produite à pour objectif l'analyse des volumes de données devenus stop grand pour être gérés de façon traditionnelle.
Ainsi, le projet \gls{papud} s'inscrit dans la dynamique d'\gls{itea3}.

\begin{table}[h]{
	\centering
	\renewcommand{\arraystretch}{1.5}
	\setlength\tabcolsep{1em}
	\begin{tabularx}{\textwidth}{|X|l|}
		\hline
		Nom complet du projet & 16037 PAPUD\\
		\hline
		Période de réalisation & Janvier 2018 - Décembre 2020 (3~ans)\\
		\hline
		Appel à projet & ITEA 3 Call 3\\
		\hline
		Partenaires & 16\\
		\hline
		Coûts estimés & 10 927 000 €\\
		\hline
		Volume de travail estimé\newline (en personne.année) & 151,88 \\
		\hline
		Pays participants & Belgique, Espagne, France, \mbox{Roumanie}, Turquie\\
		\hline
	\end{tabularx}
	\renewcommand{\arraystretch}{1}}
	\caption[Informations générales sur le \gls{project_papud}, d'après le site de \gls{itea3}]{Informations générales sur le \gls{project_papud}, d'après le site de \gls{itea3} \autocite{about_papud} \label{tab:about_papud}}
\end{table}

\section{\Glsentrytext{bull}}
\subsection*{Présentation de l'entreprise}
\gls{bull} est une entreprise française spécialisée dans la sécurité informatique et la gestion des gros volumes de données informatiques. 

L'entreprise à été rachetée en 2014 par le groupe \gls{atos}.

\subsection*{Secteurs d'activité}
D'après {le site d'\gls{atos}\autocite{bull_produits}}, les activités principales de la filiale \gls{bull} sont:
\begin{itemize}
	\item le matériel informatique et logiciel professionnel de haute sécurité;
	\item le matériel informatique et logiciel pour l'Armée et la Défense, y compris du matériel de navigation maritime et terrestre;
	\item les serveurs de calcul et de stockage, les \gls{data centers} et les solutions nuagiques (\gls{cloud});
	\item les solutions de calcul haute performance (les \og supercalculateurs\fg{});
	\item les systèmes intégrés, à savoir du matériel informatique spécifique intégré à un produit, comme par exemple l'ordinateur de bord intégré dans une voiture.
\end{itemize}
\vspace{1em}

Globalement, \gls{bull} concentre ses activités sur le matériel informatique et les logiciels de pointe en matière de sécurité et de fiabilité.
Les gammes de produits \gls{bull} s'adressent principalement à des grosses entreprises et aux états.

\section*{Pour en savoir plus}
Des informations plus détaillées sur le \gls{project_papud} sont disponible sur la page web du projet \autocite{about_papud}.