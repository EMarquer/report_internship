\chapter{Présentation d'ITEA, du projet PAPUD, et de Bull}
La seconde partie du stage s'intègre dans le \gls{project_papud}, en particulier dans le cas d'utilisation de \gls{bull}. Ce projet fait partie de l'initiative \gls{itea3} du réseau \gls{eureka}.
%
Nous verrons en détail les objectifs du projet dans le \autoref{ch:project_papud} (\autopageref{ch:project_papud}).

%Le \glsfirst{project_papud} est un projet de l'initiative \gls{itea3} du réseau \gls{eureka}.

\section{Équipe de projet}\label{sec:papud_colabo}
Les personnes avec lesquelles nous avons travaillé durant ce projet sont M. Christophe Cerisara, le maître de stage, ainsi que deux autres chercheurs de l'équipe \gls{synalp}, \mbox{Mme~Nadia~Bellalem} et \mbox{M.~Samuel~Cruz-Lara}.
Une quatrième chercheuse de la même équipe nous a rejoint vers la fin du stage, \mbox{Mme~Christine~Fay-Varnier}.

\section{\Glsentrytext{eureka} et \glsentrytext{itea3}}
\og\gls{eureka} est une initiative européenne, intergouvernementale, destinée à renforcer la compétitivité de l’industrie européenne\fg{}, d'après Wikipédia~\autocite{wiki_eureka}\footnote{Ces informations ont été confirmées par les pages dédiées à \gls{eureka} sur le site internet de la Commission Européenne~\autocite{ce_eureka} et celui du gouvernement français~\autocite{fr_eureka}, ainsi que par le site internet de \gls{eureka}~\autocite{eureka}.}.

\gls{itea3} est la troisième itération d'un programme du réseau \gls{eureka} nommé \glsfirst{itea3}.
ITEA est un programme de recherche, développement et innovation basé sur un partenariat public-privé, et fonctionnant par appels à projet.
Ces appels à projet se concentrent sur des problématiques des technologies de l'information et de la communication, et ce dans une perspective industrielle.

\gls{itea3} implique plus de 40 pays, ainsi que de nombreuses entreprises.

\pagebreak
\section{\Glsentrytext{project_papud} et cas d'utilisation de \glsentrytext{bull}}
C'est lors de la troisième vague d'appels à projets d'\gls{itea3} que le  \gls{project_papud} a été accepté.

L'objectif du \gls{project_papud} (\foreign{Profiling and Analysis Platform Using Deep Learning}, littéralement \og plateforme de profilage et d'analyse utilisant l'\gls{dl}\fg{}) est l'élaboration d'une série d'outils basés sur les techniques de l'\gls{dl}.
La plateforme ainsi produite a pour objectif l'analyse des volumes de données devenus trop grands pour être gérés de façon traditionnelle.
Ainsi, le \gls{project_papud} s'inscrit dans la dynamique d'\gls{itea3}, tout comme dans la thématique de notre stage.

Le cas d'utilisation de \gls{bull} du \gls{project_papud} est la partie du projet destinée à répondre aux besoins de l'entreprise \gls{bull} (décrits dans le \autoref{ch:project_papud} (\autopageref{ch:project_papud})).

\begin{table}[h]{
	\centering
	\renewcommand{\arraystretch}{1.5}
	\setlength\tabcolsep{1em}
	\begin{tabularx}{\textwidth}{|X|l|}
		\hline
		Nom complet du projet & 16037 PAPUD\\
		\hline
		Période de réalisation & Janvier 2018 - Décembre 2020 (3~ans)\\
		\hline
		Appel à projet & ITEA 3 Call 3\\
		\hline
		Partenaires & 16\\
		\hline
		Coûts estimés & 10 927 000 €\\
		\hline
		Volume de travail estimé\newline (en personne.année) & 151,88 \\
		\hline
		Pays participants & Belgique, Espagne, France, \mbox{Roumanie}, Turquie\\
		\hline
	\end{tabularx}
	\renewcommand{\arraystretch}{1}}

	{\footnotesize D'après le site de \gls{itea3} \autocite{about_papud}} \label{tab:about_papud}
	\caption[Informations générales sur le \glsentryname{project_papud}]{Informations générales sur le \gls{project_papud}}
\end{table}

\section{\Glsentrytext{bull}}
\subsection{Présentation de l'entreprise}
\gls{bull} est une entreprise française spécialisée dans la sécurité informatique et la gestion des gros volumes de données informatiques. 

L'entreprise a été rachetée en 2014 par le groupe \gls{atos}.

\pagebreak
\subsection{Secteurs d'activité} \label{def:cloud} \label{def:data centers}
Les activités principales de la filiale \gls{bull} sont~\autocite{bull_produits}~:
\begin{itemize}
	\item le matériel informatique et logiciel professionnel de haute sécurité;
	\item le matériel informatique et logiciel pour l'Armée et la Défense, y compris le matériel de navigation maritime et terrestre;
	\item les serveurs de calcul et de stockage, les \gls{data centers} et les solutions nuagiques (\gls{cloud});
	\item les solutions de calcul haute performance (les \og supercalculateurs\fg{});
	\item les systèmes intégrés, à savoir du matériel informatique spécifique intégré à un produit, comme par exemple l'ordinateur de bord intégré dans une voiture.
\end{itemize}
\vspace{1em}

Globalement, \gls{bull} concentre ses activités sur le matériel informatique et les logiciels de pointe en matière de sécurité et de fiabilité.
Les gammes de produits \gls{bull} s'adressent principalement à des grandes entreprises et aux administrations publiques.

\section*{Pour en savoir plus}
Des informations plus détaillées sur le \gls{project_papud} sont disponibles sur la page du projet sur le site internet d'\gls{itea3}~\autocite{about_papud}.