\chapter{Réalisation}
% l’ordre de présentation le plus logique est souvent l’ordre chronologique dans lequel les tâches ont été accomplies. Il faut présenter les points forts en distinguant nettement l’existant et la plus-value apportée par le stagiaire. Si des solutions ont été envisagées, mais non retenues, il peut être intéressant de les présenter en expliquant pourquoi elles ont été abandonnées. C’est toute la chaîne, de la prise de connaissance du problème à la solution apportée, qui doit être déroulée.

% Cette partie débouche nécessairement sur des résultats qu’il faut énoncer en précisant leurs exploitations actuelles ou à venir. Il faut annoncer clairement ce qui a été réalisé ou non par rapport à la mission de départ. Cette partie peut être complétée par des propositions personnelles du stagiaire pour prolonger son travail, et même par des critiques positives de son environnement de travail.

% Tentative plan
%5.5. define model
%5.6. preparing "papud" (import corpus, tracker, training) \& basic model
%5.7. accuracy (mention)(with tracker ?)
%5.8. understanding strange spike comportment
%5.9. multiqueue modular (implemented after some time)
%5.10. lr find
%5.11. doc

\section{Définition du modèle}

La toute première étape du projet a été de définir l'architecture à utiliser (décrite \autoref{ch:papud_model}, \autopageref{ch:papud_model}). C'est sous la forme de réunions que cette étape s'est déroulée.

Tout d'abord, nous avons élaboré le modèle avec le maître de stage.
Par la suite, nous avons présenté nos conclusions aux autres membres du projet, qui les ont approuvées.
%\clearpage
\section{Implémentation du modèle et transfert des outils de base}
L'étape suivante a été l'implémentation d'une version de base du \gls{model}, ainsi que des outils nécessaire pour l'utiliser.

Ces outils sont~:
\begin{itemize}
	\item un outil de prétraitement et de manipulation de \gls{corpus}~;
	\item un algorithme d'\gls{training} sur ces données, élaboré lors du projet précédent~;
	\item un outil d'analyse et de traçage des performances du modèle, qui est une version améliorée, plus fluide et plus puissante, des outils de visualisation du projet précédent (voir \autoref{sec:gmsnn_track}, \autopageref{sec:gmsnn_track})~;
	\item un système de sauvegarde et d'interruption de l'entraînement, similaire à celui du projet précédent (voir \autoref{subsec:gmsnn_save}, \autopageref{subsec:gmsnn_save})
\end{itemize}
\vspace{1em}

Les premiers résultats du modèle sont encourageants, autant sur sa rapidité que sur ses performances.
Ils sont présentés dans l'annexe \ref{anx:first} (\autopageref{anx:first}).
%\clearpage % Implément
\section{Adaptation du système de gestion de corpus au volume de données}
L'étape suivant l'implémentation du modèle a été l'adaptation du système de gestion de corpus aux grands volumes de \glspl{data} à traiter dans le futur.

Pour cela, un premier prototype a été réalisé.
Il utilisait des mécanismes optimisés du langage Python pour la lecture de fichier.
L'objectif de ce prototype est de maintenir uniquement une faible quantité de \glspl{data} en mémoire.

Le principe de fonctionnement est simple~:
\begin{enumerate}
	\item on charge en mémoire une portion des \glspl{data}~; \label{itm:1}
	\item on applique le \gls{preprocessing} sur ces \glspl{data}~;
	\item on transfert les \glspl{data} traitées à l'algorithme d'\gls{training}~;
	\item on recommence de l'étape \ref{itm:1} avec la portion suivante des \glspl{data}.
\end{enumerate}
\vspace{1em}

L'intégration de cet outil à l'algorithme d'\gls{training} s'est faite en parallèle de la suite du projet.%\clearpage % Implément
\section{Entrainement par batch}
La première optimisation du modèle à été l'intégration de l'algorithme d'entraînement par \foreign{batches} (voir ???). % TODO add ref

Il à été nécessaire de définir le nombre de \foreign{batches} optimal en terme de temps de calcul et de performance. Les détails concernant ce choix sont disponibles dans l'annexe \ref{anx:batch}.

Cette optimisation à permis d'accélérer considérablement l'entraînement du modèle.%\clearpage % Opti 1
\section{Changement de métrique pour évaluer la qualité du modèle}
Cette partie du projet n'est pas dédie à une optimisation, mais à la rectification d'une métrique inadaptée.

Jusqu'à cette étape, les performance étaient évaluées à partir du score utilisé pour mettre à jour le modèle. Ce score est difficile à utiliser pour se faire une idée réelle des performance du modèle. 

Il à donc été remplacé par une métrique nommée \og précision\fg{}, qui est le taux de caractères correctement prédits (le bon caractère au bon endroit).

\subsection{Précision de base}
Ce changement de métrique à été l'occasion de déterminer ce que l'on appelle la précision de base (\foreign{baseline accuracy} en anglais).

Cette valeur représente la précision d'un modèle qui répondrait toujours la même chose.
Elle permet de déterminer si le modèle produit apprend réellement de l'information, et dans quelle mesure.

Le détail de ces résultats est disponible dans les rapports des annexes \ref{anx:info} et \ref{anx:meeting}.
%\clearpage % Implément
\section{Analyse d'une variation anormale de performance dans l'apprentissage} \label{sec:spike}
La seconde optimisation apportée au modèle était dédiée à la réduction des sources d'erreur dans les données.

Dans les courbes d'apprentissage du modèle basique, d'importantes baisses de performances sont visibles, toujours sur la même portion des données.

Il a été décidé que ce phénomène devait être analysé avant de poursuivre les travaux.

L'étude du phénomène a révélé une forte présence de codes hexadécimaux (comme \lstinline|0x005f| ou \lstinline|4A6D005F|) dans le fragment des données concerné, codes qui semblaient être la cause de la baisse de performance.

Il a donc été décidé, comme décrit dans la \autoref{hex} (\autopageref{hex}), d'intégrer au prétraitement des données la gestion de ces codes.
Cette opération à eu lieu durant la réalisation de la nouvelle version du gestionnaire de données, décrite \autoref{sec:papud_mulitiqueue} (\autopageref{sec:papud_mulitiqueue}).

Le déroulement de l'analyse et les conclusions détaillées sont rapportés dans l'annexe \ref{anx:spike} (\autopageref{anx:spike}).
%\clearpage % Opti 2
%\section{Augmentation du nombre de paramètres} \label{sec:layers}

La troisième optimisation mise en place est l'augmentation du nombre de paramètres.

Comme décrit ???, une des options pour réaliser cette optimisation est l'augmentation du nombre de couches du modèle.%TODO add ref
Les performances
%\clearpage % Opti 3, echec
\section{Optimisation du taux d'apprentissage}\label{lr_opti_papud}
La troisième optimisation mise en place porte sur le réglage d'un des paramètres de l'algorithme d'apprentissage, est une valeur nommée \og taux d'apprentissage\fg{}. Il s'agit d'une optimisation courante en \gls{dl} \autocite{LearningRateOptimisation}.

Ce paramètre permet de déterminer la vitesse d'apprentissage du modèle.
Un mauvais réglage peut entraîner des conséquences catastrophiques sur le modèle, comme un apprentissage extrêmement lent, voire la divergence de l'apprentissage\footnote{On parle de divergence quand la performance du \gls{model} empire au long de l'apprentissage. Ce terme est utilisé par opposition à la convergence souhaitée des performances vers le meilleur score possible.}.

Il est donc nécessaire de fixer correctement ce paramètre.
Les algorithmes que nous avons étudiés reposent sur des séries d'\glspl{training} brefs~: on entraîne le modèle avec différentes valeurs du taux d'apprentissage, et on conserve celle qui a donné le meilleur résultat pour les entraînements à venir.

Un outil permettant la détermination du taux d'apprentissage idéal a donc été implémenté, et utilisé.

La nouvelle valeur définie a permis d'augmenter considérablement la vitesse d'amélioration de la qualité du modèle. En effet, il fallait précédemment plus de 400 époques pour atteindre la qualité maximale du modèle, contre moins de 100 époques avec le nouveau taux d'apprentissage. % TODO donner valeur

Le détail du processus de choix et des résultats est présenté dans les annexes \ref{anx:lr_1} (\autopageref{anx:lr_1}) et \ref{anx:lr_2} (\autopageref{anx:lr_2}).
%\clearpage % Opti 4
\section{Mise en place d'un système de gestion de corpus plus puissant}\label{sec:papud_mulitiqueue}
La dernière optimisation mise en place est la transformation du système de corpus en une version plus puissante. Le nouvel outil créé à été intégralement doté de tests unitaires et testé.

\subsection{Fichiers multiples}
L'étude des \glspl{data} disponibles a révélé une structure en nombreux fichiers successifs.
Ainsi, à la demande de notre maître de stage, nous avons donné au nouvel outil la capacité à utiliser plusieurs fichiers comme source de \glspl{data} continue.
Cette fonctionnalité à été implémentée de façon optimisée en temps de calcul et en utilisation mémoire.

\subsection{Prétraitement modulaire}
D'autre part, durant le déroulement du projet nous avons remarqué que l'ajout ou la modification d'une étape de \gls{preprocessing} était ardue avec l'ancien outil.
En gardant cela à l'esprit, nous avons développé un système de \gls{preprocessing} modulaire, dans lequel chaque étape du traitement est séparée et interchangeable.
Cette architecture permet d'ajouter ou de retirer à volonté des étapes au traitement.
C'est d'ailleurs lors de l'implémentation de ces modules que le \gls{preprocessing} mentionnée \autoref{sec:spike} (\autopageref{sec:spike}) et \autoref{hex} (\autopageref{hex}) à été intégré.

\subsection{Processus multiples}
Une autre amélioration, tirant profit de l'aspect modulaire du traitement, est l'utilisation de multiples processus en parallèle.
Chacun d'entre eux est dédié à une étape du \gls{preprocessing}.
Cela permet de répartir la charge de travail sur les différents processus, à la façon d'un travail à la chaîne.
Ainsi, l'outil est beaucoup plus rapide que la version précédente.

\subsection{Performances}
Le nouvel outil est largement plus rapide et efficace que son prédécesseur.
De plus, il augmente la facilité d'utilisation et d'entretien.
Il reste cependant plus lent que la version adaptée aux petits fichiers.

La description des performances et le détail du fonctionnement de l'outil sont disponibles dans le rapport de l'annexe \ref{anx:multi_process} (\autopageref{anx:multi_process}).
\pagebreak%\clearpage % Opti 5
%\section{Documentation du code}
Pour ce projet, un soin tout particulier à été porté à la documentation du code.

%TODO transférer 
La raison est que le projet que nous avons réalisé est la première pierre du projet PAPUD cas d'utilisation BULL.
Ainsi, nous avons laissé une base de code propre et intégralement documentée pour notre successeur sur le projet.

Un des documents réalisé est disponible dans l'annexe \ref{anx:info}.%\clearpage

% TODO  Nota: layers mis de coté, save&load idem

