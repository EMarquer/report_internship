
\usepackage{longtable}
\usepackage{caption}
\usepackage{pdfpages}
\usepackage{subcaption}
\usepackage[defaultlines=4,all]{nowidow}

\usepackage{tikz}
\usepackage{pgf}
\usepackage{pgfplots}
\usepackage{amsmath}
\usepackage[normalem]{ulem}
\usepackage{float}

\usepackage{makeidx}
\usepackage{titletoc}
\usepackage{tocloft}
\usepackage[nottoc]{tocbibind}
\usepackage[backend=biber,backref=true,hyperref=true,sorting=none,citestyle=numeric,]{biblatex}%,sorting=nyt, autocite=footnote
\usepackage{csquotes}
\usepackage{etoc}
\usepackage[xindy,acronym,section,style=altlist,nopostdot]{glossaries}%nonumberlist%altlist%long3colheader%nopostdot
\setacronymstyle{long-short-desc}
\usepackage{array}
\usepackage{tabularx}

\renewcommand{\seename}{voir aussi}
\newcommand{\foreign}[1]{\emph{#1}}

\renewcommand*{\glossaryentrynumbers}[1]{\space p. #1}

\newcommand*{\defref}[1]{Défini \autoref{#1} (\autopageref{#1}).}

\newcommand{\newglos}[3]
{%
	% The glossary entry the acronym links to   
	\newglossaryentry{#1}{name={#2}, text={#2},
		description={#3}}}

\newcommand{\newglosacr}[4]
{%
	% The glossary entry the acronym links to   
	\newglossaryentry{#1}{name={#2}, text={#2},
		description={#3}}
%
	% Define the acronym and use the see= option
	\newglossaryentry{#1-}{type=\acronymtype, name={#2}, text={#2},
	description={}, first={#4\glsadd{#1}}, see=[Glossaire~:]{#1}}}

\newcommand{\printixgls}
{\printnoidxglossary[type=entity]
	\printnoidxglossary[type=\acronymtype]
	\printnoidxglossary[type=main]}

%%%%%%%%%%%%%%%%%%%%%%%%%%%%%%%%%%%%%%%%%%%%%%%%%%%%%%%%%%%%%%%%%%%%%%%%%%%%%%%%
\usepackage{listings}
\usepackage{textcase}
%\lstinputlisting{source_filename.py}
%\lstinputlisting[language=Python]{source_filename.py}
%\lstinputlisting[language=Python, firstline=37, lastline=45]{source_filename.py}

\makeatletter
\providecommand\phantomsection{}
\newcommand{\listsname}{Listes des tables, des figures et des fragments de code}
\newcommand{\lists}{
	\clearpage
	\chapter{\listsname}
	\setcounter{tocdepth}{2}
	\section*{\listtablename}
	\@starttoc{lot}
	\section*{\listfigurename}
	\@starttoc{lof}
	\section*{\lstlistlistingname}
	\@starttoc{lol}
}

\newcommand{\termsname}{Glossaire, accronymes et noms d'entités}
\newcommand{\terms}{
	\clearpage
	\chapter{\termsname}
	\printnoidxglossary[type=main]
	\newpage
	\printnoidxglossary[type=\acronymtype]
	\newpage
	\printnoidxglossary[type=entity]
}
\makeatother

\newcommand{\setlistingname}[3]{%
	\renewcommand\lstlistingname{#1}
	\renewcommand\lstlistlistingname{#2}
	\def\lstlistingautorefname{#3}
}

\setlistingname{Fragment de code}{Liste des fragments de code}{Fragment de code}

%%%%%%%%%%%%%%%%%%%%%%%%%%%%%%%%%%%%%%%%%%%%%%%%%%%%%%%%%%%%%%%%%%%%%%%%%%%%%%%%

\usepackage{tocloft}
\renewcommand{\cftdot}{.}
\renewcommand{\cftpartleader}{\cftdotfill{\cftdotsep}} % for parts
\renewcommand{\cftchapleader}{\cftdotfill{\cftdotsep}}

%%%%%%%%%%%%%%%%%%%%%%%%%%%%%%%%%%%%%%%%%%%%%%%%%%%%%%%%%%%%%%%%%%%%%%%%%%%%%%%%
\interfootnotelinepenalty=10000

%%%%%%%%%%%%%%%%%%%%%%%%%%%%%%%%%%%%%%%%%%%%%%%%%%%%%%%%%%%%%%%%%%%%%%%%%%%%%%%%
\newenvironment{report}[1]{\section{#1}
	\hrule}{\newpage}


%%%%%%%%%%%%%%%%%%%%%%%%%%%%%%%%%%%%%%%%%%%%%%%%%%%%%%%%%%%%%%%%%%%%%%%%%%%%%%%%
\def\medpoint{\textperiodcentered}